\documentclass{article}
\usepackage[utf8]{inputenc}
\begin{document}

\section{Referat mandag d. 26. februar - SW2-405a}
\subsection{Gennemgang/kritik af arbejdsblade.}

Overordnet er vi på det rigtige spor.

Der er lidt problemer med l\&l. Pilene peger ind på et informationssystem. Så vi skal tænke over hvad er det der skal laves? Hvis man ikke har fundet ud af det kan l\&l blive meget bred. Vi skal have et nyt afsnit 2.1, der skal handle om administration og interview. Det kan hjælpe med at indskrænke inden l\&l. l\&l hjælper med at gøre det relevant, men er ikke godt til at finde ud af hvad for et system der skal laves. Så kan der komme mere fokus it-system som vi kan lave analyse på.

Vi skal have skrevet mere om hvad vi får ud af at lave l\&l .

Initierende problem. Sæt de to forslag sammen: “hvilke problem … og er der mulighed for at forbedre …”

Indledning er meget brede. Lidt skudt forbi (Legepladser, lukket havneområde). Vend tilbage til dette afsnit senere senere.

Organisation: Under regler og love skal kan vi skrive mere om standardreglement. Hvad gør politiet f.eks.?

Mennesker: Brian savner en interessant analyse.

Generelt: Vi skal ikke bruge “man”, det er talesprog og bliver meget upræcist.

Der skal ændres lidt på “Vestre Baadelaug vil som grundregel undgå passive medlemmer, det værende medlemmer uden båd”

Det er bedre at bruge direkte citat end at ligge ord i munden.

Afsnittet om gæstesejlere er ikke færdig. Der skal vi havde flere informationer.

E-conomic: Det skal flyttes op til ny 2.1, så kan vi konkludere at det ser ud til at det er velunderstøttet

Kilderne er gode, hvis der bliver citeret direkte, han har ingen problemer med dem som de er nu. Interview på CD som tekst. Så gemmer vi selv lydfiler, skriv at lyd filer kan findes ved .

\subsection{Snak om retningsvalget}

Simon fortæller om det. Brian: “ja, det er vigtig at få styr på hvad havnefogeden, Hvad gør han hvis der ikke er pladser. Har han syge afløser. Kender han marinabooking?”

Brian sender pdf med rettelser på mail.

Næste møde 8:30 næste mandag.
\end{document}