\documentclass{article}
\usepackage[utf8]{inputenc}
\usepackage[T1]{fontenc}
\begin{document}

\section{3-03-2014}
Kasper starter med kort at give status på hvor vi er med projektet.

\subsection{Gennemgang/kritik af arbejdsblade}
2.0 skal laves til en subsection. Hvorfor har vi lavet et interview? Til sidst skal der konkluderes på vores afgrænsning. Så ved vi hvad vi skal fokusere på. Der kan godt være længere, vi kan sætte afsnittet om navision ind.

Menneske er alle der bliver påvirket af systemet. Både primære og secundære.

Intressant skal være en del af mennesker. Brian vil putte den ind i starten af mennesker. Forklare hvad menneskernes interresse i havnen er.

Organisation kan have mere om love. Hvem bestemmer i kluberne, evt et diagram.

Teknologi, der skal vi se på nogle af de teknologier der skal bruges til løsningen. Hvilke muligheder er der med Internet of Things og Cyber-physical system.

Refleksioner om Problem: Det handler for meget om gæster. Vi skal også tænke på hvad medlemmerne og alle andre får ud af det.

Kan man evt. lave et system hvor man kan se hvem der er i havnen.

Når man skriver fordele kan vi lave det på punktform.

Peter og Dorthe “afviste alle” vs “kan ikke se”. I Rapporten må der godt bruges mere direkte citat, så man ikke liger ord i munden. Citer evt. bagside afkuverter.

Arbejde videre men en lidt klarer vision.

\subsection{Opfølgning på retningsvalget (onsdagsmødet)}
Vi skal havde lavet en brainstorm over hvad der skal understøttes. Vi skal selv finde på hvad fremtidens havne systems skal kunne.

\subsection{Eventuelt}
Ting der mangler inden statusseminaret:
\begin{enumerate}
    \item En liste over functionaliteter.
    \item Løsningsforslag, så vi let kan komme over i løsnings fasen.
    \item Send en udtask til problemformulering
    \item Ellers mangler der ikke store ting bare skrive lidt mere til de afsnit som Brian talte om.
\end{enumerate}

\subsection{Aftale næste vejledermøde}
Næste vejledermøde aftales til statusseminaret. Vi sender udkast til en problemformulering onsdag middag.

\end{document}