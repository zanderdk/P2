\documentclass{article}
\usepackage[utf8]{inputenc}
\begin{document}

\section{Referat mandag d. 10. februar - SW2-405a}

\subsection{Præsentation af foreløbig arbejde}

Gruppen starter med at præsentere foreløbende brainstorm og laudon \& laudon model over projektet.

Gruppen fortæller at de har taget kontakt med aalborg sejlforening. De har aftalt at gruppen kan se hvordan deres nuværende system fungerer. 

\subsection{Problemanalyse og program}

Gruppen spørger om et interview med en bådklub er tilstrækkeligt til en problemanalyse. Vejleder svarer at det kommer an på hvor generelt programmet skal være. Vejlederen siger at man sagtens kan lave et problem ud fra interviewet med en enkelt bådklub.

Omkring specifikation af programmet, skal der være enkelte punkter der dækker programmet. Specifikationerne behøver ikke at være så formel.

Gruppen spørger hvorvidt 1 kontaktperson er nok til problemanalysen. Vejlederen svarer at det vil være relevant at snakke med flere personer, gerne over flere dage, så eventuelle spørgsmål kan besvares over flere omgange. 

Gruppen spørger om det ville være smart at koordinere interview med 2 andre P2 grupper med samme emne. Vejlederen svarer at dette ville være godt, da de andre grupper kan have andre indgangsvinkler på emnet, og desuden skåner et samarbejde også kontaktpersonen.

Vejlederen fortæller at man ikke behøver at være innovativ mht. programmet. Det gælder om at lave et velfungerende program, der opfylder behovene.

Omkring modellering, fortæller vejleder at modellering mht. studieordningen, passer meget godt i forhold til objekt orienteret programmering, hvilket er paradigmet for programmeringssproget C\# vi benytter i projektet. F.eks. vil en lejer og en lejekontrakt være naturlige objekter i programmeringen. Herunder nedarvning af f.eks. forskellige bådtyper med forskellige egenskaber.

Gruppen spørger hvornår det ville være fordelagtigt at påbegynde programmeringen af projektet. Vejleder svarer at det vil være tilstrækkeligt at begynde på programmeringen omkring status seminar. Han fortæller dog også at man sagtens kan begynde at implementere nogle funktioner i programmet som dette falder gruppen ind. Gruppen skal dog være opmærksom på at noget af dette skal skrottes senere hen.

Før bådudlejningen kan modelleres, skal man udforske virkeligheden.

Gruppen spørger om der er nogle begrænsninger i forhold til brugen af C-sharp. Vejlederen påpeger at det vigtigste er at kernefunktioner i programmet skal virke. Dvs. er et grafisk interface sekundært. Der ligger meget programmering i håndteringen af grafiske elementer, hvilket ligger til grund for valget af kernefunktioner som primær aktivitet.

Vejlederen påminder gruppen om at OOP bliver relativt begrebsrigt, og det er derfor vigtigt at gruppen er opmærksom på at følge med i kurset.

\subsection{Rapport sprog}

Gruppen spørger om vejlederen holdning til sproget i rapporten. Vejlederen fortæller at det vil gøre gruppens arbejde lettere, hvis rapporten bliver skrevet på dansk i forhold til engelsk. Han anbefaler derfor at gruppen skriver rapporten på dansk, men påpeger at dette valg selvfølgelig er op til gruppen.


\subsection{Samarbejde mellem vejleder}

Omkring samarbejdet mellem vejlederen og gruppen, snakkes der løst om 1 møde om ugen. Inden hvert vejledermøde vil vejlederen gerne have en dagsorden. Hvis vejlederen skal læse noget af gruppens arbejde, skal dette materiale sendes flere dage før, så vejlederen har tid til at læse dette. Derudover skal sendt materiale være gennemlæst for stavefejl og grammatiske fejl. Der aftales at der ingen vejledermøder er ved kursusgange. 





Vejlederen er enig i brugen af Laudon \& Laudon.

Fra d. 7-14. marts er vejlederen ude at rejse. 19-23 maj er vejlederen ligeledes ude at rejse.

\end{document}
