%!TEX root = Master.tex
\documentclass[a4paper,12pt,twoside,openright]{memoir}

\newcommand{\subject}{Programmering og Problemløsning}

%%%% PACKAGES %%%%

\usepackage[T1]{fontenc}
\usepackage[utf8]{inputenc}
\usepackage{fourier}
\usepackage[danish]{babel}
\usepackage{csquotes}
\usepackage[table]{xcolor}
\usepackage{microtype}
\usepackage{enumitem}
\usepackage[short]{datetime}
\usepackage{adjustbox}
\usepackage{amsmath}
\usepackage{amsthm}
\newtheorem{mydef}{Definition}
\usepackage{graphicx}
\graphicspath{{images/}}
\usepackage{tikz}
\usetikzlibrary{shapes.geometric,arrows,shapes.misc}
\usepackage{subcaption}
\usepackage{tikz}
\usepackage{verbatim}
\usepackage{listings}
\lstdefinestyle{customc}{
  belowcaptionskip=1\baselineskip,
    breaklines=true,
    frame=L,
    xleftmargin=\parindent,
    language=C,
    showstringspaces=false,
    basicstyle=\footnotesize\ttfamily,
    keywordstyle=\bfseries\color{green!40!black},
    numbers=left,
    commentstyle=\itshape\color{purple!40!black},
    identifierstyle=\color{blue},
    stringstyle=\color{orange},
}

\lstset{
  basicstyle=\ttfamily,
    numbers=left,
    columns=fullflexible,
    showstringspaces=false,
    commentstyle=\color{gray}\upshape
}

\lstdefinelanguage{XML}
{
  basicstyle=\footnotesize\ttfamily\color{blue}\bfseries,
    morestring=[b]",
    morestring=[s]{>}{<},
    morecomment=[s]{<?}{?>},
    numbers=left,
    stringstyle=\color{black},
    identifierstyle=\color{blue},
    keywordstyle=\color{cyan},
    morekeywords={xmlns,version,type}% list your attributes here
}

% stop floats from floating path the next subsection
\usepackage{placeins}
\makeatletter
\AtBeginDocument{%
  \expandafter\renewcommand\expandafter\section\expandafter
  {\expandafter\@fb@secFB\section}%
  \newcommand\@fb@secFB{\FloatBarrier
    \gdef\@fb@afterHHook{\@fb@topbarrier \gdef\@fb@afterHHook{}}}%
    \g@addto@macro\@afterheading{\@fb@afterHHook}%
    \gdef\@fb@afterHHook{}%
}
\makeatother

\makeatletter
\AtBeginDocument{%
  \expandafter\renewcommand\expandafter\subsection\expandafter
  {\expandafter\@fb@subsecFB\subsection}%
  \newcommand\@fb@subsecFB{\FloatBarrier
    \gdef\@fb@afterHHook{\@fb@topbarrier \gdef\@fb@afterHHook{}}}%
    \g@addto@macro\@afterheading{\@fb@afterHHook}%
    \gdef\@fb@afterHHook{}%
}
\makeatother

\makeatletter
\AtBeginDocument{%
  \expandafter\renewcommand\expandafter\subsubsection\expandafter
  {\expandafter\@fb@subsubsecFB\subsubsection}%
  \newcommand\@fb@subsubsecFB{\FloatBarrier
    \gdef\@fb@afterHHook{\@fb@topbarrier \gdef\@fb@afterHHook{}}}%
    \g@addto@macro\@afterheading{\@fb@afterHHook}%
    \gdef\@fb@afterHHook{}%
}
\makeatother

\makeatletter
\AtBeginDocument{%
  \expandafter\renewcommand\expandafter\paragraph\expandafter
  {\expandafter\@fb@parFB\paragraph}%
  \newcommand\@fb@parFB{\FloatBarrier
    \gdef\@fb@afterHHook{\@fb@topbarrier \gdef\@fb@afterHHook{}}}%
    \g@addto@macro\@afterheading{\@fb@afterHHook}%
    \gdef\@fb@afterHHook{}%
}
\makeatother

%Algorithm
\usepackage{epstopdf}
\usepackage[linesnumbered,ruled,vlined]{algorithm2e}
\usepackage{algorithmicx}
\usepackage[noend]{algpseudocode}

% support for sidste side page number
\usepackage{totpages}
\usepackage{intcalc}
\usepackage{refcount}


% ¤¤ Referencer, bibtex og url'er ¤¤ %
\usepackage[hyphens]{url}

\usepackage[english]{varioref}
\usepackage[colorlinks]{hyperref} 
\hypersetup{                            % Opsætning af hyperlinks
  colorlinks = false,
             linkcolor = black,
             anchorcolor = black,
             citecolor = black,
             breaklinks = true,
			 pdfkeywords = {havn, bådklub, administration, databasehåndtering},
             pdfsubject = {\subject}
}
\usepackage{cleveref}						% Giver flere bedre mulighed for at lave krydshenvisninger
%\usepackage{natbib}											% Litteraturliste med forfatter-år og nummererede referencer


\usepackage[marginclue,draft,inline,mode=multiuser,silent]{fixme}
\FXRegisterAuthor{si}{asi}{Simon}
\FXRegisterAuthor{mi}{ami}{Mikkel}
\FXRegisterAuthor{ka}{aka}{Kasper}
\FXRegisterAuthor{fr}{afr}{Frederik}
\FXRegisterAuthor{al}{aal}{Alexander}

\newcommand{\appendixpages}[0]{
  \intcalcAdd
  {1}
  {
    \intcalcSub
    {\getpagerefnumber{TotPages}}
    {\getpagerefnumber{appendix_start}}
  }
}
\newcommand{\numberpages}[0]{
  \intcalcSub{\getpagerefnumber{TotPages}}{\appendixpages}
}

%%%% CUSTOM SETTINGS %%%%

% ¤¤ Marginer ¤¤ %
\setlrmarginsandblock{3.5cm}{2.5cm}{*}	% \setlrmarginsandblock{Indbinding}{Kant}{Ratio}
\setulmarginsandblock{2.5cm}{3.0cm}{*}	% \setulmarginsandblock{Top}{Bund}{Ratio}
\checkandfixthelayout 									% Laver forskellige beregninger og sætter de almindelige længder op til brug ikke memoir pakker

%	¤¤ Afsnitsformatering ¤¤ %
\setlength{\parindent}{0mm}           	% Størrelse af indryk
\setlength{\parskip}{3mm}          			% Afstand mellem afsnit ved brug af double Enter
\linespread{1,2}												% Linie afstand
% length of abstract indentations
\setlength{\absleftindent}{0mm}
\setlength{\absrightindent}{0mm} 



\usepackage[
bibencoding=utf8,
  hyperref=true,
  backend=biber,
  sorting=none
  ]{biblatex}
  \addbibresource{kilder.bib}

  % ¤¤ Indholdsfortegnelse ¤¤ %
\setsecnumdepth{subsection}		 			% Dybden af nummerede overkrifter (part/chapter/section/subsection)
  %\maxsecnumdepth{subsubsection}					% Ændring af dokumentklassens grænse for nummereringsdybde
  \settocdepth{subsection} 								% Dybden af indholdsfortegnelsen

  \pagestyle{plain}

  % ¤¤ Kapiteludssende ¤¤ %

  \chapterstyle{ger}
  \setlength{\afterchapskip}{0pt}
  \setlength{\beforechapskip}{0pt}

