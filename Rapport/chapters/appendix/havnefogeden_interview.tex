(00:30) Per Jensen, går hele sommeren på havnen. Står for alt fra rengøring til maling, og græsslåning. Mange timer, 7 dage om ugen. Stor del er service af gæster.

(01.20) Hvordan blev du havnefoged? Baggrund som sælger, uddannet maskinesnedker. erfaring med mekanik. Ingen særlig bådinteresse. Et af kravene for ansøgning er uafhængig.

(03.00) Ansat af Vestre Baadelaug. Arbejde på broer bliver faktureret til ANF.

(03.50) Arbejdsdage i vinteren er meget lidt. Mest on-call. I sommer sæson: Start med morgenrunde, hvor der sættes flag, tømmes skrald, og tjekke gæster. I højsæson er rengøring også en stor del. Hverdag skal toiletter rengøres. Service til gæstesejlere, med spørgsmål. Huller fyldes ud med småting, som græsslåning. Tid til overs går til maling og andre lign.

Fortæller ting til generalforsamling. Men har selv mulighed for at bestille f.eks. håndværkere. Der er afsat beløber, men de daglige bliver overladt til havnefogedens vurdering.

(07.18) Hvad er forholdet til medlemmerne i klubben? Alle kender havnefogeden, og han kender de fleste. Nogen tror at de er hans 'chef'.

(08.00) Skridt for skridt behandling af gæster? Gæster klare som regel sig selv. Hvis et klubmedlem forlader havnen natten over, så skal de vende deres skilt. 80% af gæsterne trækker selv biletter i automaten. Fogeden går en tur om aftenen, og tjekker biletterne, og kræver op de steder hvor der ikke er blevet betalt. Havnefogeden gør sig synlig i dagtimerne, så han er tilgængelig for spørgsmål og anden hjælp.

(08.45) Hvordan håndterer havnefogeden gæster? Gæsterne får et bånd de skal sætte på deres båd. Som udgangspunkt klarer gæsterne sig selv. Bånd skifter farve for dato. Der er mulighed for at købe bånd af havnefogeden.

(09.50) Havnefogeden kender bådene på standere (flag), som alle i klubben skal have. Hvis de ikke har sådan en, så ved han at de er gæster.Der burde også stå hjemstavnshavn bag på både. 

(10.50) Kunne havnefogeden tænke sig et andet IT-system? Nej, han ser en fordel i at tage gåturen rundt. Det vil han skyde på er lige så nemt. 

(11.30) Hvor lang tid kunne havnen klare sig uden fogeden i højsæsonen? Havnen kunne sagtens klare sig. Medlemmer i klubben kan godt fylde ud i korte perioder.

(12.10) Der kommer cirka 4000 overnatninger i løbet af juni,juli,august. men allerede når vejret begynder at blive godt. Omkring påsken, og i hellidage. Generelt stor udskiftning.

(13.00) Er der problemer med gæsterne som tager andre pladser? Nej, da fogeden blive ringet op og kan vende skilte i god nok tid. Indforstået regel imellem sejlere, at man altid har ret til sin hjemplads.

(14.00) Har i pladsmangel? Meget få situationer. Mulighed for at ligge på, på ydersiden af både.

(15.00) Kontakt imellem medlemmer og fogeden? det er både over mail, og telefon, men også meget personlig kontakt, fordi fogeden er hernede alligevel.

(15.40) Der er imellem 450 og 500 pladser, i begge havne tilsammen.

(16.00) Hvor mange medlemmer er ude og sejle pr. dag? svært at sige, men omkring 100.

(16.15) Fogeden holder styr på havnen, ved at have et generelt overblik. Han var selv skeptisk i starten, men synes nu at det er kommet meget naturligt.

(17.00) Det tog den første sæson at blive kendt med de indforståede regler som havnefoged. Ansættelse var et bundt nøgler og en besked på at du finder ud af resten hen ad vejen. Hvis du gør noget forkert, skal du nok få det at vide. Langt de fleste medlemmer er behjælpelige og positive.

(18.30) Bestyrelsen siger: Hvis folk ikke klager, så ser de det som at fogeden gør det godt.

(18.50) Fogeden har gået med en skridt tæller, og ligger på en sommerdag på omkrig 20 kilometers gang.

(19.15) Mødetid er fri, men omkring klokken 7. og så skal flaget tages ned kl. 20 om aftenen. Har mulighed for at forlade havnen i løbet af dagen.

(20.30) I løbet af sæsonen er havnefogeden on-call, så han skal være tilgængelig de meste af døgnet.

(21.20) Fogeden kender til systemer til registering, men tror at det vil tage for meget administrativt arbejde. Fogeden tror heller ikke at det vil give en tidsbesparelse.

(22.20) Mange af gæsterne kommer fra limfjorden. Nogle lidt ældre sejlere, vil heller ikke ud på de større vande. Ellers er der mange fra, Norge, Sverige, Tyskland, Holland. Der er også en del fra sjælland. 



