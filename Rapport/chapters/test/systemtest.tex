\section{Systemtest}
En systemtest af et stykke software er en test der udføres på et komplet system. Systemtest er med til at evaluere om systemet overholder sine specificerede krav. For at teste om løsningen lever op til de krav der tidligere er stillet, vil der i dette afsnit vil der blive demonstreret, hvordan udvalgte usecases kan udføres i programmet. De udvalgte usecases, er valgt på baggrund af deres evne til, at fremvise funktionaliteten i programmet. Demonstrationen er opstillet således, at der startes med en defination af inputtet, en potentiel antagelse, og en hypotese omkring hvilket output der fremkommer. Dernæst en gennemgang af gøremåden, og ud fra outputtet bedømmes resultatet som enten som en \enquote{succes} eller \enquote{fejl}. Dette er en test af hvorvidt programmet kan udføre de forudstillede usecases, og dertil leve op til de krav vi har stillet det.

\textbf{Databasen}
I databasen er der generet nogle specifikke medlemmer ud over de tilfældige der bliver lavet ved programmets kørsel.

Der er generet en havnefogeden med \enquote{medlemsnummer: 4} og \enquote{passwordet: havnefogedeVB1234}. Havnefogeden har skrive-tilladelse, som vil give ham fuld adgang til at tilføje/ændre/slette i databasen. 

Der er generet et medlem ved navn \enquote{Johanne Hoffmann}, båden \enquote{Den Usynkelige II}, medlemsnummer \enquote{5},  passwordet \enquote{Johanne4395} og telefonnummeret \enquote{22 95 41 87}. Derudover der genreret 2 andre medlemmer der hedder Johanne i databasen, Johanne Friis og Johanne Snoep. 

\textbf{usecase 1: Et medlem ønsker at forlade sin plads i mere end 24 timer.}
Johanne Hoffman vil gerne rejse væk den 8/10-2014 og komme hjem den 18/10-2014.
\begin{enumerate}
	\item Johanne klikker på \enquote{Medlemslogin} og \enquote{Loginscreen} kommer dernæst frem og afventer input.
	\item Johanne indtaster sit \enquote{medlemsnummer: 5} og \enquote{password: Johanne4395}.
	\item Johanne klikker \enquote{Login} og forsidefanen kommer frem.
	\item Johanne navigere til brugeradministrationsfanen, og information omkring hende kommer frem.
	\item Johanne klikker på \enquote{Tilføj Ny Rejse} og \enquote{Tilføj Rejse} bliver fremvist.
	\item Johanne vælger henholdsvis den 8/10-2014 og 18/10-2014.
	\item Johanne klikker OK, og kan nu se sin udrejse og hjemkomst i rejse feltet.
\end{enumerate}

\textbf{Forventning:} Johanne kan logge ind med sine login oplysninger, og hun kan melde sin rejse til systemet.
\textbf{Resultat:} Succes


\textbf{usecase 15: Havnefogeden vil gerne se hvornår et medlem vender tilbage til medlemmets plads.}
Det antages at Havnefogeden kun har \enquote{navnet: Johanne} og \enquote{telefonnummeret: 22 95 41 87}
\begin{enumerate}
	\item Havnefogeden klikker på \enquote{Medlemslogin} og \enquote{Loginscreen} frem vises.
	\item Havnefogeden indtaster sit \enquote{medlemsnummer: 4} og \enquote{password: havnefogedeVB1234}.
	\item Havnefogeden klikker \enquote{Login} og forsidefanen kommer frem.
	\item Havnefogeden navigere til søgfanen.
	\item Havnefogeden indtaster \enquote{navnet: Johanne} hvor der kommer tre hits frem.
	\item Havnefogeden indtaster dernæst \enquote{telefonnummeret: 22 95 41 87} og insnævre resultaterne til det rigtige medlem.
	\item Havnefogeden dobbeltklikker på medlemmet og får vist informationen omkring \enquote{medlemmet: Johanne Hoffmann}.
	\item Havnefogeden aflæser rejsedatoen i rejsefeltet.
\end{enumerate} 

\textbf{Forventing:} havnefogeden kan logge ind. Havnefogeden kan finde den rigtige Johanne ud fra de to stykker af information han havde omkring Johanne Hoffmann.
\textbf{Resultat:} Succes


\textbf{Kasser eller havnefogede vil tilføje ny båd til et medlem}
Det antages at Havnefogeden vil tilføje en ny båd til medlemmet Johanne Hoffmann. Det antages også at havnefogeden har samme oplysninger, som i foregående scenarie. Båden er 12m lang og 6.5m bred, den har \enquote{registreringsnummeret: 4785963}, den hedder \enquote{Eurika} og den har ingen plads.
\begin{enumerate}
	\item Havnefogeden klikker på \enquote{Medlemslogin} og \enquote{Loginscreen} fremvises.
	\item Havnefogeden indtaster sit medlemsnummer:\enquote{4} og passwordet:\enquote{havnefogedeVB1234}
	\item Havnefogeden klikker \enquote{Login} og forsidefanen kommer frem.
	\item Havnefogeden navigere til søgfanen.
	\item Havnefogeden indtaster navnet: \enquote{Johanne Hoffmann} hvor hun kommer frem.
	\item Havnefogeden dobbeltklikker på medlemmet og får vist information omkring medlemmet \enquote{Johanne Hoffmann}.
	\item Havnefogeden klikker på \enquote{Tilføj Ny Båd} og \enquote{NewBoatPopup} fremvises.
	\item Havnefogeden indtaster \enquote{navnet: Eurika}, \enquote{registreringsnummere: 4785963}, \enquote{længden: 12} og \enquote{bredden: 6,5}.
	\item Havnefogeden vælger at båden ingen plads har.
	\item Havnefogeden klikker Ok
	\item Havnefogeden kan nu se og inspicere båden Eurika.
\end{enumerate}

\textbf{Forventning:} havnefogeden kan logge ind og finde Johanne Hoffmann i søgefunktionen. Han kan også tilføje båden Eurika og dens  tilsvarende information.
\textbf{Resultat:} Succes.

% Dette er en test for det nuværende programs funktionalitet i forhold til de usescases, der er blevet opstillet for programmet.
% Ved Success forstås der at denne funktion er (delvis) mulig.
% Ved Implied forstås der at det er underforstået og/eller ude af programmets omfang.
% Ved Failure forstås der at denne funktion ikke er mulig i programmets nuværende stadie.
% \subsection{Medlemmer}


% \begin{enumerate}
	% \item{\bf{Medlem forlader sin plads i mere end 24 timer.}}
		% HANDLING: Medlemmet vælger \"tilføj\" under rejser, og vælger en afrejse og hjemkomst dato - Rejsen er nu gemt.
	  % \begin{enumerate}
			% \item Success -  a) Medlem melder til systemet et gyldigt tidspunkt for afrejse og hjemkomst.
			% \item Success -  b) Systemet melder til medlemmet at rejsen er registreret. 
			% \item Implied -  c) Systemet venter til efter tidspunktet for afrejse.
			% \item Failure -  d) Når pladsen tømmes, melder systemet at pladsen er fri.
	   % \end{enumerate}
			
	% \item{\bf{Alternativt medlem melder ikke afrejse til systemet.}}
	  % \begin{enumerate}
			% \item Failure -  aa) Systemet underetter havnefogeden om at en medlemsbåd har været væk fra havnen i mere end 24 timer
	   % \end{enumerate}
	   
	% \item{\bf{Alternativt medlem afrejser ikke på det meldte tidspunkt.}}
	  % \begin{enumerate}
			% \item Failure -  ba) Systemet underetter både havnefogeden samt medlemmet om uoverensstemmelsen.
	   % \end{enumerate}
	   
	% \item{\bf{Medlem vender tilbage til sin plads efter minimums 24 timers afrejse og der ligger en gæst på pladsen.}}
	  % \begin{enumerate}
			% \item Failure -  a) Systemet melder at pladsen er optaget.
	   % \end{enumerate}
	   
	% \item{\bf{Medlem vender ikke tilbage til sin plads på det meldte tidspunkt.}}
	  % \begin{enumerate}
			% \item Failure -  a) Systemet melder uoverensstemmelsen til havnefogeden
	   % \end{enumerate}

	% \item{\bf{Medlem ønsker at annullere en allerede anmeldt rejse.}}
		% HANDLING: Medlemmet logger ind og markere rejsen, dernæst klikker medlemmet på fjern - Rejsen er nu slettet
	  % \begin{enumerate}
			% \item Success -  a) Medlem vælger den pågældende rejse fra listen over registrerede rejser.
			% \item Success -  b) Medlem annullerer rejsen.
			% \item Success -  c) Systemet melder tilbage at rejsen er annulleret.
	   % \end{enumerate}

% \subsection{Gæster}
	% \item{\bf{Gæst er ankommet til havnen, og vil leje en plads.}}
		% HANDLING: Gæsten registrere sigselv i system og logger ind med udleveret chipkort. Gæsten kan nu se kortet over potentielt ledige pladser.
	  % \begin{enumerate}
			% \item Success -  a) Gæst registrere sig selv i systemet.
			% \item Failure -  b) Gæst benytter chipkortet til at logge ind.
			% \item Success -  c) Gæst finder oversigten over havnen og finder en ledig plads
			% \item Failure -  d) Gæst vælger den ledige plads og booker den.
			% \item Failure -  e) Gæst betaler for pladsen.
			% \item Failure -  f) Systemet melder til en at der er nye ankomne.
	   % \end{enumerate}
	   
	% \item{\bf{Alternativt: Gæsten finder ikke selv en plads.}}
	  % \begin{enumerate}
			% \item Success - a) Gæst registrere sig selv i systemet.
			% \item Failure - b) Gæst indtaster hans/hendes båds specifikationer.
			% \item Failure - c) Systemet returnerer en liste over pladser som vil passe hans behov.
			% \item Failure - d) Systemet videresender gæsten til betaling.
	   % \end{enumerate}

	% \item{\bf{Gæst vil se informationer om plads og betaling.}}
		% HANDLING: Gæsten indsætter chipkort og logger ind. I Brugeradministrations tabben vil information omkring gæsten samt hans båd og rejser befinde sig.
	  % \begin{enumerate}
			% \item Failure -  a) Gæst indsætter chipkort i automat.
			% \item Success -  b) systemet viser informationer vedrørende gæsten.
	   % \end{enumerate}
     
	% \item{\bf{Alternativt: Chipkort er bortkommet.}}
	  % \begin{enumerate}
			% \item Failure -  a) Gæster melder til systemet at chipkort er bortkommet.
			% \item Failure -  b) Systemet henviser gæsten til Havnefogeden.
	   % \end{enumerate}
     
	% \item{\bf{Gæst forlader havnen på eller før anmeldte afrejse tidspunkt.}}
	  % \begin{enumerate}
			% \item Failure -  a) Gæst aflever chipkort og får udleveret en kvittering.
			% \item Failure -  b) Gæst får returneret berettiget kapital.
	   % \end{enumerate}
    
	% \item{\bf{Alternativt: Gæst bliver liggende i havnen efter det anmeldte afrejse tidspunkt.}}
	  % \begin{enumerate}
			% \item Failure -  a) Systemet melder uoverensstemmelsen til havnefogeden.
			% \item Failure -  b) Chipkortet deaktiveres.
			% \item Failure -  c) Ved efterfølgende forsøg på brug af chipkort, henvendes der til havnefogeden.
	   % \end{enumerate}
			
	% \item{\bf{Gæst vil forlænge ophold.}}
		% HANDLING: Gæsten logger ind, og benytter ændre rejse funktionen. (Den opdateres ikke visuelt)
	  % \begin{enumerate}
			% \item Failure -  a) Gæst indsætter chipkort i automat.
			% \item Success -  b) Gæst melder ønskede ny afrejse dato til systemet.
			% \item Failure -  c) Systemet melder at den nye dato er registreret.
	   % \end{enumerate}
     
	% \item{\bf{Alternativt: Nye afrejse dato overlapper med medlemshjemkomst.}}
	  % \begin{enumerate}
			% \item Failure -  a) Systemet melder at datoerne overlapper, og foreslår ny plads.
	   % \end{enumerate}

% \subsection{Havnefoged}
	% \item{\bf{havnefogeden registrerer et medlemmets hjemkost.}}
		% HANDLING: Havnefogeden logger ind med hans login. Dernæst finder havnefogeden medlemmet og ændre datoen for hjemkomst i hans rejse.
	  % \begin{enumerate}
			% \item Implied -  a) Medlem meddeler tidlig hjemkomst til havnefogeden, for eksempel via telefon, mail eller lignende.
			% \item Success -  b) havnefogeden indtaster (ny) dato i systemet.
			% \item Success -  c) Systemet returnere at datoen er accepteret.
	   % \end{enumerate}

	% \item{\bf{Alternativt: En gæst har lejet pladsen for en længere periode.}}
	 % \begin{enumerate}
			% \item Failure -  a) Systemet foreslår en ny plads til gæsten.
			% \item Implied -  b) havnefogeden snakker med gæsten.
	   % \end{enumerate}
      
	% \item{\bf{havnefogeden vil gerne se hvilke pladser der er ledige.}}
		% HANDLING: Havnefogeden logger ind og klikker på kort tabben hvori der er et kort over havnen.
	  % \begin{enumerate}
			% \item Success -  a) havnefogeden åbner overbliksfunktionen i programmet.
			% \item Success -  b) havnefogeden kan nu se på et kort over havnen, hvilke pladser der er ledige.
	   % \end{enumerate}
      
	% \item{\bf{havnefogeden vil gerne se hvilke nye gæster der er ankommet indenfor et tidsrum.}}
	  % \begin{enumerate}
			% \item Failure -  a) havnefogeden åbner gæster funktionen i programmet.
			% \item Failure -  b) havnefogeden vælger det ønskede tidsrum.
			% \item Failure -  c) Programmet viser nye ankomne gæster fra det specificerede tidsrum.
	   % \end{enumerate}
  
	% \item{\bf{havnefogeden vil gerne se hvilke pladser der endnu ikke er betalt for.}}
	  % \begin{enumerate}
			% \item Failure -  a) havnefogeden åbner gæster funktionen i programmet.
			% \item Failure -  b) havnefogeden åbner ubetalte pladser.
			% \item Failure -  c) Programmet præsenterer listen af pladser, der endnu ikke er betalt for. Tiden for hvor lang tid pladsen har været ubetalt vises også.
	   % \end{enumerate}
	   
	% \item{\bf{havnefogeden vil gerne se hvornår et medlem vender tilbage til medlemmets plads.}}
	  % \begin{enumerate}
			% \item Success -  a) havnefogeden åbner medlems funktionen i programmet.
			% \item Success -  b) havnefogeden skriver i søgefeltet medlemmets navn eller medlemsnummer.
			% \item Success -  c) Programmet præsenterer en liste over matchende medlemmer.
			% \item Success -  d) havnefogeden vælger det søgte medlem.
			% \item Success -  e) Programmet viser alle kendte detaljer om medlemmet, herunder hvornår medlemmet forventes tilbage.
	   % \end{enumerate}
	   
	% \item{\bf{havnefogeden bliver advaret af en fejl ved en stander.}}
	  % \begin{enumerate}
			% \item Failure -  a) Uventet fejl opstår ved en stander på havnen, eller personen ved standeren trykker på hjælp.
			% \item Failure -  b) havnefogeden får besked om dette.
	   % \end{enumerate}

% \subsection{Andre}
	% \item{\bf{Automatisk arrangementsforslag.}}
	  % \begin{enumerate}
			% \item Failure -  a) Systemet melder på baggrund af aktiviteten i havnen, at det ville være en god dag for et arrangement.
			% \item Failure -  b) Et medlem af klubben ser denne melding, og foreslår et arrangement.
	   % \end{enumerate}
      
	% \item{\bf{Automatisk medlemstjek.}}
	  % \begin{enumerate}
			% \item Failure -  a) Systemet registrerer uoverensstemmelse med angivet rejseplan.
			% \item Failure -  b) Systemet notificerer havnefogeden omkring overensstemmelsen.
	   % \end{enumerate}
    
	% \item{\bf{Bestyrelsen eller havnefogeden vil se statistik over aktivitet i havnen.}}
	  % \begin{enumerate}
			% \item Failure -  a) Statistik funktionen åbnes og viser de ønskede statistikker.
	   % \end{enumerate}
			
	% \item{\bf{Kasser eller havnefogede vil tilføje ny båd til et medlem}}
	  % \begin{enumerate}
		% \item Success - a) havnefogeden logger ind i systemet med administrator rettigheder.
		% \item Success - b) Medlemmet bliver fundet via søge funktionen.
		% \item Success - c) havnefogeden tilføjer opretter en ny båd med pågældende information.
		% \item Success - d) Systemet melder tilbage at båden er registreret.
	   % \end{enumerate}

% \end{enumerate}
