\chapter{Konklusion}
\label{cha:konklusion}

Igennem projektets forløb, har programmets struktur ændret sig, efterhånden som gruppen fik indsigt i arkitektoniske designmønstre, som eksempelvis MVC. Ideerne fra disse mønstre har hjulpet med til en forbedring af programmets struktur, men også til at forstå vigtigheden af god softwarearkitektur.

En af målene med løsningen var at designe en model, der stemmer overens med en bådklubs administrative ansvarsområder indenfor både, bådpladser, medlemmer og gæster. Dette mål er blevet opfyldt tilfredsstillende, i betragtning af projektets begrænsede tidshorisont. En mere komplet løsning, ville inkludere en model for gæstebetaling og medlemskontingent.

Et andet mål med programmet var, at implementere en platform til formidling af information omkring havnen mellem havnefogeden, medlemmerne og gæsterne. Målet er opnået, da det er muligt for alle parter at se informationer om hinanden, såvidt de har de nødvendige tilladelser. Dette opnås ved hjælp af programmets kortfunktion samt søgefunktionen. En videreudvikling i denne retning, kunne indebære implementation af funktioner som realtids-beskeder.

Programmet er designet til at opfylde et konkret behov, men hvor det har været muligt, er en mere generel løsning implementeret. Et eksempel på en sådan løsning er programmets mulighed for opkobling til sensorer. Løsningen er designet til at være modulær, så forskellige sensorer kan benyttes i programmet.