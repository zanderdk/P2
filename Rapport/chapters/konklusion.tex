\chapter{Konklusion}
\label{cha:konklusion}

Igennem projektets forløb har programmets struktur ændret sig efterhånden som gruppen fik indsigt i arkitektoniske design mønstre som MVC. Idéerne fra disse mønstre har hjulpet med at forbedre programmets struktur, men også til at forstå vigtigheden i god software arkitektur.

En af målene med programmet var at designe en model der stemmer overens med en bådklubs administrative ansvarsområder indenfor både, bådpladser, medlemmer og gæster. Dette mål er blevet opfyldt tilfredsstillende, givet projektets begrænsede tidshorisont. En mere komplet løsning, ville inkludere en model for gæstebetaling og medlemskontingent.

Et andet mål med programmet var at implementere en platform til formidling af information omkring havnen, mellem havnefogeden, medlemmerne og gæsterne. Dette mål er delvist opnået. Det er muligt for alle parter at se informationer om hinanden, såvidt de har tilladelse til dette. Dette opnås ved hjælp af programmets kortfunktion, samt søgefunktionen. For at opnå dette mål i en højere grad, kunne nyttige funktioner såsom muligheden for realtids-beskeder, implementeres.

Programmet er designet til at opfylde et konkret behov, men hvor muligt, er en mere generel løsning implementeret. Et eksempel på en sådan løsning er programmets mulighed for opkobling til sensorer. Løsningen er designet til at være modulær, så forskellige sensorer kan benyttes i programmet.
