\chapter{Konklusion}
\label{cha:konklusion}
\frnote{forskellige aspekter er upræsist}

Problemanalysen havde til formål at analysere de forskellige aspekter af det initierende problem i \cref{initierende}. Der blev udarbejdet en undersøgelse af Vestre Baadelaugs nuværende it-system, baseret på to interviews foretaget med henholdsvis Kassereren og havnefogeden for Vestre Baadelaug. Ud fra problemanalysen blev det konkluderet at der var visse risici forbundet med implementeringen af et nyt it-system til Vestre Baadelaug. Disse risici er beskrevet i \cref{Opsummering_på_Refleksioner} og bunder primært i at Vestre Baadelaug ikke selv mener at der behov for nyt it-system. Trods det vurderes der stadig en mulig vinding ved at udvikle en et it-system til brug for havnefogeden, medlemmer samt gæster.  

% I problemanalysen kunne vi påpeje udfordringer og risici i forbindelse med implementationen af et nyt it-system til Vestre Baadelaug. Den største risici var at repræsentanterne fra Vestre Baadelaug ikke udviste interesse for et nyt it-system. Dette modsiger vi ved IOT INDSÆT KILDE LOL :D, hvor vi mener at siden alt bliver digitaliseret, så er det en naturlig overgang Vestre Baadelaug skal igennem i fremtiden. Vi lavede en løsning som vi vurdere værende brugervenlig og funktionel, på baggrund af opfyldelsen af kravene til designet samt de fuldendte usecases.

Igennem projektets forløb, har programmets struktur ændret sig, efterhånden som gruppen fik indsigt i arkitektoniske designmønstre, som eksempelvis MVC. Ideerne fra disse mønstre har hjulpet med til en forbedring af programmets struktur, men også til at forstå vigtigheden af god softwarearkitektur.

% Vi har igennem projektet arbejdet meget på programmets struktur og forsøgt at finde en god løsning. Vi fik indblik i diverse arkitektoniske designmøsntre som eksempelvis Model View Controller, og de har inspireret mange ændringer i strukturen, som har forbedret programmet men også givet os forståelse for go softwarearkitektur..

En af målene med løsningen var at designe en model, der stemmer overens med en bådklubs administrative ansvarsområder indenfor både, bådpladser, medlemmer og gæster. Dette mål er blevet opfyldt tilfredsstillende, i betragtning af projektets begrænsede tidshorisont. En mere komplet løsning, ville inkludere en model for gæstebetaling og medlemskontingent.

% Målet med løsningen var at designe et modul der kan administrere en bådklubs administrative ansvarsområder, med fokus på både, bådpladser, medlemmer og gæster. Vi mener at programmet nu er en løsning til de basale behov af en bådklubs administration. (Use the setion below)

Et andet mål med programmet var, at implementere en platform til formidling af information omkring havnen mellem havnefogeden, medlemmerne og gæsterne. Målet er opnået, da det er muligt for alle parter at se informationer om hinanden, såvidt de har de nødvendige tilladelser. Dette opnås ved hjælp af programmets kortfunktion samt søgefunktionen. En videreudvikling i denne retning, kunne indebære implementation af funktioner som realtids-beskeder.
 % På grund af tidshorisonten var det ikke muligt at tilføje disse.


Programmet er designet til at opfylde et konkret behov, men hvor det har været muligt, er en mere generel løsning implementeret. Et eksempel på en sådan løsning er programmets mulighed for opkobling til sensorer. Løsningen er designet til at være modulær, så forskellige sensorer kan benyttes i programmet.
% Vi konkludere at løsningen opfylder de krav vi har stillet for at fungerende program, men trods dette er der massere af muligheder for videreudvikling af programmet. Projektet vil være interessant at fortsætte på et senere semester. 

\frnote{PAS PÅ: ikke resumé. Husk at konkludere: VI KONKLUDERE}

\frnote{afslut med overordenet konklusion}