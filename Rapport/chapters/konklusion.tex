\chapter{Konklusion}
\label{cha:konklusion}
I dette kapitel vil der blive opsummeret på de væsentlige punkter i rapporten. Ud fra dette vil der blive konkluderet på hvorvidt problemformuleringen er blevet løst eller ej.

I problemanalysen havde vi kontakt med Vestre Baadelaug, for at finde problemer i forbindelse med administrationen af en bådklub. Det blev konkluderet at Vestre Baadelaug ingen problemer havde med deres nuværende system. I stedet for, så vi på muligheder, der ville være med et fuldt integreret it-system. Vi kom frem til en række udfordringer og risici i forbindelse med implementationen af et nyt it-system til Vestre Baadelaug. Den største risiko var, at repræsentanterne fra Vestre Baadelaug ikke udviste interesse for et nyt it-system. Dette modsiger vi under refleksionen i \cref{sec:refleksioner}, og delvist i IoT \cref{sec:internet_of_things}, hvor vi mener at eftersom alt bliver digitaliseret, så er det en naturlig overgang Vestre Baadelaug skal igennem i fremtiden. Ud fra disse overvejelser kom vi frem til denne problemformulering:

\begin{displayquote}
\textit{Hvordan kan man modellere en bådklubs administrative ansvarsområder indenfor både, bådpladser, medlemmer og gæster, samt implementere et software program der kan benytte modellerne til formidling af informationer mellem havnefoged, medlemmer og gæster, med det formål at forbedre overblikket over havnen?}
\end{displayquote}

Ud fra denne problem formulering blev der lavet en løsning, som vi vurderer til at være funktionel, på baggrund af opfyldelsen af kravene til designet samt de fuldendte use cases.

Vi har igennem projektet arbejdet meget på programmets struktur, og forsøgt at finde en god løsning. Vi fik indblik i diverse arkitektoniske designmønstre, eksempelvis Model View Controller, og de har inspireret mange ændringer i strukturen, som har forbedret programmet men også givet os forståelse for god softwarearkitektur.

Et af målene med løsningen var at designe en model, der stemmer overens med en bådklubs administrative ansvarsområder indenfor både, bådpladser, medlemmer og gæster. Den implementerede model dækker de fundamentale aspekter af administrationen af de ovenstående ansvarsområder, men inkorporerer ikke mere avancerede aspekter, såsom gæstebetalinger og medlemskontingenter.

Et andet mål med programmet var, at implementere en platform til formidling af information omkring havnen i mellem havnefogeden, medlemmerne og gæsterne. Målet er opnået, da det er muligt for alle parter at se informationer om hinanden, såvidt de har de nødvendige tilladelser. Dette opnås ved hjælp af programmets kortfunktion samt søgefunktionen. En videreudvikling i denne retning, kunne indebære implementation af funktioner som realtids-beskeder.

Vi konkludere at løsningen opfylder de krav vi har stillet for at fungerende program, men trods dette er der massere af muligheder for videreudvikling af programmet. Projektet vil være interessant at fortsætte med, i et efterfølgende forløb. 
