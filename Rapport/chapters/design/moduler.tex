\section{Moduler}
\label{sec:moduler}

Moduler defineres, i denne rapport, som logisk grupperede funktionaliteter og modeller i løsningen. Hvert modul skal være uafhængigt af andre moduler, og udelukkende håndtere ét enkelt ansvar. Et login modul skal derfor ikke have ansvaret, for at oprette en ny bruger. I stedet vil et dedikeret modul, til at oprette nye brugere, være favorabelt. Hvis hvert modul kun har ét ansvar, kan moduler nemmere genbruges i andre sammenhænge, eller udskiftes efter behov.

\begin{figure}
  \centering
  \includegraphics[width=\textwidth]{moduler-diagram.pdf}
  \caption{Overordnet oversigt over moduler i løsningen og deres relationer.}
  \label{fig:mod}
\end{figure}

På \cref{fig:mod} ses en oversigt over implementerede moduler og deres relationer. Modellerne, som benyttes i modulerne, er beskrevet i \cref{sec:klasse_design}. Hvert enkelt modul importerer de klasser der skal bruges. Som en del af et modul, kan der indgå \enquote{views}, der håndterer brugergrænsefladen. \enquote{Views} er beskrevet i \cref{sec:wpf}. Udover \enquote{views}, består et modul også af \enquote{controllers}, der styrer modulets funktionalitet overfor modellerne, samt eventuelle støtteklasser til disse.

I billedforklaringen, på \cref{fig:mod}, forstås \enquote{Fokus}, som brugergrænsefladens fokus. Det vil sige det brugeren har visuelt fokus på. Hvis brugeren eksempelvis klikker på \enquote{Ny gæst} i Loginmodulet, vil gæsteoprettelsesmodulet modtage det visuelle fokus, og brugeren kan nu interagere med dette. I billedforklaringen nævnes også \enquote{Data}. Med dette forstås det, at arbitrær data sendes i pilens retning. Eksempelvis kan data være et C\# metodekald, eller et C\# objekt der sendes rundt mellem moduler. \enquote{Data \& Fokus} repræsenterer scenarier, hvor der både sendes \enquote{Data} samt at \enquote{Fokus} ændres.

På \cref{fig:mod} kategoriseres nogle moduler som værende \enquote{Adgangskrævende funktioner}. Dette betyder, at disse moduler kun kan benyttes af validerede brugere. Hvis en bruger ikke er logget ind gennem loginmodulet, eller ikke har de nødvendige brugertilladelser, kan brugeren ikke benytte sig af disse funktioner.

Eksterne moduler, er moduler som ikke nødvendigvis er en direkte del af softwareløsningen. Dette er blandt andet hardware moduler.

I den fulde løsning skal flere moduler implementeres, hvis funktionaliteten af løsningen skal øges. De beskrevne moduler er tænkt som en grobund for efterfølgende moduler.

I de følgende afsnit er alle de implementerede moduler beskrevet ved modulets funktionalitet, brugen af andre moduler og hvordan modulet overordnet er implementeret.

\subsection{Persistenslag}
\label{sub:persistenslag}

Persistenslagsmodulet har til opgave at manipulere og læse data fra en vilkårlig datastruktur.

\subsubsection{Funktionalitet}
\label{ssub:persistenslag_funktionalitet}

En datastruktur kunne eksempelvis være en database eller en anden fil på harddisken. Modulet tilbyder basale CRUD (create, read, update og delete) operationer, hvorfra alle tænkelige persistenslagsoperationer kan opbygges af.

Et eksempel på brugen af persistenslagsmodulet, kunne være loginmodulet. Loginmodulet skal verificere om et indtastet brugernavn matcher et indtastet kodeord. Til dette kan loginmodulet tilgå databasen igennem persistenslagets read operation, og herfra arbejde videre på den fundne data.

\subsubsection{Implementation}
\label{ssub:persistenslag_implementation}
\begin{figure}
  \centering
  \includegraphics[width=\textwidth]{persistenslag-diagram.pdf}
  \caption{dette er tekst for tekst skyld }
  \label{fig:permod}
\end{figure}
\frnote{lidt tekst til billedet}

Persistenslagsmodulet er opbygget således at selve datastrukturen der manipuleres og læses fra, kan skiftes ud. Herved indkapsles tilgangen til datastrukturen på en sådan måde, at et program kan gå fra at benytte en xml fil som data lager, til at benytte en database uden at ødelægge anden eksisterende kode.

Som man kan se på \cref{fig:permod} består persistenslagsmodulet af en enklet klasse DALController. Denne klasse er kun bevidst om en konkret implementation af tilgangen til en datastruktur. Dette sker via interfacet IDAL. Når et andet modul ønsker at lave en læse operation på persistenslag modulet, videredelegeres denne operation til den underliggende implementation af tilgangen til en datastruktur.
\subsection{Login}
\label{sub:login}

Det første modul der interageres med, er loginmodulet, som har til formål at validere, og videresende brugeren til programmets hovedfunktioner.

\subsubsection{Funktionalitet}
\label{ssub:login_funktionalitet}

Loginmodulet har det primære ansvar for at angive adgangsniveauer, og sikre databeskyttelse. Fra loginmodulet kan man vælge at oprette sig som ny gæst, eller logge ind i systemet. Hvis brugeren vælger at oprette sig som ny gæst, så vil brugeren blive sendt til et separat modul.

\subsubsection{Implementation}
\label{ssub:login_implementation}

\begin{figure}
  \centering
  \includegraphics[width=\textwidth]{login-diagram.pdf}
  \caption{Diagram over Loginmodulet} \label{fig:login}
\end{figure}

Loginmodulet består af et standby-view, et MemberLogin-view, et eksternt chiplæsermodul samt en loginController. Fra standby view er der adgang til GuestCreatormodulet, som er beskrevet i \cref{sub:GuestCreator}. Brugeren kan også indlæses et chip-kort, eller skifte til MemberLogin-view. Fra MemberLogin-view, kan der indtastes medlemsnummer og kode. Se \cref{fig:login}.

LoginControlleren modtager data fra memberLogin-view, eller chiplæsermodulet. Denne data valideres, via persistenslaget, i forhold til databasens data. Hvis der ikke kan gennemføres en succesfuld validering, sender loginControlleren brugeren tilbage til det view, der blev sendt data fra, hvor brugeren promptes for nyt indput.

Hvis valideringen er succesfuld, sendes brugeren samt brugerens data, videre til funktionsstyringsmodulet.

\subsection{Styringsmodulet}
\label{sub:styringsmodul}

Styringsmodulet har til opgave at instantiere og kontrollere de moduler som udgør specifikke funktionaliteter for brugeren.

\subsubsection{Funktionalitet}
\label{ssub:hovedmodul_funktionalitet}

Når en bruger har været igennem en succesfuld validering fra loginmodulet, bliver brugeren sendt videre til styringsmodulet. Dette modul fungerer som skelettet for de omkringliggende moduler. Brugeren bevæger sig derfor hele tiden rundt indenfor dette moduls rammer, indtil brugeren ønsker at logge ud af programmet. Styringsmodulet bør altid instantiere et default modul med basal information til brugeren.

\subsubsection{Implementation}
\label{ssub:hovedmodul_implementation}

\begin{figure}
  \centering
  \includegraphics[width=\textwidth]{styringsmodul-diagram.pdf}
  \caption{Dette er bare noget filler tekst.}
\end{figure}

Styringsmodulet implementeres som en tabcontroller, der tilføjer andre moduler som faneblade. Default fanebladet er forsidemodulet. Dertil tilføjer styringsmodulet et kortmodul, et brugerinformationsmodul samt et søgningsmodul. Disse moduler bliver dog kun tilføjet, hvis brugeren som er logget ind, har de nødvendige tilladelser. Når brugeren ønsker at forlade programmet, logges der ud, og brugeren sendes tilbage til loginmodulet. Styringsmodulet indeholder også en metode, kaldet \enquote{SelectUser} der tager en bruger som parameter. Ved metodekald vises brugeradministrationsmodulet, med informationer omkring den bruger, der blev givet som parameter. 
\subsection{Brugeradministration}
\label{sub:Brugeradministration}

Brugeradministrationsmodulet har til opgave at kommunikere brugerdata mellem brugeren og databasen.

\subsubsection{Funktionalitet}
\label{ssub:Brugeradministration_funktionalitet}

Kommunikation mellem databasen og brugeren indebærer visning af information omkring en bruger, modtage inputs fra brugeren samt læse fra og skrive i databasen. At læse fra databasen indebærer, at den relevante information bliver hentet fra databasen, og vist på en acceptabel måde. Når der skrives til databasen menes der, at brugerens input bliver gemt i databasen. 

\subsubsection{Implementation}
\label{ssub:Brugeradministration_implementation}

\begin{figure}
  \centering
  \includegraphics[width=\textwidth]{brugerinformation-diagram.pdf}
  \caption{Dette er tekst for tekst skyld}
  \label{fig:brugermod}
\end{figure}

For at implementere dette består brugeradministrationsmodulet af tre \enquote{popups} og fire \enquote{controllere}. Popups'ne og controllerne har parvis deres tilsvarende ansvarsområde som de sammen står for at bearbejde. Popup'erne tager brugerens input om tilføjelse og ændring af en bruger, og videregiver det til dens respektive controller. Controlleren kommunikere til persistenslaget som dernæst udfører brugerens ønskede handling. Popup'erne bruges kun til tilføjelse eller ændring af data. Dette gøres fordi der i brugerens situation skabes klarhed omkring, hvordan brugeren skal bære sig ad med at tilføje eller ændre data.

For at implementere dette består brugeradministrationsmodulet af et båd-, medlem- og rejse-delmodul. Delmodulerne har hvert deres tilsvarende ansvarsområde og er designet til at kunne læse, tilføje, redigere eller slette data fra databasen ud fra brugerens input. Læsningen af data foregår ved at et delmodul viser dets data i de respektive tekstfelter. For at redigere eller tilføje data, bliver der benyttet popup vinduer. Ved sletning af data markeres det ønskede elementer og der klikkes på fjern-knappen. Ved at bruge popup vinduer, kan brugeren lettere kan skelne mellem at læse data og skrive data.

\subsection{GuestCreator}
\label{sub:GuestCreator}

GuestCreatormodulet har til opgave at oprette en ny gæst i databasen når brugeren er en gæst der ønsker adgang til systemet.

\subsubsection{Funktionalitet}
\label{ssub:GuestCreator_funktionalitet}
GuestCreatoren popper op når man i loginmodulet ønsker at oprette en ny gæst. Brugeren indtaster dernæst information omkring sig selv, båden samt udrejse dato. Dernæst bliver gæsten gemt i databasen.

\begin{figure}
  \centering
  \includegraphics[width=\textwidth]{newguest-diagram.pdf}
  \caption{GuestCreator diagram}
  \label{fig:guestcreator}
\end{figure}

\subsubsection{Implementation}
\label{ssub:GuestCreator_implementation}

GuestCreatormodulet består af et GuestCreator-element og en GuestController. Det indtastede data fra brugeren i GuestCreator-elementet sendes videre til GuestControlleren, som kommunikere med persistenslaget der derefter gemmer informationen om den nye gæst. 
\subsection{Kort}
\label{sub:kort}

\begin{figure}
  \centering
  \includegraphics[width=\textwidth]{map-diagram.pdf}
  \caption{Oversigt over kort modulet.}
  \label{fig:map_diagram}
\end{figure}

Kortmodulet har til opgave at lave et kort, der giver et visuelt overblik over alle bådpladser der eksisterer på en havn.

\subsubsection{Funktionalitet}
\label{ssub:kort_funktionalitet}

Kortet præsenterer bådpladsers status i form af en beskrivende farve. Eksempelvis er en optaget bådplads status repræsenteret med farven rød, imens en fri bådplads status repræsenteres med farven grøn. Kortet skal modellere virkeligheden, således at virkelighedens informationer er reflekteret i kortet. Hvis et andet modul ændrer en bådplads status, skal dette modul skifte bådplads beskrivende farve til den tilsvarende. 


\subsubsection{Implementation}
\label{ssub:kort_implementation}

Det implementerede kort er opbygget af to lag. Et statisk baggrundslag, som viser havnens broer. Dertil et overliggende lag bestående af bådpladsrepræsentationer, der ligger i forhold til deres placering på baggrundslaget. En bådpladsrepræsentation er en reference til en bådplads i databasen. Når en bådplads i databasen ændrer status, ændres dennes repræsentation sig i takt.

Når der klikkes på en bådplads på kortet, kan følgende scenarier ske, afhængigt af hvilken bruger der klikker.

\begin{tabu} to \textwidth {XX}
  \toprule
  \textbf{Tilstand} & \textbf{Reaktion} \\
   \midrule
   Brugeren ligger selv ved bådpladsen & Brugeren bliver videresendt til et modul der viser oplysninger om brugeren \\
   \midrule
  Der ligger en anden bruger på bådpladsen, og brugeren har de nødvendige adgangstilladelser & Brugeren videresendes, til et modul der viser oplysninger om brugeren \\
  \midrule
   Brugeren er en gæst uden en eksisterende bådplads. Bådpladsen er ledig & Brugeren bliver sendt til et reservationsmodul \\ 
   \midrule
   Brugeren er en gæst med en eksisterende bådplads. Bådpladsen er ledig & Brugeren spørges om hvorvidt brugeren vil flytte til den nye plads \\
   \bottomrule 
\end{tabu}


Ved situationer der ikke matcher de ovenstående kriterier, sker der ingenting.



\subsection{Søgemodul}
\label{sub:s_searchmodul}

Søgemodulet implementerer en sorteringsfunktion, som muliggør sortering af brugere.

\subsubsection{Funktionalitet}
\label{sub:funktionalitet}

Søgemodulets funktion er at sortere en liste af medlemmer og gæster, ud fra brugerdefinerede kriterier. Der kan sorteres efter alle datafelter som en bruger kan have. Derudover er det muligt at sortere efter datafelter som en brugers båd eller rejser indeholder.

\subsubsection{Implementation}
\label{sub:implementation}

Søgemodulet er inddelt i de følgende tre elementer:

\begin{itemize}
	\item \textbf{Søgebetingelseselementet} \\
		Dette element håndterer og validerer den information som en bruger måtte indputte. Ud fra disse oplysninger konstruerer betingelseselementet et unikt prædikat per datafelt. Dette prædikat tager en bruger som input, og returnerer en boolsk \enquote{sand}, hvis inputtet er en delstreng af brugernes pågældende felt.

	\item \textbf{Søgeelement} \\
    Søgeelementet har til formål at sortere en liste af bruger, på baggrund af prædikaterne fra betingelseselementet. Derudover står den også for at hente brugeroplysningerne fra databasen. Når søgeelementet bliver notificeret om, en ændring i en søgebetingelse, henter den det tilhørende prædikat fra betingelseselementet, og tilføjer prædikatet til en liste af søgekriterier. Ud fra denne liste sorteres listen af brugere, således at kun brugere, der returnerer \enquote{sand} for samtlige prædikater i listen af søgekriterier, bliver vist.

	\item \textbf{Bruger grænsefladen}
		Brugergrænsefladen har ansvaret for, at fremvise den sorterede liste, samt kalde sorteringsmetoden i søgeelementet når en bruger interagerer med brugergrænsefladen.
\end{itemize}

% subsection implementation (end)

% subsection funktionalitet (end)

% subsection s_gemodul (end)

\clearpage
\subsection{Forsiden}
\label{sub:Welcome}

Forsidemodulet har til formål at præsentere programmet for brugeren efter log ind.

\subsubsection{Implementation}
\label{ssub:Welcome_implementation}
Forsidemodulet åbnes fra styringsmoduler og er den første fane der vises. Se \cref{fig:forsidemod}. Denne fane er tiltænkt til at indeholde genveje med relevans for den bruger, der er logget ind. Derudover vil dette også være et oplagt sted, at have en opslagstavle, med  information fra havnefogeden eller bestyrelsen. Forsiden er med vilje konstrueret simpelt, for at opnå en neutral start. Alternativt kunne en mere funktionel fane vælges af den individuelle bruger, til at fremstå som deres egen forside.

\begin{figure}
  \centering
  \includegraphics[width=\textwidth]{forside-diagram.pdf}
  \caption{Diagram over forsidemodulet} \label{fig:forsidemod}
\end{figure}

