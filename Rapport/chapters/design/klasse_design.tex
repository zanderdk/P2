\section{Klasse Design}
\label{sec:klasse_design}

Som beskrevet i \cref{sec:klasse_teori}, er det vigtigt at klassehierarkiet modellerer de begreber fra virkeligheder, som programmet håndterer. Dette afsnit dokumenterer de grundlæggende modeller, som programmet er bygget op efter. 

For at give et overblik over klassehierarkiet, er der lavet et UML-klassediagram, som ses på \cref{fig:UML}. På figuren kan man se de vigtigste klasser, der er brugt i løsningen. Af figuren fremgår også klassernes relationer.

De følgende underafsnit beskriver programmets primære klasser og data der er inderholdt i klasserne.

\begin{figure}
  \centering
  \vspace*{-4.5cm}
  \makebox[\textwidth]{
    \includegraphics[width=1.2\paperwidth, angle = 270]{UML.png}
  }
 	\caption{UML-klassediagram over klasser der bruges til at modellere løsningen.} \label{fig:UML}  
 \end{figure}

\subsection{Modellering af Brugere}
\label{sub:brugere_af_programmet}

I klassehierarkiets top ligger den abstrakte klasse \enquote{User}. En \enquote{User} er en generel bruger af systemet. Denne klasse definerer alle fællestræk, de nedarvende klasser skal have. Dette inkluderer brugerrettigheder og basale informationer som telefonnummer, navn og adresse. Der findes tre forskellige interfaces, der indeholder forskellige informationer om brugere: \enquote{IFullPersonalInfo}, \enquote{ISailor} og \enquote{ILoginable}.

Subklasserne til \enquote{User} opdeler brugere som enten medlemmer af bådklubben, gæster eller havnefoged. Til disse tre brugertyper bruges klasserne \enquote{HarbourMaster}, \enquote{Guest} og \enquote{Member}. Denne opdeling er lavet, fordi der indgår forskellige felter på tværs af disse subklasser, og fordi de tre brugertyper er forskellige kategorier i programmet. 

For at differentiere subklasserne, implementeres forskellige kombinationer af interfaces. Eksempelvis må en havnefoged ikke have en båd, og derfor implementerer klassen \enquote{HarbourMaster} ikke \enquote{ISailer} interfacet. Til gengæld deler \enquote{HarbourMaster}, \enquote{IFullPersonalInfo} og \enquote{ILoginable} med \enquote{Member} klassen, da de begge har behov for at indeholde mere personinformation, samt at kunne logge ind med brugernavn/medlemsnummer.

\subsection{Modellering af Rejser}
\label{sub:rejser}

En medlems rejse modelleres med klassen \enquote{Travel}. Denne indeholder information om rejsens start- og slutdato. \enquote{Travels} bruges også, til at angive hvor lang tid en gæst ligger i havnen. \enquote{Travel} implementerer \enquote{IEquatable} interfacet, som gør det muligt at sammenligne med andre klasser af samme type. Klasser der implementerer ISailor indeholder en liste af \enquote{Travels}.
 
\subsection{Modellering af Både}
\label{sub:bade}

Der er lavet en klasse til både, der hedder \enquote{Boat}. Denne klasse bruges til at gemme data tilhørende en båd. Der gemmes bådens navn og størrelse, så der kan tjekkes hvorvidt en given plads er stor nok til at rumme båden.

\subsection{Modellering af Bådpladser}
\label{sub:pladser}

Der er to klasser til repræsentation af bådpladser. Klassen \enquote{WaterSpace} modellerer vandpladser og \enquote{LandSpace} modellerer landpladser. Begge disse klasser nedarver fra klassen \enquote{BoatSpace}, som indeholder generelle informationer om en bådplads. 

\subsection{Associationer}
\label{sub:associationer}

I klassehierarkiet benyttes to typer associationer, binær associationer og én-til-mange associationer. Den binære association på \cref{fig:UML} mellem både og bådpladser viser, at der er en reference indeholdt i båden til dens tilhørende bådplads, og vice versa. Denne association håndterer, at når en båd tildeles en plads, får samme plads også tildelt en reference til den pågældende båd. Associationen sikrer derved også, at man ikke kan tildele en plads en båd, hvis pladsen allerede inderholder en reference til en anden båd. 

En \enquote{ISailor} indeholder en liste af \enquote{Boat} objekter og en liste af \enquote{Travel} objekter. Dette udgør en én-til-mange association fra ISailor til boats. Denne association sikrer at hvis der tilføjes et objekt til en liste indeholdt i en ISailor, tildeles objektet automatisk en reference til samme ISailor. Hvis en båd får tildelt en ISailor som sin ejer, tilføjes båden også til listen af ejerens Både.

\subsection{Forbindelser til eksterne moduler}

For at kunne anvende eksterne moduler, såsom chiplæseren og bådsensoren, er der konstrueret interfaces, der skal implementeres af de eksterne moduler. Dette sikrer at de eksterne moduler overholder, de standarder der sættes af programmet. Samtidigt forsimpler det også implementering af disse eksterne moduler, da interfaces samler de nødvendige overvejelser.
