\chapter{Perspektivering}
\label{cha:perspektivering}

\frnote{skal læses igennem meget tale sprog}

Løsningen er udviklet baseret på vores erfaringer fra interviews med personer fra Vestrebaadehavn. Vores løsning kan have stor betydning for en organisation som Vestre Baadelaug. Den kan hjælpe organisationen med at blive mere effektiv. Det kan også hjælpe dem med at få orden i tinge. Med det overblik som en software løsning som vores kan skabe er der godt grund til at tro det vil blive bedre i fremtiden. Det er altså relevant at have overblik i en havn fordi der kan gøre oplevelsen bedre for folk der komme på besøg i havnen, som for eks. gæster.

Resultatet kan også bruges andre steder. For eksempel kunne man bruget et ligende system i andre foreninger. Der kunne være en tennisklub hvor de skal koordinere baner så alle kan få lov til at spille. De vil altså på samme måde som havnen have glæde af at få et overblik, som et kort ville give dem.

Hvis projektet skulle fortsætte, kunne man arbejde med at lave løsningen en mere generisk. På den måde vil den kunne virke på flere slagt foreninger. Der er allerede meget i løsningen som kan genbruges i andre programmer til andre foreninger. Der er allerede funktioner til at styre medlemmer og der brugertilladelser, det er noget man ville kunne bruge i fremtiden i andre foreninger. Løsningen er også delt op i faner. Så hvis en ander foregnige har brug for en ny funktion kan man tilføj faner med denne funktion.

Det blev valgt fra at skrive om udlejning af både, men det ville stadigvæk være intressant at se på udlejning af ting. Det kunne for eksempel bruges af klubber med faciliteter/udstyr som de skal udleje. Hvis man har ting der skal udlejes er det godt at have et overblik over tinge. Det kan være baner der skal udlejes eller noget andet der skal holdes styr på.

Der er ikke tænke over om projektet kan ville påvirke miljøet, men den ville nok ikke have en stor påvirkning af miljøet.

\frnote{skrive noget om at det ikke har den store samfundmæssige konsekvens}
