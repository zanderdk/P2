\chapter{Perspektivering}
\label{cha:perspektivering}

Løsningen, der er beskrevet i denne rapport, er udviklet baseret på vores erfaringer opnået ved interviews med Vestre Baadelaug. Løsningens primære funktionalitet er at skabe overblik. Dette kan Vestre Baadelaug, men også lignende bådklubber, drage nytte af. Ligesom i Vestre Baadelaug, vil de sandsynligvis opleve en administrativ problemstilling, i form af håndtering af deres medlemmer, bådpladser og gæster.

Hvis projektet skulle videreudvikles, kunne man arbejde på, at gøre løsningen yderligere generisk. På den måde vil den kunne anvendes af flere forskellige slags foreninger. Der er allerede meget i programmet, som kan genbruges i løsninger til andre foreninger. Eksempelvis er der allerede funktioner til at holde styr på medlemmer og deres tilladelser til funktionaliteter i programmet. Det er noget, som andre foreninger vil kunne udnytte i fremtiden. Løsningens brugergrænseflade er også delt op i separate faner. Hvis en anden forening har brug for en ny funktion, kan en ny fane nemt tilføjes, uden at det vil kræve en komplet revurdering af brugergrænsefladen.

Da Vestre Baadelaug ikke udlejer udstyr, blev det fravalgt, at skrive om udlejning af udstyr i forbindelse med denne løsning. Men det er stadigvæk interessant at se på udlejning, i forhold til brug i andre foreninger, da mange klubber ejer udstyr, som udlejes til medlemmers brug. Hvis man har aktiver der skal udlejes, er det godt at have et overblik over disse, for at kunne kontrollere lejeperioder og minimere tab. I den valgte kontekst er udlejning af bådpladser på årsbasis, så her er der ikke væsentlige problematikker ved kontrol af lejeperioder. Men hvis en squash-klub anvendte systemet til at registrere brug af banerne, kunne det være relevant med en tjek ind system, der målte hvor lang tid du anvendte banen, så du betalte for dit præcise brug.

\section{Brugertest i Vestre Baadelaug}

Som en afsluttende fase af programtestning, bør der foretages en brugertest. Løsningen blev udviklet på trods af en umiddelbar mangel på interesse for en ny løsning, fra den tiltænkte målgruppe. Hvis løsningen skal tages i brug af slutbrugeren, er det en nødvendighed at funktionaliteten er i top, for at overkomme status quo bias \cite{statusquo}.

Programmet er på udgivelsestidspunktet af denne rapport ikke i en tilstand, hvor det vil kunne anvendes fuldt ud af Vestre Baadelaug.