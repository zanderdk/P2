\chapter{Perspektivering}
\label{cha:perspektivering}

\frnote{skal læses igennem meget tale sprog}

Løsningen beskrevet i denne rapport er udviklet baseret på vores erfaringer opnået ved interviews med Vestre Baadelaug. Løsningens primære funktionalitet er at skabe overblik. Dette kan Vestre Baadelaug, men også lignende organisationer drage nytte af. Lignende organisationer kunne være idrætsforeninger der udlejer baner til medlemmer. Ligesom i Vestre Baadelaug, er her en administrativt  problemstilling med hensyn til håndtering af foreningens ressourcer. Mange af elementerne i en løsning til en sejlklub, går altså igen i andre organisationer.

%Løsningen er udviklet baseret på vores erfaringer fra interviews med Vestre Baadelaug. Vores løsning kan have stor betydning for en lignende organisation, da den kan bidrage til en øget effektivitet. Løsningen kan også hjælpe dem med at få orden i tingene. Med det overblik som en software løsning som vores kan skabe, er der god grund til at tro det vil, blive bedre i fremtiden. Det er altså relevant at have overblik i en havn, fordi det kan gøre oplevelsen bedre, for folk der komme på besøg i havnen, eksempelvis gæster.

%Resultatet kan også bruges andre steder. For eksempel kunne man bruge et lignende system i andre foreninger. Det kunne være en tennisklub, hvor de skal koordinere baner så alle kan få lov til at spille. De vil altså på samme måde som havnen, have glæde af at få det overblik, som et kort ville give dem.

Hvis projektet skulle videreudvikles, kunne man arbejde med at gøre løsningen endnu mere generisk. På den måde vil den kunne virke for flere slags foreninger. Der er allerede meget i programmet, som kan genbruges i løsninger til andre foreninger. Eksempelvis er der allerede funktioner til at holde styr på medlemmer og deres brugertilladelser. Det er noget som vil kunne bruges i fremtiden i andre foreninger. Løsningens brugergrænseflade er også delt op i seperate faner. Så hvis en anden forening har brug for en ny funktion, kan en ny fane tilføjes med denne funktion.

Det blev fravalgt, at skrive om udlejning af både i forbindelse med denne løsning, men det ville stadigvæk være interessant at se på udlejning. Det kunne for eksempel bruges af klubber med faciliteter eller udstyr som de skal udleje. Hvis man har ting der skal udlejes, er det godt at have et overblik over tingene. Det kan være baner der skal udlejes eller noget andet der skal holdes styr på.

Der er ikke tænkt over hvorvidt projektet vil påvirke miljøet, men den vil nok ikke have en stor påvirkning på miljøet.

\frnote{skrive noget om at det ikke har den store samfundsmæssige konsekvens}
