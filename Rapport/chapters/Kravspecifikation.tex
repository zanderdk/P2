
\label{sec:Kravspecifikation}

Ud fra problemformuleringen er der opstillet en kravspecifikation, som har til formål at udspecificere de krav til løsningen, som er underbygget i problemanalysen. Kravspecifikationen er opstillet som små afsnit med en kort beskrivelse indeni. Beskrivelsen vil desuden indeholde en reference til problemanalysen, hvor behovet for dette krav er dokumenteret.
 
\begin{enumerate}
  \item \label{itm:udlejning} \textbf{Systemet skal kunne håndtere udlejning og betaling af vandlejepladser}

Når gæster benytter havnens faciliteter skal der betales for dette. Som nævnt i \cref{sub:tek_betaling}, bruger Vestre Baadelaug en betalingsautomat, som gæsterne skal benytte til at betale for leje af vandlejepladser. Løsningen skal bygge videre på det nuværende system, og skal fungere som et komplet gæste-betalingssystem. Dette vil sige at gæsterne indberetter hvilken vandlejeplads de ligger på, samt hvor mange dage de vil ligge i havnen. Denne information skal gemmes i en database, så det let kan tilgås ved behov. Betalingen bliver klaret igennem det nuværende systems betalingsservice. Løsningen kan bruge oplysningerne til at se, om der vil opstå en vandlejeplads-konflikt, f.eks.\ hvis medlemmet der ejer pladsen kommer tidligere hjem end forventet. Ved at integrere sikres et funktionelt system med henblik på besparelse og tilvænningsperioden  for det nye system minimeres.

  \item \label{itm:brugergrupper} \textbf{Systemet skal have mulighed for forskellige brugertyper med forskellige adgangsniveauer.} \\\\
  Da systemet skal håndtere mange vigtige informationer omkring havnens ressourcer, er det vigtigt at begrænse alle og enhver til at overskrive f.eks.\ felter i en database omkring medlemmernes vandlejepladser.

  Der skal derfor være mulighed for at definere forskellige brugertyper med forskellige rettigheder, som beskrevet i \cref{sec:refleksioner}. Man kan forestille sig at administrator-brugeren har alle rettigheder, det vil sige at han kan læse, skrive og slette informationer. En anden brugergruppe kunne være medlemmer af klubben. De kan f.eks.\ defineres til kun at have læse rettigheder. Derudover kan de tildeles skrive-rettigheder på felter som eksempelvis afrejse- og hjemrejse tidspunkter.

\item \label{itm:reg_an_afkomst} \textbf{Systemet skal registrere afrejse og ankomst af gæster såvel som medlemmer.} 

  Det blev erfaret i \cref{sub:havnefoged} at havnefogeden bruger meget tid på at holde øje med nye gæster i havnen. Hvis systemet hele tiden var opdateret med besøgende gæster, kunne havnefogeden hurtigt få et overblik over status i havnen. Systemet skal også holde styr på når et medlem tager af sted. På den måde vides der om der er en fri plads som en anden sejler kan tage. Systemet skal kunne registrere når der kommer nye gæster. Dette gør at systemet hele tiden ved hvilke pladser der er frie, hvornår medlemmer kommer tilbage og hvornår gæster sejler igen. Dette er med til at lette arbejdet for havnefogeden.

\item \label{itm:gemme_pladser_baade} \textbf{Systemet skal gemme hvilke pladser der er ejet, hvilken båd der er på pladsen, og af hvilken ejer.}

  Klubbens nuværende systems kan det. Medlemmerne vil have gavn af, let at kunne se hvem der ejer en given plads. Dette vil også give kassereren et overblik over hvem der har både hvorhenne. Dette beskrives nærmere i \cref{sec:refleksioner}.

\item \label{itm:fejl} \textbf{Systemet skal have mulighed for at informere havnefogeden i tilfælde af fejl eller begivenheder.}

  Som beskrevet i \cref{sub:havnefoged}, er en stor del af havnefogedens arbejde på havnen, at hjælpe medlemmer og gæstende sejlere. Det ville derfor være optimalt hvis løsningen kunne informere havnefogeden hvis noget går galt med den automatiserede betjening. En sådan fejl kunne f.eks.\ være hvis brugeren af systemet har problemer med betaling ved billetmaskinen.

  Man kan forestille sig flere måder hvorpå havnefogeden kan blive informeret, f.eks. SMS, email, en rød lampe i kontoret, eller en hel anden ting. For at gøre det lettere for havnefogeden at hjælpe, kan fejlmeddelelsen indeholde informationer som eksempelvis standerens lokation og hvilke trin brugeren har foretaget, for at udløse fejlen. Man kan føre ideen med fjernhjælp helt ud, ved at tilføje en højtaler ved hver billetautomat. På den måde kan havnefogeden assistere brugeren igennem fejlen ved hjælp af en mikrofon.

\item \label{itm:vis_information} \textbf{Systemet skal kunne fremvise information omkring pladser, medlemmer, samt gæster.}

  I \cref{sub:gaster_havnefogeden} beskrives det hvordan havnefogeden administrerer hvilke gæster der har betalt for pladsleje. Derudover bruger han et andet system til at holde styr på, hvornår klubbens medlemmer vender tilbage til deres vandplads fra udflugter.

  Derfor skal løsningssystemet kunne præsentere relevante informationer som medlemmer af klubben eller havnefogeden kan benytte sig af i arbejdet. En liste af relevante informationer følger nedenfor.

  \begin{itemize}
  \item Hvilke pladser er ledige? Hvilke er optaget?
  \item Hvis pladsen er optaget, ligger der en gæst eller et medlem?
  \item Hvis der ligger en gæst på en given plads.
    \begin{itemize}
      \item Hvem er gæsten?
      \item Har gæsten betalt?
      \item Hvornår forlader gæsten havnen?
      \item Hvornår vender klubbens medlem tilbage, og vil have pladsen tilbage?
    \end{itemize}
  \item Hvem \enquote{ejer} en given plads?
  \end{itemize}

\item \label{itm:soege} \textbf{Systemet skal kunne søge igennem den lagrede data efter en række søgekriterier.}

  Løsningen skal være i stand til at søge i det lagrede data ud fra forskellige brugerdefinerede søgekriterier. Derudover skal det også være muligt for brugeren at hente, ændre eller slette i den lagrede data ud fra disse søgekriterier. Eksempelvis kunne havnefogeden ønske at vide hvem der ligger på en bestemt plads, derfor bør det være muligt at hente data om personen som ligger på pladsen, ud det pågældende pladsnummer.
  
\item \label{itm:gui} \textbf{Systemet skal være brugervenligt.}  
    programmet skal have en grafisk brugergrænseflade som vil blive betegnet som GUI. GUI'en har til formål at gøre software programmet mere brugervenligt, og skabe et mere overskueligt overblik over havne for havnefogeden samt gæster og medlemmer. Behovet for dette krav er underbygget af \cref{sec:aspekter} hvor det er beskrevet at havnefogeden på visse tidspunkter har overbliksproblemer.

\item \label{itm:events} \textbf{Systemet skal have et modul, der kan håndtere arrangementer i klubben og det skal give overblik over hvad der sker i havnen.}

I forbindelse med Vestre Baadelaugs ønske om et socialt liv på havnen, som beskrevet i \cref{subs:social}, skal der inkluderes funktioner som forbedrer de sociale muligheder. Det ville være fordelagtig hvis systemet kan holde styr på sociale arrangementer og lave en event-kalender. Medlemmer kan så opslå initiativer til spontane arrangementer, eksempelvis fælles aftensmad, hvis vejret viser sig at være godt. Systemet kunne bruge dens viden om medlemmer der ligger i havnen til selv lave opslag til arrangement forslag.

\end{enumerate}

\section{Afgrænsning af krav}

\begin{enumerate}
  \item \label{itm:afg_udlejning} Krav \ref{itm:udlejning} \\
  Under kravet omkring udlejning og betaling, ligger fokus på at få funktionalitet af udlejningen til at virke. Betalingssystemet og integrationen med Vestre Baadelaugs nuværende betalingssystem bliver nedprioriteret i forhold til funktionalitet af systemet, siden det vil være en ressourcekrævende opgave.

  \item \label{itm:events} \textbf{Krav 5} \\
  Kravet omkring fejlmelding til havnefogeden er ikke essentielt for programmets kernefunktionalitet, og er derfor afgrænses der fra dette, under implementation af løsningen.

  \item \label{itm:events} \textbf{Krav 6} \\
  Dette er et vidtspændende krav om understøttelsen af gæster i systemet. Dette har medført, at kravet kun vil blive opfyldt gradvist i løsningen. De underkrav som ikke implementeres, vurderes til ikke have høj nok relevans for kernefunktionaliteten af programmet. Desuden vil implementation af disse ikke demonstrerer yderligere akademisk kendskab.

  \item \label{itm:events} \textbf{Krav 9} \\
  Eventkalender anses som et socialt redskab, der kan bruges af Vestre Baadelaug, til at promovering events og LOBOP. Da dette ikke er nødvendigt for løsningens funktionalitet, og falder udenfor programmets primære formål, vil det være nederst på prioritetslisten.
\end{enumerate}
% section Kravspecifikation (end)
