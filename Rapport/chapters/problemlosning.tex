\chapter{Problemløsning}
\label{cha:problemlosning}
\frnote{mangler intro}

Der er blevet udarbejdet en problemformulering og en kravspecifikation. Ud fra dette skal der laves en løsning. Ifølge studieordningen skal der udarbejdes et større program af høj kvalitet. Det er også et krav at programmet skal skrives i C\#. I dette kapitel vil der være en beskrivelse af denne løsning. Der vil blive beskrevet hvordan løsningen blev lavet. Der vil blive gjort rede for det overordnede design af løsningen. C\# er oprindeligt lavet af Microsoft og bruges primært på Microsofts Windows styresystem. Derfor laves løsningen et Windows program i Microsoft Visual Studio. Til programmet er også lavet en simpel brugergrænseflade. Et Windows program passer godt til Vestre Baadelaug, da de allerede anvender Windows-baserede systemer til administrationen.

\kanote{ovenstående skal tages op til overvejelse}

\section{Anvendte Use Cases}

Herpå følger de use cases, der har indgået som basis, for de funktioner der eksisterer i løsningen. De er inddelt i fire kategorier, alt efter hvorvidt den pågældende use case tager udgangspunkt fra enten et medlem, en gæst, havnefogeden eller andre. Når der skrives at \enquote{systemet melder}, i sammenhæng med en plads, hentydes der til den visuelle repræsentation af pladsens tilgængelighed ændres i brugergrænsefladen, i samme stil som systemet beskrevet i \cref{sub:gaster_havnefogeden}.

\subsection{Medlemmer}  
  \begin{enumerate}

    \item{\bf{Medlem forlader sin plads i mere end 24 timer}}
      \begin{enumerate}
        \item Medlemmet melder til systemet et gyldigt tidspunkt for afrejse og hjemkomst.
        \item Systemet melder til medlemmet at rejsen er registreret.
        \item Systemet venter til efter tidspunktet for afrejse.
        \item Når pladsen tømmes, melder systemet at pladsen er fri.
      \end{enumerate}

    \paragraph{Alternativ: medlem melder ikke afrejse til systemet}
      \begin{enumerate}
        \item Systemet underretter havnefogeden om at en medlemsbåd har været væk fra havnen i mere end 24 timer.
      \end{enumerate}

    \paragraph{Alternativ: medlem afrejser ikke på meldte tidspunkt}
      \begin{enumerate}
        \item Systemet underetter både havnefogeden samt medlemmet om uoverensstemmelsen.
      \end{enumerate}


    \item{\bf{Medlem vender tilbage til sin plads efter minimum 24 timers afrejse}}
      \begin{enumerate}
        \item Systemet melder at pladsen er optaget.
      \end{enumerate}


    \item{\bf{Medlem vender ikke tilbage til sin plads på meldte tidspunkt}}
      \begin{enumerate}
        \item Systemet melder uoverensstemmelsen til havnefogeden.
      \end{enumerate}


    \item{\bf{Medlem ønsker at annullere en allerede anmeldt rejse}}
      \begin{enumerate}
        \item Medlem vælger den pågældende rejse fra listen over registrerede rejser.
        \item Medlem annullerer rejsen.
        \item Systemet melder tilbage at rejsen er annulleret.
      \end{enumerate}

	  
\subsection{Gæster}


    \item{\bf{Gæst er ankommet til havnen, og vil leje en plads}}
      \begin{enumerate}
        \item Gæst registrere sig selv i systemet.
        \item Gæst benytter chipkortet til at logge ind.
        \item Gæst finder oversigten over havnen og finder en ledig plads.
        \item Gæst vælger den ledige plads og booker den.
        \item Gæst betaler for pladsen.
        \item Systemet melder til havnefogeden at der er nye ankomne.
      \end{enumerate}

    \paragraph{Alternativ: Gæst finder ikke selv en plads}
      \begin{enumerate}
        \item Gæst registrere sig selv i systemet.
        \item Gæst indtaster sin båds specifikationer.
        \item Systemet returnerer en liste over pladser som vil passe hans behov.
        \item Systemet videresender gæsten til betaling.

      \end{enumerate}

    \item{\bf{Gæst vil se informationer om plads og betaling}}
      \begin{enumerate}
        \item Gæst indsætter chipkort i automat.
        \item Systemet viser informationer vedrørende gæsten.
      \end{enumerate}

    \paragraph{Alternativ: Chipkort er bortkommet}
      \begin{enumerate}
        \item Gæster melder til systemet at chipkort er bortkommet.
        \item Systemet henviser gæsten til havnefogeden.
      \end{enumerate}


    \item{\bf{Gæst forlader havnen på eller før anmeldte afrejse tidspunkt}}
      \begin{enumerate}
        \item Gæst aflever chipkort og får udleveret en kvittering.
        \item Gæst får returneret berettiget kapital.
      \end{enumerate}

    \paragraph{Alternativ: Gæst bliver liggende i havnen efter det anmeldte afrejse tidspunkt}
      \begin{enumerate}
        \item Systemet melder uoverensstemmelsen til havnefogeden.
        \item Chipkortet deaktiveres.
        \item Ved efterfølgende forsøg på brug af chipkort, henvendes der til havnefogeden.
      \end{enumerate}


    \item{\bf{Gæst vil forlænge ophold}}
      \begin{enumerate}
        \item Gæst indsætter chipkort i automat.
        \item Gæst melder ønskede ny afrejse dato til systemet.
        \item Systemet melder at den nye dato er registreret.
      \end{enumerate}

    \paragraph{Alternativ: Nye afrejse dato overlapper med medlemshjemkomst}
      \begin{enumerate}
        \item Systemet melder at datoerne overlapper, og foreslår ny plads.
      \end{enumerate}

\subsection{Havnefoged}
    \item{\bf{Havnefogeden registrerer et medlemmets hjemkost}}
      \begin{enumerate}
        \item Medlem meddeler tidlig hjemkomst til havnefogeden, for eksempel via telefon, mail eller lignende.
        \item Havnefogeden indtaster (ny) dato i systemet.
        \item Systemet returnere at datoen er accepteret.
      \end{enumerate}
  
    \paragraph{Alternativ: En gæst har lejet pladsen for en længere periode.}
      \begin{enumerate}
        \item Systemet foreslår en ny plads til gæsten.
        \item Havnefogeden snakker med gæsten.
      \end{enumerate}

    \item{\bf{Havnefogeden vil gerne se hvilke pladser der er ledige}}
      \begin{enumerate}
        \item Havnefogeden åbner overbliksfunktionen i programmet.
        \item Havnefogeden kan nu se på et kort over havnen, hvilke pladser der er ledige.
      \end{enumerate}

    \item{\bf{Havnefogeden vil gerne se hvilke nye gæster der er ankommet indenfor et tidsrum}}
      \begin{enumerate}
        \item Havnefogeden åbner gæster funktionen i programmet.
        \item Havnefogeden vælger det ønskede tidsrum.
        \item Programmet viser nye ankomne gæster fra det specificerede tidsrum.
      \end{enumerate}

    \item{\bf{Havnefogeden vil gerne se hvilke pladser der endnu ikke er betalt for}}
      \begin{enumerate}
        \item Havnefogeden åbner gæster funktionen i programmet.
        \item Havnefogeden åbner ubetalte pladser.
        \item Programmet præsenterer listen af pladser, der endnu ikke er betalt for. Tiden for hvor lang tid pladsen har været ubetalt vises også.
      \end{enumerate}

    \item{\bf{Havnefogeden vil gerne se hvornår et medlem vender tilbage til medlemmets plads}}
      \begin{enumerate}
        \item Havnefogeden åbner medlems funktionen i programmet.
        \item Havnefogeden skriver i søgefeltet medlemmets navn eller medlemsnummer.
        \item Programmet præsenterer en liste over matchende medlemmer.
        \item Havnefogeden vælger det søgte medlem.
        \item Programmet viser alle kendte detaljer om medlemmet, herunder hvornår medlemmet forventes tilbage.
      \end{enumerate}

    \item{\bf{Havnefogeden bliver advaret af en fejl ved en stander}}
      \begin{enumerate}
        \item Uventet fejl opstår ved en stander på havnen, eller personen ved standeren trykker på hjælp.
        \item Havnefogeden får besked om dette.
      \end{enumerate}

\subsection{Andre}

	
    \item{\bf{Automatisk arrangementsforslag}}
      \begin{enumerate}
        \item Systemet melder på baggrund af aktiviteten i havnen, at det ville være en god dag for et arrangement.
        \item Et medlem af klubben ser denne melding, og foreslår et arrangement.
      \end{enumerate}

    \item{\bf{Automatisk medlemstjek}}
      \begin{enumerate}
        \item Systemet registrerer uoverensstemmelse med angivet rejseplan.
        \item Systemet notificerer havnefogeden omkring overensstemmelsen.
      \end{enumerate}

    \item{\bf{Bestyrelsen eller havnefogeden vil se statistik over aktivitet i havnen}}
      \begin{enumerate}
        \item Statistik funktionen åbnes og viser de ønskede statistikker.
      \end{enumerate}
	  
	\item{\bf{Kassér eller havnefoged vil tilføje ny båd til et medlem}}
	  \begin{enumerate}
		\item Havnefogeden logger ind i systemet med administrator rettigheder.
		\item Medlemmet bliver fundet via søge funktionen.
		\item Havnefogeden tilføjer opretter en ny båd med pågældende information.
		\item Systemet melder tilbage at båden er registreret.
	   \end{enumerate}
\end{enumerate}


\chapter{Relevant Teori for Design og Implementation}

\kanote{intro tekst}

\section{Internet of Things} % (fold)
\label{sec:internet_of_things}

\subsection{Konceptet}
\label{sub:iot_koncept}


\enquote{Internet of Things} (IoT) er et paradigme der beskriver, at alle fysiske genstande skal være unikt identificerbare, sådan at en computer nemt kan administrere disse genstande. Derudover handler IoT om at forbinde disse genstande i et netværk \cite{kopetz2011real}. En mere beskrivende definition af IoT ville derfor være \enquote{a world-wide network of interconnected objects uniquely addressable, based on standard communication protocols} \cite{iot_survey_2010}. IoT er ofte beskrevet ud fra forskellige visioner for paradigmet. I \cite{iot_survey_2010}  beskrives tre visioner; \enquote{Things oriented}, \enquote{Internet oriented} og \enquote{Semantic oriented}. Disse visioner er opstået ud fra de forskellige interessenter i IoT. Se \cref{fig:iot_visions}.

Den første definition af IoT, \enquote{Things oriented}, stammer fra Auto-ID labs, som er et verdensomspændende netværk af forskere indenfor Radio Frequency IDentification (RFID) og anden sensor teknologi. Målet med IoT er i denne definition at forbedre synligheden og muligheden for at spore genstande. For at opnå dette er Electronic Product Code (EPC) standarden blevet udviklet. Denne standard har til formål at sprede brugen af RFID samt andre teknologier \cite{iot_survey_2010}.

En anden vision for IoT, \enquote{Internet oriented}, bygger videre på den første definition. Denne vision består i at alle objekter selv kan forbinde sig til hinanden og computere. Konsortiummet CASAGRAS har fokus på at skabe \enquote{a world where things can automatically communicate to computers and each other providing services to the benefit of the human kind} \cite{iot_survey_2010}.


Den sidste vision der beskrives her, \enquote{Semantic oriented}, beskriver hvordan den store mængde data der genereres i \enquote{Things oriented} delen af IoT, kan organiseres så mennesker nemmere kan se meningen i den store mængde data. Vision går altså ud på at sætte struktur på den megen data, for at forbedre mængden af viden man kan trække ud af dette data \cite{iot_semantics_2012}.


\begin{figure}
  \centering
  \includegraphics[width=\textwidth]{iot_visions.jpg}
  \caption{Internet of Things paradigmet ud fra tre visioner. Fra \cite{iot_survey_2010}.}
  \label{fig:iot_visions}
\end{figure}

\subsection{Anvendelse af Internet of Things på Vestre Baadehavn}
\label{sub:iot_vestre_baadehavn}

Som beskrevet i \cref{sub:iot_koncept}, er der mange muligheder for hvilke sensorer der kan benyttes til at identificere objekter. Dette afsnit vil se på RFID og ultralyds sensorer, samt kameraer med billedgenkendelses software, som er mulige sensorer der kan benyttes på en havn som Vestre Baadehavn.

RFID står for Radio Frequency IDentification, og som navnet antyder kan teknologien identificere objekter ved hjælp af radio frekvenser. Et RFID-tag, er en lille chip der indeholder en antenne, og en lille mængde data. Et objekt med et RFID-tag er nu \enquote{tagget}, og data om objektet kan læses ved hjælp af en RFID læser. Man kategoriserer typisk RFID-tags i passive og aktive. Et aktivt RFID-tag kræver strøm for at blive læst, og er derfor ofte tilsluttet et batteri. Et passivt RFID-tag kræver ingen strøm fra et batteri, men får derimod strømmen fra RFID læseren \cite{want2006rfid}.

Lad os antage at alle både er blevet tagget, sådan at de hver har en unik identifikation. Dette kunne gøres ved hjælp af passive RFID-tags, da disse er billige og kan sidde på både uden nogen direkte afhængighed af elektricitet. Derudover kan hver vandlejeplads registrere, ved hjælp af en sensor, hvorvidt vandlejepladsen er optaget. Hvis der ligger en båd, kan en anden sensor også se hvem der ejer båden.

Sensoren der registrerer hvorvidt der ligger en båd ved en vandlejeplads, kunne være en ultrasonisk sensor. Denne kan ved hjælp af lydbølger detektere tilstedeværelse af et objekt. Den kan dog ikke med stor præcision identificere objektet. En anden løsning er et kamera der ved hjælp af billed genkendelse kan identificere båden.

En computer forbundet til disse to typer sensorer, kan nu effektivt administrere en kæmpe havn med mange vandlejepladser. Når en ny båd lægger til på en vandlejeplads, vil computeren med det samme vide dette. Computeren kan derudover også kategorisere båden som en medlemsbåd eller som en gæstebåd. Da computeren nu har registreret at vandlejepladsen er optaget, skal den vende et skilt eller tænde for en diode, eller på anden måde ændres pladsens status til optaget.


\subsubsection{Opsummering}

I dette projekt arbejdes der udelukkende på en software løsning, og ikke på implementering af hardware. Derfor vil sensorerne som skal registrere bådene i havnen, blot blive simuleret, da deres tekniske opbygning ikke er direkte relevant for implementeringen af software løsningen. 

De simulerede sensorer kan opfatte tre tilstande. Enten kan den registrere at der ligger en ukendt båd, en kendt båd, eller ingen båd. Den ukendte båd vil være en gæst, og den kendte båd et medlem. Når en sensor opfanger et skift fra én tilstand til en anden, vil den sende et signal til softwareløsningen med info om den nye tilstand.

Det vurderes, på baggrund at den stigende interesse for \enquote{Internet of Things}, at denne slags sensorer er en del af en realistisk fremtidshorisont. 
\sinote{Mere? }

\section{Teori om Klasser i C\#}
\label{sec:klasse_teori}

C\# er et objektorienteret programmeringssprog. I objektorienterede programmeringssprog er det let at dele programmer op, og på den måde lave mere velorganiserede programmer, der nemmere vedligeholdes. En af grundsøjlerne bag det objektorienterede paradigme, er objekter. For at lave et objekt skal man definere en klasse, som er skabelonen for objektet \cite{michaelis2012essential}. 

Klasser er ofte en abstraktion eller en modellering af et koncept eller gestande fra den virkelige verden. Da det objektorienterede programmeringsparadigme ønskes anvendt, skal en model konstrueres, der kan anvendes i forbindelse med løsningen af problemformuleringen. I denne løsning ønskes der en modellering af de koncepter og genstande, der findes på en havn, og dette er derfor blevet undersøgt. For hvert relevant koncept og genstand, er der indgået overvejelser omkring den optimale måde at modellere dem i klasser.

I afsnit \cref{sec:klasse_design} beskrives hvilke valg der er taget i forbindelse med modelleringen af forskellige koncepter og genstande i løsningen.

\subsection{Teori om Klassehierarki og Klassediagrammer}
\label{sub:uml_teori}

For bedre at kunne opnå en optimal modellering, sættes klasser ind i et klassehierarki for at øge overskueligheden. Et klassehierarki er en organisering af et sæt af klasser og deres indbyrdes relationer. Eksempelvis er relationen mellem klassen \enquote{Dyr} og \enquote{Kat}, at \enquote{Kat} klassen nedarver fra \enquote{Dyr} klassen. \enquote{Kat} klassen er altså en subklasse fra \enquote{Dyr} klassen.

Et klassediagram beskriver opbygningen af et system ved at vise systemets klasser, deres attributter, metoder, og relationerne mellem objekter \cite{martin2006agile}. Et klassediagram er en af de vigtigste byggesten i objektorienteret modellering. Det kan bruges både til generel konceptuel modellering og til detaljeret modellering, der kan omsættes til kildekode. 

I et klassediagram er klasser repræsenteret med bokse. Boksen indeholder tre dele. Den første del indeholder navnet på klassen, Og hvilken type klasse det er. I C\# kunne det for eksempel være en normal klasse eller et interface. Den anden del indeholder klassens attributter. Den sidste del indeholder operationer som klassen kan foretage. Relationer mellem forskellige klasser, vises med pile.

For at give et overblik over klassehierarkiet i løsningen, er der konstrueret et klassediagram efter UML standarden i \cref{sec:klasse_design}.

\subsection{MVC}
\enquote{Model-View-Controller}, herefter kaldet MVC, er et designmønster, der bruges til at lave software med en brugergrænseflade \cite{mvcLecture}. Det opdeler en given software applikation i tre sammenhængende dele. Det centrale komponent, \enquote{Model}, som indeholder data og modellerer problemområdet. Et ydre komponent, \enquote{View}, der afvikler den visuelle repræsentation af information til brugeren. Den tredje del, \enquote{Controlleren}, som implementerer systemets funktionaliteter. Fordelen ved MVC, et program, med lav kobling hvor man let kan skifte dele ud. Man kan eksempelvis let skifte til et andet grafisk framework, ved blot at skifte det ydre \enquote{View} komponent ud.

\frnote{læst igennem (kasper), mangler mere}


\subsection{MVC}
\enquote{Model-View-Controller}, herefter kaldet MVC, er et designmønster, der bruges til at lave software med en brugergrænseflade \cite{mvcLecture}. Det opdeler en given software applikation i tre sammenhængende dele. Det centrale komponent, \enquote{Model}, som indeholder data og modellerer problemområdet. Et ydre komponent, \enquote{View}, der afvikler den visuelle repræsentation af information til brugeren. Den tredje del, \enquote{Controlleren}, som implementerer systemets funktionaliteter. Fordelen ved MVC, et program, med lav kobling hvor man let kan skifte dele ud. Man kan eksempelvis let skifte til et andet grafisk framework, ved blot at skifte det ydre \enquote{View} komponent ud.

\frnote{læst igennem (kasper), mangler mere}
\subsection{Databasehåndtering}
\label{sec:database}
I dette projekt benyttes Entity Framework, herefter \enquote{Entity}, som er Microsofts anbefalede teknologi til at tilgå data \cite{entity}. Entity er et \enquote{object-relation mapping framework} \cite{lerman2010programming}, der gør det muligt at konvertere data i mellem objekter og andre typesystemer som for eksempel en database. Dette giver programmøren en virtuel objekt database, hvor det fremstår, som at man kan gemme objekter i en database, der normalt kun kan gemme enkelte talværdier. Entity nedsætter behovet for at skrive kode, der har med data tilgang at gøre. Det er disse fordele der har gjort at Entity blev valgt. Entity er en del af .NET.


\subsection{Code First}
\label{sub:code_first}
Entity bruger en model for hvert objekt den skal gemme. Brugere af Entity skal definere modellen. Der er flere måder at gøre dette på, men i dette projekt bruges \enquote{Code First}. Med \enquote{Code First} skal programøren først skrive sin model som en C\# klasse. Derefter kan Entity bruges til at gemme disse klasser i en database. Entity kan ud fra felterne i den modellerede klasse selv oprette de nødvendige kolonner i databasen. Entity er altså en abstraktion for tilgangen til en underliggende database. Dermed kan brugeren af Entity nøjes med at bekymrer sig om programmeringen af modellen som skal gemmes.

\subsection{Annoteringer}
\label{sub:annoteringer}
Selvom Entity kan gøre mange ting automatisk, skal brugeren angive forskellige informationer. Når Entity skal lave database tabeller, skal det vide hvilket felt i C\# klassen der skal være \enquote{Primary Key}. En primary key (PK) er et unikt indeks til databasen, så den kan identificere de forskellige elementer i tabellen. Et felt i en klasse angives som PK ved at kalde feltnavnet for \enquote{<klassenavn>Id}. Denne konvention kan dog udelades, hvis bare feltet annoteres med \enquote{[Key]}.

\subsection{Migreringer}
\label{sub:migreringer}

Migreringer skal benyttes når programmøren gerne vil flytte data fra én database til en anden. En sådan situation kan opstå hvis en eksisterende database mangler en bestemt kolonne, f.eks. antal æbler. Denne database har været i brug, og indeholder derfor vigtig information, som ikke må gå tabt. Opgaven er nu at flytte den vigtige information fra den eksisterende database til en ny database der har den ønskede antal æbler kolonne.

Dette foregår i Entity ved hjælp af Package Manager Console i Visual Studio. Først skal programmøren fortælle Entity at den skal slå migrationer til. Dette gøres ved at skrive \enquote{Enable-Migrations} i Package Manager Console.

Nu kan programmøren lave en ændring i en model, i dette tilfælde tilføje et felt holder informationer om antallet af æbler. I Package Manager Console skrives nu \enquote{Add-Migration <migrations tekst>}, hvor migrations tekst er en beskrivende tekst til migrationen. Entity klarer nu arbejdet med at udregne hvilke ændringer der skal foretages når der skal opgraderes til denne nyligt oprettede migration, samt hvad der skal ske hvis der skal nedgraderes fra denne migration.

For at anvende den nye migration, skrives \enquote{Update-Database} i Package Manager Console.

\section{Database} 
\label{sec:database}

En database er en måde at organisere og strukturer støre mængder data på ved brug af tabeller. En database bruges til at hånter opgaver som at gemme, hente eller søge i lagrede data. En fil lagrede på hardisk kan ikke skrives til af flere processer på samme tid, en database derimod kan håndtere andmodningger fra flere aktører synkront. Det er et krav til vores system er at flere bruger skal kunne bruge systemet på samme tid uafhængigt af hinanden. Derfor skal flere instanser af programmet have mulighed for at tilgå den samme data både synkront og asynkront, dette er en væsentlig funktionalitet som brugen af en database tilføjer til programmet \cite{DatabaseMicosoftOffice}. 

\subsection{Databasetransaktion}
\label{sub:databasetransaktion}

Hvergang man udføre en eller flere operation på databasen, benyttes en databasetransaktion til at sikre at alle operationerne udføres. En databasetransaktion er en gruppering af operation, som databasen betraktes som en enhed, ingen operation kan udføre for uden at resten udføres. Såfremt en operation fejler fortrydes alle andre grupperede operationer \cite{databasetransaktion}.

\subsection{ACID}
\label{sub:acid}

ACID er et set af egenskaber som sikre at en database handler som beskrevet tiddeligere under en hver database transaktion. ACID er et akronym de følgende fire egenskaber: 

\begin{itemize}
	\item \textbf{Atomicity} \\
		I tilfælde af at en operation fejler skal alle operationer i transaktionen gendannes, så database har samme tilstand som før transaktionens start. 

	\item \textbf{Consistency} \\
		Alt data skal valideres før det skrives til databasen. Denne egenskab skal sikre at alt data fra transaktion er valideret før nogen anden del af data gemmes. Derved sikres at databasen altid havner en valideret tilstand.

	\item \textbf{Isolation} \\
		Transaktionerne skal udføres sekventielt og isoleret fra hindanden. En transaktion må ikke kunne se ændringer fra en anden ufærdig transaktion. Ved at udføre alle transaktion sekventielt sikres det at en transaktion ikke benytter sig af ugyldigt data fra en anden transaktion.

	\item \textbf{Durability} \\
		Denne egenskab skal sikre at når en transaktion er udført, er den gemt på en permanent lagerplads. Så selv hvis strømmen går skal alle effekterne af transaktionen være udført.
\end{itemize}

% subsection acid (end)
\section{Windows Presentation Foundation}

Windows Presentation Foundation (WPF) er et framework til at udvikle brugergrænseflader til Windows .NET platformen. Kernen i WPF er en vektor-baseret layout-motor. Udover kernen har WPF extra funktioner der inkluderer databinding, templates, animation, media services og andet. I dette projekt benyttes WPF's databinding og template funktioner. Disse vil blive beskrevet i dette afsnit.

I WPF består en brugergrænseflade af nogle visuelle elementer som beskrives ved hjælp af opmærkningssproget XAML. Sammen med XAML filen, hører en \enquote{Code-behind} fil. I denne C\# klasse kan eventuel logik tilhørende XAML filen programmeres. På den måde adskilleles udseende og adfærd \cite{microsoft_wpf}.

\subsection{Databinding}

Databinding er en process der forbinder en brugergrænseflade med programlogik. Fordelen ved at bruge databinding er, at brugergrænseflade-koden kan sepereres fra programlogikken. På den måde opnås en lav kobling mellem komponenterne.

Med databinding kan modeller, beskrevet som C\# klasser, direkte inkorporeres i brugergrænsefladen. Ved brug af events som INotifyPropertyChanged, der kaldes når en egenskab ændres, sørger WPF selv for at opdatere informationen der vises i brugergrænsefladen, så denne svarer til modellens status. Hvis der bruges en TwoWay-binding vil data i modellen blive opdateret efter hvad brugen indtasker i grænsefladen.
 
\frnote{find et andet ord istedet for view, eller forklar hvad der er}

For at databinde et view til en model i WPF, sættes DataContext egenskaben i sit view, til klassen der skal benyttes som model. I XAML filen tilhørende view elementet, kan modellens egenskaber og felter nu benyttes til at vise informationer.

\subsection{Templates}

For at begrænse duplikering af kode, kan ofte brugte WPF elementer flyttes ud i en ekstern fil. På den måde kan forskellige view elementer benytte samme funktionalitet eller visuelle element, uden at kopiere allerede skrevet kode. Et template konstrueres ved at nedarve fra WPF klassen \enquote{UserControl}. I den tilhørende XAML fil, skrives nu de ønskede visuelle elementer der skal kunne genbruges.


\chapter{Design og Implementation}

\kanote{intro tekst igen}

\section{Klasse Design}
\label{sec:klasse_design}
C\# er et objektorienteret programmeringssprog. I objektorienteret programmeringssprog er det let at dele programmer op, og på den måde lave mere velorganiserede programmer, der er nemmere vedligeholdes. En af grundsøjlerne bag det objektorienterede paradigme, er objekter. For at lave et objekt skal man definere en klasse, som er skabelonen for objektet. Klasser er ofte en abstraktion eller en modellering, af et koncept eller gestande fra den virkelige verden. Da det objektorienterede programmeringsparadigme ønskes anvendt, skal en model konstrueres, der kan anvendes i forbindelse med løsningen af problemformuleringen. I denne løsning ønskes der en modellering af de koncepter og genstande, der findes på en havn, og dette er derfor blevet undersøgt. For hvert relevant koncept og genstand, er der indgået overvejelser, omkring den optimale måde at modellere dem i klasser.

I følgende afsnit beskrives hvilke valg der er taget i forbindelse med modelleringen af forskellige koncepter og genstande i løsningen. 

\subsubsection{UML-klassediagram}

For at bedre kunne opnå en optimal modellering, sættes klasser ind i et klassehierarki for at øge overskueligheden. Et klassehierarki er en organisering af et sæt af klasser og deres indbyrdes relationer. Et klassediagram beskriver opbygningen af et system ved at vise systemets klasser, deres attributter, metoder, og relationerne mellem objekter \cite{martin2006agile}. For at give et overblik over klassehierarkiet er der lavet et klassediagram efter UML standarden. UML-klassediagram for løsningen kan ses herunder.

\frnote{Her skal der være et billede at vores flotte UML diagram}

\subsubsection{Modellering af Brugere}
\label{sub:brugere_af_programmet}

I klassehierarkiets top ligger den abstrakte klasse \enquote{User}. En \enquote{User} er en generel bruger af systemet. Denne klasse definerer alle fællestræk de nedvarende klasser skal have. Dette inkluderer brugerrettigheder og basale informationer som telefonnummer, navn og adresse. Der findes tre forskellige interfaces, der indeholder forskellige informationer om brugere: \enquote{IFullPersonalInfo}, \enquote{ISailor}, \enquote{ILoginable}.

Subklasserne til \enquote{User} opdeler brugere som enten medlemmer af klubben, gæster eller havnefoged. Til disse tre brugertyper bruges klasserne \enquote{HarbourMaster}, \enquote{Guest} og \enquote{Member}. Denne opdeling er lavet fordi der indgår forskellige felter på tværs af disse subklasser. 

For at differentiere subklasserne, implementeres forskellige kombinationer af interfaces. Eksempelvis må en havnefoged ikke have en båd, og derfor implementerer klassen \enquote{HarbourMaster} ikke \enquote{ISailer} interfacet. Til gengæld deler \enquote{HarbourMaster}, \enquote{IFullPersonalInfo} og \enquote{ILoginable} med \enquote{Member} klassen, da de begge har behov for at gemme mere personinformation, samt at kunne logge ind med brugernavn/medlemsnummer.

\subsection{Modellering af Rejser}
\label{sub:rejser}

En medlems rejse modelleres med klassen Travel. Den indeholder information om rejsen start- og slutdato. Rejser bruges også til at angive hvor lang tid en gæst ligger i havnen. Travel implementere IEquatable, som gør det muligt at sammenligne med andre klasser af samme type. Klasser der implementere ISailor indeholder en liste af Travels.

\subsubsection{Modellering af Både}
\label{sub:bade}

Der er en klasse til både der hedder Boat. Denne klasse bruges til at gemme data tilhørende en båd. Der gemmes dens navn og størrelse, så man kan tjekke hvorvidt en given plads er stor nok til at rumme båden.

Der er to klasser til repræsentation af bådpladser. Klassen WaterSpace modellere vandlegepladser og LandSpace modellere landlejeplads. Begge disse klasser nedarver fra klassen BoatSpace, som indeholder generelle informationer om bådpladser.


\section{Reflection i C\#}
\label{sec:refleksion}

\kanote{mangler noget omkring hvorfor refleksion}

Refleksion (fra engelsk: \textit{reflection}) er en funktion C\# stiller til rådighed, til undersøgelse af objekter i et program under kørselstid \cite{michaelis2012essential}. Et eksempel på brugen af refleksion kan være, at programmøren ønsker at finde et felt i en klasse ved navn.

I dette projekt bruges refleksion til at gøre metoder der tilgår databasen, meget generiske, hvilket sikrer stor kodegenbrug. Som beskrevet i \cref{sec:database}, består databasen af forskellige tabeller, herunder \enquote{Users}, \enquote{Boats}, \enquote{BoatSpaces} og \enquote{Travels}. Lad os forestille os en metode der skal slette et objekt fra en specifik tabel. Metodesignaturen for en sådan metode kunne se ud som vist i \cref{lst:refleksion_remove1}.


\begin{lstlisting}[label=lst:refleksion_remove1]
public void RemoveUser(User user)
\end{lstlisting}

Problemet med denne metode er at denne metode kun virker på \enquote{User} objekter. Dette betyder at der udover denne metode også skal skrives en \enquote{RemoveBoat}, \enquote{RemoveBoatSpace} og en \enquote{RemoveTravel} metode. For at undgå dette, kan der laves en \enquote{Remove} metode, med en parameter der afgør hvilken tabel der skal slettes noget i. Metodesignaturen kunne se ud som i \cref{lst:refleksion_remove2}.


\begin{lstlisting}[label=lst:refleksion_remove2]
public void Remove<T>(T item, string table)
\end{lstlisting}

Problemet er, at metoden \enquote{Remove} skal have \enquote{hard-codet} alle de forskellige navne på alle eksisterende tabeller ind i denne metode. Hvis navnet på en tabel så ændrer sig, vil metoden ikke længere virke.

For at løse ovenstående problemer kommer refleksion ind i billedet. Metoden VerifyTable som ses i \Cref{lst:refleksion_verifytable} er et kodeeksempel fra programmet. VerifyTable bruger refleksion til at gennemsøge en klasse, her LobobContext, for at finde en egenskab af typen DbSet<T>. DbSet<T> er en databasetabel, der gemmer oplysninger af en arbitrær type \enquote{T}. Hvis den ønskede tabel bliver fundet, returneres denne. Hvis ingen tabel bliver fundet, returneres null.

Først finder metoden alle egenskaber der er defineret på typen \enquote{LobobContext}. DBsetType er den egenskab der ønskes fundet. Metoden itererer nu alle egenskaber igennem, indtil en egenskab der matcher DBsetType er fundet. Hvis der er et match i foreach løkken, findes egenskab navnet, og lægges over i dbSetTarget. Det er nu muligt at dynamisk oprette en DbSet<T> ud fra typen \enquote{LobobContext}. Denne DbSet<T> returneres, og kan nu bruges i en \enquote{Remove} metode.

\begin{lstlisting}[label=lst:refleksion_verifytable, caption={Metode der tjekker om en tabel af typen T findes i databasen.}]
private DbSet<T> VerifyTable<T>(LobopContext context) where T : class
{
    Type lobobContextType = typeof(LobopContext);
    PropertyInfo[] properties = lobobContextType.GetProperties();
    Type DBsetType = typeof(DbSet<T>);
    string dbSetTarget = string.Empty;

    foreach (PropertyInfo item in properties)
    {
        if (DBsetType == item.PropertyType)
        {
            // table found
            dbSetTarget = item.ToString().Split(' ')[1];
            DbSet<T> dbSet = (DbSet<T>)lobobContextType.GetProperty(dbSetTarget).GetValue(context, null);
            return dbSet;
        }
    }
    // table of type not found
    return null;
}
\end{lstlisting}

\begin{lstlisting}[label=lst:refleksion_remove3, caption={Remove metode der kan tage en vilkårlig klasse ind, finde den rette tabel og derefter slette det parametiserede objekt}]
  public void Remove<TInput>(TInput item) where TInput : class
  {
      using (var db = new LobobContext())
      {
          DbSet<TInput> dbSet = VerifyTable<TInput>(db);
          if (dbSet == null) throw new KeyNotFoundException("table ikke fundet");
          dbSet.Remove(item);
          db.SaveChanges();
      }
  } 
\end{lstlisting}

Den færdige Remove metode, der nu virker på tværs af alle databasetabeller. Den er blevet generisk ved hjælp af refleksion. En forsimplet implementation der benytter VerifyTable af Remove metoden ses i \cref{lst:refleksion_remove3}. Hvis en Titanic båden skal fjernes fra \enquote{Boats} tabellen, kaldes koden set i \cref{lst:refleksion_remove4}. Der er nu ingen behov for at specificere i hvilken tabel Titanic skal fjernes fra, da Remove selv kan regne dette ud, ud fra den generiske typeparameter <Boat>.

\begin{lstlisting}[label=lst:refleksion_remove4]
Remove<Boat>(titanic);
\end{lstlisting}

\section{Håndtering af Brugertilladelser} % (fold)
\label{tilladelser}

De forskellige aktører for systemet, har forskellige niveauer af tilladelser i havnen. Eksempelvis skal havnefogeden have adgang til at se alle medlemmernes rejser, i modsætning til et menigt medlem, der kun har adgang til at se sine egne rejser. For at kunne modellere de administrative ansvarsområder korrekt, er det nødvendigt at kunne skelne imellem brugere og deres tilladelser.

Der findes tre grader af brugertilladelse i denne løsning; ingen, læse eller skrive. Graden \enquote{ingen}, er den laveste grad, og betyder at den respektive bruger ikke har nogen som helst form for adgang til den pågældende funktionalitet. Graden \enquote{læse} gør funktionaliteten synlig for brugeren. Graden \enquote{skrive} giver brugeren mulighed for at anvende funktionaliteten. Brugertilladelser er akkumulerende, således at skrivetilladelsen implicit også er en læsetilladelse. Ikke alle funktionaliteter kræver tilladelse.

Alle brugere har for enhver type tilladelse én af de tre grader af brugertilladelser. Dette giver mulighed for, at differentierer imellem forskellige instanser af brugermodellen, således at det ikke er nødvendigt at oprette en ny brugermodel, for enhver kombination af brugertilladelser.

Hvis hver individuel funktionalitet skulle have sin egen tilladelse, ville programmet blive yderst uoverskueligt. For at undgå dette, er funktionaliteterne blevet inddelt efter adgangssammenhæng, som så afhænger af den samme tilladelse. Dette kan gøres ved funktionaliteter, der alligevel altid vil have et parallelt adgangsniveau. Eksempelvis bør en bruger, der kan ændre sit eget navn i systemet, også have muligheden for at ændre sin adresse. Derfor er funktionaliteterne for at ændre både navn og adresse baseret på den samme tilladelse, der i programmet hedder \enquote{PersonalInfo}.

Brugere tilgår systemet via et chipkort, som indeholder information, der kan identificere brugerens unikke data i systemet. Det er også muligt for medlemmer at tilgå systemet via deres medlemsnummer og et kodeord. Når en bruger tilgår systemet, bliver programmet initialiseret ud fra de tilladelser, brugeren har. Synligheden bliver på denne måde begrænset, således at al information ikke er frit tilgængeligt, samt at brugergrænsefladen holdes så relevant som muligt.

\section{Moduler}
\label{sec:moduler}

\subsection{Login}
\label{sub:login}

Det første modul der interageres med, er loginmodulet, som har til formål at validere, og videresende brugeren til programmets hovedfunktioner.

\subsubsection{Funktionalitet}
\label{ssub:login_funktionalitet}

Loginmodulet har det primære ansvar for at angive adgangsniveauer, og sikre databeskyttelse. Fra loginmodulet kan man vælge at oprette sig som ny gæst, eller logge ind i systemet. Hvis brugeren vælger at oprette sig som ny gæst, så vil brugeren blive sendt til et separat modul.

\subsubsection{Implementation}
\label{ssub:login_implementation}

\begin{figure}
  \centering
  \includegraphics[width=\textwidth]{login-diagram.pdf}
  \caption{Diagram over Loginmodulet} \label{fig:login}
\end{figure}

Loginmodulet består af et standby-view, et MemberLogin-view, et eksternt chiplæsermodul samt en loginController. Fra standby view er der adgang til GuestCreatormodulet, som er beskrevet i \cref{sub:GuestCreator}. Brugeren kan også indlæses et chip-kort, eller skifte til MemberLogin-view. Fra MemberLogin-view, kan der indtastes medlemsnummer og kode. Se \cref{fig:login}.

LoginControlleren modtager data fra memberLogin-view, eller chiplæsermodulet. Denne data valideres, via persistenslaget, i forhold til databasens data. Hvis der ikke kan gennemføres en succesfuld validering, sender loginControlleren brugeren tilbage til det view, der blev sendt data fra, hvor brugeren promptes for nyt indput.

Hvis valideringen er succesfuld, sendes brugeren samt brugerens data, videre til funktionsstyringsmodulet.

\subsection{Persistenslag}
\label{sub:persistenslag}

Persistenslagsmodulet har til opgave at manipulere og læse data fra en vilkårlig datastruktur.

\subsubsection{Funktionalitet}
\label{ssub:persistenslag_funktionalitet}

En datastruktur kunne eksempelvis være en database eller en anden fil på harddisken. Modulet tilbyder basale CRUD (create, read, update og delete) operationer, hvorfra alle tænkelige persistenslagsoperationer kan opbygges af.

Et eksempel på brugen af persistenslagsmodulet, kunne være loginmodulet. Loginmodulet skal verificere om et indtastet brugernavn matcher et indtastet kodeord. Til dette kan loginmodulet tilgå databasen igennem persistenslagets read operation, og herfra arbejde videre på den fundne data.

\subsubsection{Implementation}
\label{ssub:persistenslag_implementation}
\begin{figure}
  \centering
  \includegraphics[width=\textwidth]{persistenslag-diagram.pdf}
  \caption{dette er tekst for tekst skyld }
  \label{fig:permod}
\end{figure}
\frnote{lidt tekst til billedet}

Persistenslagsmodulet er opbygget således at selve datastrukturen der manipuleres og læses fra, kan skiftes ud. Herved indkapsles tilgangen til datastrukturen på en sådan måde, at et program kan gå fra at benytte en xml fil som data lager, til at benytte en database uden at ødelægge anden eksisterende kode.

Som man kan se på \cref{fig:permod} består persistenslagsmodulet af en enklet klasse DALController. Denne klasse er kun bevidst om en konkret implementation af tilgangen til en datastruktur. Dette sker via interfacet IDAL. Når et andet modul ønsker at lave en læse operation på persistenslag modulet, videredelegeres denne operation til den underliggende implementation af tilgangen til en datastruktur.
\subsection{Kort}
\label{sub:kort}

\begin{figure}
  \centering
  \includegraphics[width=\textwidth]{map-diagram.pdf}
  \caption{Oversigt over kort modulet.}
  \label{fig:map_diagram}
\end{figure}

Kortmodulet har til opgave at lave et kort, der giver et visuelt overblik over alle bådpladser der eksisterer på en havn.

\subsubsection{Funktionalitet}
\label{ssub:kort_funktionalitet}

Kortet præsenterer bådpladsers status i form af en beskrivende farve. Eksempelvis er en optaget bådplads status repræsenteret med farven rød, imens en fri bådplads status repræsenteres med farven grøn. Kortet skal modellere virkeligheden, således at virkelighedens informationer er reflekteret i kortet. Hvis et andet modul ændrer en bådplads status, skal dette modul skifte bådplads beskrivende farve til den tilsvarende. 


\subsubsection{Implementation}
\label{ssub:kort_implementation}

Det implementerede kort er opbygget af to lag. Et statisk baggrundslag, som viser havnens broer. Dertil et overliggende lag bestående af bådpladsrepræsentationer, der ligger i forhold til deres placering på baggrundslaget. En bådpladsrepræsentation er en reference til en bådplads i databasen. Når en bådplads i databasen ændrer status, ændres dennes repræsentation sig i takt.

Når der klikkes på en bådplads på kortet, kan følgende scenarier ske, afhængigt af hvilken bruger der klikker.

\begin{tabu} to \textwidth {XX}
  \toprule
  \textbf{Tilstand} & \textbf{Reaktion} \\
   \midrule
   Brugeren ligger selv ved bådpladsen & Brugeren bliver videresendt til et modul der viser oplysninger om brugeren \\
   \midrule
  Der ligger en anden bruger på bådpladsen, og brugeren har de nødvendige adgangstilladelser & Brugeren videresendes, til et modul der viser oplysninger om brugeren \\
  \midrule
   Brugeren er en gæst uden en eksisterende bådplads. Bådpladsen er ledig & Brugeren bliver sendt til et reservationsmodul \\ 
   \midrule
   Brugeren er en gæst med en eksisterende bådplads. Bådpladsen er ledig & Brugeren spørges om hvorvidt brugeren vil flytte til den nye plads \\
   \bottomrule 
\end{tabu}


Ved situationer der ikke matcher de ovenstående kriterier, sker der ingenting.





\chapter{Test af Implementation}

\section{Systematisk Test af Program}
\label{sec:systematisk_test_af_program}

\subsection{Test Driven Development}
\label{sub:test_driven_development}

For at sikre at programmet er let at teste, er kernefunktionerne i programmet udviklet jævnfør softwareudviklingsprocessen \enquote{Test Driven Development}. I Test Driven Development (TDD) skrives unit tests af en metode, før selve implementeringen af funktionen. Denne udviklingsprocess sikrer, at nye funktionaliteter ikke ødelægger de forud eksisterende \cite{martin2006agile}. Programmøren tvinges også til at tænke på, hvordan metoden bruges ved kald. Det sikres hermed, at metoden er overskueligt konstrueret, således at den kan kaldes uden unødigt besvær. Et andet aspekt af TDD, er at testkoden tjener som dokumentation af den testede metodes funktionalitet. Tilgengæld riskeres et øget tidsforbrug, der dog muligvis kan fraskrives fejlfinding.

\subsection{Unit Testing i Microsoft Visual Studio}
\label{sub:unit_testing_i_microsoft_visual_studio}

Microsofts unit test framework for managed code er anvendt til unit testing. Frameworket kan teste \enquote{managed code}, som c\# falder ind under. Når tests er skrevet, kan Test Explorer i Visual Studio køre testene. Når testene er færdige, vil resultaterne blive præsenteret som failed tests, passed tests, not run tests og skipped tests \cite{msdn_unittest}.

Før der skrives unit tests, skal der oprettes et unit test projekt. Herefter kan der tilføjes klasser annoteret med \enquote{[TestClass]}. Inde i disse klasser, tilføjer man de metoder, annoteret med \enquote{[TestMethod]}, som skal køres. Et eksempel på en test metode, kan se i \cref{lst:test_notfound}.
Alle tests til systemet kan findes på CD'en der afleveres som bilag.

\begin{lstlisting}[label=lst:test_notfound, caption={Eksempel på testfunktion}]
  [TestMethod]
  public void ShouldThrowWhenNotFound()
  {
      var space = new WaterSpace(404404, 4.3, 5.4);
      try
      {
          var test = BoatDetector.BoatAt(space);
      }
      catch (KeyNotFoundException _)
      {
          return;
      }
      Assert.Fail("No exception was thrown.");
  }
\end{lstlisting}



\begin{figure}
  \centering
  \includegraphics{test_explorer.png}
  \caption{Resultat af tests i Test Explorer i Visual Studio 2013}
  \label{fig:test_explorer}
\end{figure}


\section{Systemtest}
En systemtest af et stykke software, er en test, der udføres på et komplet system. Systemtest er med til at evaluere, hvorvidt systemet overholder de specificerede krav. For at teste løsningen overfor de krav, der tidligere er stillet, vil der i dette afsnit blive demonstreret, hvordan udvalgte use cases kan udføres i programmet. De udvalgte use cases, er valgt på baggrund af deres evne til, at fremvise funktionaliteten i programmet. Demonstrationen er opstillet således, at der startes med en definition af input, en potentiel antagelse samt en hypotese for output. Dernæst en gennemgang af fremgangsmåden, og ud fra output bedømmes resultatet, som enten en \enquote{succes} eller en \enquote{fejl}. Dette er en test af hvorvidt programmet kan udføre de forudstående use cases, og derved leve op til de tilhørende krav.

\subsection{Test Databasen}
For at gøre programmet lettere at teste, er der lavet en test database. I denne database bliver der automatisk genereret fire specifikke medlemmer og nogle tilfældige medlemmer. Ved at have en database med nogle faste medlemmer der ikke ændre sig, sikres at det hele tiden er mulig at udføre systemtestene og få det samme resultat. 

Der er generet en havnefogeden med medlemsnummer: \enquote{4} og passwordet: \enquote{testpass}. Havnefogeden har skrivetilladelse, som vil give ham fuld adgang til at tilføje/ændre/slette i databasen.

Ved system start, generes der også et medlem ved navn \enquote{Johanne Hoffmann}, båden \enquote{Den Usynkelige II}, medlemsnummer \enquote{5}, passwordet \enquote{testpass} og telefonnummeret \enquote{22 95 41 87}. Derudover er der genereret 2 andre medlemmer med navnet Johanne; Johanne Friis og Johanne Snoep.

\subsection{Udførelse af System Test}

\textbf{Use case 1: Et medlem ønsker at forlade sin plads i mere end 24 timer.}

Johanne Hoffmann vil gerne rejse væk den 8/10-2014 og komme hjem den 18/10-2014.
\begin{enumerate}
	\item Johanne klikker på \enquote{Medlemslogin} og \enquote{Loginscreen} kommer dernæst frem og afventer input.
	\item Johanne indtaster sit medlemsnummer: \enquote{5} og password: \enquote{testpass}.
	\item Johanne klikker \enquote{Login} og forsidefanen kommer frem.
	\item Johanne navigere til brugeradministrationsfanen, og information omkring hende kommer frem.
	\item Johanne klikker på \enquote{Tilføj Ny Rejse} og \enquote{Tilføj Rejse} bliver fremvist.
	\item Johanne vælger henholdsvis den 8/10-2014 og 18/10-2014.
	\item Johanne klikker OK, og kan nu se sin udrejse og hjemkomst i rejse feltet.
\end{enumerate}

\textbf{Forventning:} Johanne kan logge ind med sine login oplysninger, og hun kan melde sin rejse til systemet.

\textbf{Resultat:} Succes


\textbf{Use case 15: Havnefogeden vil gerne se hvornår et medlem vender tilbage til medlemmets plads.}

Det antages at Havnefogeden kun har navnet: \enquote{Johanne} og telefonnummeret: \enquote{22 95 41 87}
\begin{enumerate}
	\item Havnefogeden klikker på \enquote{Medlemslogin} og \enquote{Loginscreen} frem vises.
	\item Havnefogeden indtaster sit medlemsnummer: \enquote{4} og password: \enquote{testpass}.
	\item Havnefogeden klikker \enquote{Login} og forsidefanen kommer frem.
	\item Havnefogeden navigerer til søgefanen.
	\item Havnefogeden indtaster navnet: \enquote{Johanne} hvor der kommer tre hits frem.
	\item Havnefogeden indtaster dernæst telefonnummeret: \enquote{22 95 41 87} og indsnævre resultaterne til det rigtige medlem.
	\item Havnefogeden dobbeltklikker på medlemmet og får vist informationen omkring medlemmet: \enquote{Johanne Hoffmann}.
	\item Havnefogeden aflæser rejsedatoen i rejsefeltet.
\end{enumerate} 

\textbf{Forventning:} havnefogeden kan logge ind. Havnefogeden kan finde den rigtige Johanne ud fra de to stykker af information han havde omkring Johanne Hoffmann.

\textbf{Resultat:} Succes


\textbf{Kasser eller havnefoged vil tilføje ny båd til et medlem}

Det antages at Havnefogeden vil tilføje en ny båd til medlemmet Johanne Hoffmann. Det antages også at havnefogeden har samme oplysninger, som i foregående scenarie. Båden er 12 meter lang og 6.5 meter bred, den har \enquote{registreringsnummeret: 4785963}, den hedder \enquote{Eurika} og den har ingen plads.
\begin{enumerate}
	\item Havnefogeden klikker på \enquote{Medlemslogin} og \enquote{Loginscreen} fremvises.
	\item Havnefogeden indtaster sit medlemsnummer: \enquote{4} og passwordet: \enquote{testpass}
	\item Havnefogeden klikker \enquote{Login} og forsidefanen kommer frem.
	\item Havnefogeden navigere til søgefanen.
	\item Havnefogeden indtaster navnet: \enquote{Johanne Hoffmann} hvor hun kommer frem.
	\item Havnefogeden dobbeltklikker på medlemmet og får vist information omkring medlemmet \enquote{Johanne Hoffmann}.
	\item Havnefogeden klikker på \enquote{Tilføj Ny Båd} og \enquote{NewBoatPopup} fremvises.
	\item Havnefogeden indtaster navnet: \enquote{Eurika}, registreringsnummeret: \enquote{4785963}, længden: \enquote{12} og bredden: \enquote{6,5}.
	\item Havnefogeden vælger at båden ingen plads har.
	\item Havnefogeden klikker Ok
	\item Havnefogeden kan nu se og inspicere båden \enquote{Eurika}.
\end{enumerate}

\textbf{Forventning:} havnefogeden kan logge ind og finde Johanne Hoffmann i søgefunktionen. Han kan også tilføje båden Eurika og dens tilsvarende information.

\textbf{Resultat:} Succes.

% Dette er en test for det nuværende programs funktionalitet i forhold til de usescases, der er blevet opstillet for programmet.
% Ved Success forstås der at denne funktion er (delvis) mulig.
% Ved Implied forstås der at det er underforstået og/eller ude af programmets omfang.
% Ved Failure forstås der at denne funktion ikke er mulig i programmets nuværende stadie.
% \subsection{Medlemmer}


% \begin{enumerate}
	% \item{\bf{Medlem forlader sin plads i mere end 24 timer.}}
		% HANDLING: Medlemmet vælger \"tilføj\" under rejser, og vælger en afrejse og hjemkomst dato - Rejsen er nu gemt.
	  % \begin{enumerate}
			% \item Success -  a) Medlem melder til systemet et gyldigt tidspunkt for afrejse og hjemkomst.
			% \item Success -  b) Systemet melder til medlemmet at rejsen er registreret. 
			% \item Implied -  c) Systemet venter til efter tidspunktet for afrejse.
			% \item Failure -  d) Når pladsen tømmes, melder systemet at pladsen er fri.
	   % \end{enumerate}
			
	% \item{\bf{Alternativt medlem melder ikke afrejse til systemet.}}
	  % \begin{enumerate}
			% \item Failure -  aa) Systemet underetter havnefogeden om at en medlemsbåd har været væk fra havnen i mere end 24 timer
	   % \end{enumerate}
	   
	% \item{\bf{Alternativt medlem afrejser ikke på det meldte tidspunkt.}}
	  % \begin{enumerate}
			% \item Failure -  ba) Systemet underetter både havnefogeden samt medlemmet om uoverensstemmelsen.
	   % \end{enumerate}
	   
	% \item{\bf{Medlem vender tilbage til sin plads efter minimums 24 timers afrejse og der ligger en gæst på pladsen.}}
	  % \begin{enumerate}
			% \item Failure -  a) Systemet melder at pladsen er optaget.
	   % \end{enumerate}
	   
	% \item{\bf{Medlem vender ikke tilbage til sin plads på det meldte tidspunkt.}}
	  % \begin{enumerate}
			% \item Failure -  a) Systemet melder uoverensstemmelsen til havnefogeden
	   % \end{enumerate}

	% \item{\bf{Medlem ønsker at annullere en allerede anmeldt rejse.}}
		% HANDLING: Medlemmet logger ind og markere rejsen, dernæst klikker medlemmet på fjern - Rejsen er nu slettet
	  % \begin{enumerate}
			% \item Success -  a) Medlem vælger den pågældende rejse fra listen over registrerede rejser.
			% \item Success -  b) Medlem annullerer rejsen.
			% \item Success -  c) Systemet melder tilbage at rejsen er annulleret.
	   % \end{enumerate}

% \subsection{Gæster}
	% \item{\bf{Gæst er ankommet til havnen, og vil leje en plads.}}
		% HANDLING: Gæsten registrere sigselv i system og logger ind med udleveret chipkort. Gæsten kan nu se kortet over potentielt ledige pladser.
	  % \begin{enumerate}
			% \item Success -  a) Gæst registrere sig selv i systemet.
			% \item Failure -  b) Gæst benytter chipkortet til at logge ind.
			% \item Success -  c) Gæst finder oversigten over havnen og finder en ledig plads
			% \item Failure -  d) Gæst vælger den ledige plads og booker den.
			% \item Failure -  e) Gæst betaler for pladsen.
			% \item Failure -  f) Systemet melder til en at der er nye ankomne.
	   % \end{enumerate}
	   
	% \item{\bf{Alternativt: Gæsten finder ikke selv en plads.}}
	  % \begin{enumerate}
			% \item Success - a) Gæst registrere sig selv i systemet.
			% \item Failure - b) Gæst indtaster hans/hendes båds specifikationer.
			% \item Failure - c) Systemet returnerer en liste over pladser som vil passe hans behov.
			% \item Failure - d) Systemet videresender gæsten til betaling.
	   % \end{enumerate}

	% \item{\bf{Gæst vil se informationer om plads og betaling.}}
		% HANDLING: Gæsten indsætter chipkort og logger ind. I Brugeradministrations tabben vil information omkring gæsten samt hans båd og rejser befinde sig.
	  % \begin{enumerate}
			% \item Failure -  a) Gæst indsætter chipkort i automat.
			% \item Success -  b) systemet viser informationer vedrørende gæsten.
	   % \end{enumerate}
     
	% \item{\bf{Alternativt: Chipkort er bortkommet.}}
	  % \begin{enumerate}
			% \item Failure -  a) Gæster melder til systemet at chipkort er bortkommet.
			% \item Failure -  b) Systemet henviser gæsten til Havnefogeden.
	   % \end{enumerate}
     
	% \item{\bf{Gæst forlader havnen på eller før anmeldte afrejse tidspunkt.}}
	  % \begin{enumerate}
			% \item Failure -  a) Gæst aflever chipkort og får udleveret en kvittering.
			% \item Failure -  b) Gæst får returneret berettiget kapital.
	   % \end{enumerate}
    
	% \item{\bf{Alternativt: Gæst bliver liggende i havnen efter det anmeldte afrejse tidspunkt.}}
	  % \begin{enumerate}
			% \item Failure -  a) Systemet melder uoverensstemmelsen til havnefogeden.
			% \item Failure -  b) Chipkortet deaktiveres.
			% \item Failure -  c) Ved efterfølgende forsøg på brug af chipkort, henvendes der til havnefogeden.
	   % \end{enumerate}
			
	% \item{\bf{Gæst vil forlænge ophold.}}
		% HANDLING: Gæsten logger ind, og benytter ændre rejse funktionen. (Den opdateres ikke visuelt)
	  % \begin{enumerate}
			% \item Failure -  a) Gæst indsætter chipkort i automat.
			% \item Success -  b) Gæst melder ønskede ny afrejse dato til systemet.
			% \item Failure -  c) Systemet melder at den nye dato er registreret.
	   % \end{enumerate}
     
	% \item{\bf{Alternativt: Nye afrejse dato overlapper med medlemshjemkomst.}}
	  % \begin{enumerate}
			% \item Failure -  a) Systemet melder at datoerne overlapper, og foreslår ny plads.
	   % \end{enumerate}

% \subsection{Havnefoged}
	% \item{\bf{havnefogeden registrerer et medlemmets hjemkost.}}
		% HANDLING: Havnefogeden logger ind med hans login. Dernæst finder havnefogeden medlemmet og ændre datoen for hjemkomst i hans rejse.
	  % \begin{enumerate}
			% \item Implied -  a) Medlem meddeler tidlig hjemkomst til havnefogeden, for eksempel via telefon, mail eller lignende.
			% \item Success -  b) havnefogeden indtaster (ny) dato i systemet.
			% \item Success -  c) Systemet returnere at datoen er accepteret.
	   % \end{enumerate}

	% \item{\bf{Alternativt: En gæst har lejet pladsen for en længere periode.}}
	 % \begin{enumerate}
			% \item Failure -  a) Systemet foreslår en ny plads til gæsten.
			% \item Implied -  b) havnefogeden snakker med gæsten.
	   % \end{enumerate}
      
	% \item{\bf{havnefogeden vil gerne se hvilke pladser der er ledige.}}
		% HANDLING: Havnefogeden logger ind og klikker på kort tabben hvori der er et kort over havnen.
	  % \begin{enumerate}
			% \item Success -  a) havnefogeden åbner overbliksfunktionen i programmet.
			% \item Success -  b) havnefogeden kan nu se på et kort over havnen, hvilke pladser der er ledige.
	   % \end{enumerate}
      
	% \item{\bf{havnefogeden vil gerne se hvilke nye gæster der er ankommet indenfor et tidsrum.}}
	  % \begin{enumerate}
			% \item Failure -  a) havnefogeden åbner gæster funktionen i programmet.
			% \item Failure -  b) havnefogeden vælger det ønskede tidsrum.
			% \item Failure -  c) Programmet viser nye ankomne gæster fra det specificerede tidsrum.
	   % \end{enumerate}
  
	% \item{\bf{havnefogeden vil gerne se hvilke pladser der endnu ikke er betalt for.}}
	  % \begin{enumerate}
			% \item Failure -  a) havnefogeden åbner gæster funktionen i programmet.
			% \item Failure -  b) havnefogeden åbner ubetalte pladser.
			% \item Failure -  c) Programmet præsenterer listen af pladser, der endnu ikke er betalt for. Tiden for hvor lang tid pladsen har været ubetalt vises også.
	   % \end{enumerate}
	   
	% \item{\bf{havnefogeden vil gerne se hvornår et medlem vender tilbage til medlemmets plads.}}
	  % \begin{enumerate}
			% \item Success -  a) havnefogeden åbner medlems funktionen i programmet.
			% \item Success -  b) havnefogeden skriver i søgefeltet medlemmets navn eller medlemsnummer.
			% \item Success -  c) Programmet præsenterer en liste over matchende medlemmer.
			% \item Success -  d) havnefogeden vælger det søgte medlem.
			% \item Success -  e) Programmet viser alle kendte detaljer om medlemmet, herunder hvornår medlemmet forventes tilbage.
	   % \end{enumerate}
	   
	% \item{\bf{havnefogeden bliver advaret af en fejl ved en stander.}}
	  % \begin{enumerate}
			% \item Failure -  a) Uventet fejl opstår ved en stander på havnen, eller personen ved standeren trykker på hjælp.
			% \item Failure -  b) havnefogeden får besked om dette.
	   % \end{enumerate}

% \subsection{Andre}
	% \item{\bf{Automatisk arrangementsforslag.}}
	  % \begin{enumerate}
			% \item Failure -  a) Systemet melder på baggrund af aktiviteten i havnen, at det ville være en god dag for et arrangement.
			% \item Failure -  b) Et medlem af klubben ser denne melding, og foreslår et arrangement.
	   % \end{enumerate}
      
	% \item{\bf{Automatisk medlemstjek.}}
	  % \begin{enumerate}
			% \item Failure -  a) Systemet registrerer uoverensstemmelse med angivet rejseplan.
			% \item Failure -  b) Systemet notificerer havnefogeden omkring overensstemmelsen.
	   % \end{enumerate}
    
	% \item{\bf{Bestyrelsen eller havnefogeden vil se statistik over aktivitet i havnen.}}
	  % \begin{enumerate}
			% \item Failure -  a) Statistik funktionen åbnes og viser de ønskede statistikker.
	   % \end{enumerate}
			
	% \item{\bf{Kasser eller havnefogede vil tilføje ny båd til et medlem}}
	  % \begin{enumerate}
		% \item Success - a) havnefogeden logger ind i systemet med administrator rettigheder.
		% \item Success - b) Medlemmet bliver fundet via søge funktionen.
		% \item Success - c) havnefogeden tilføjer opretter en ny båd med pågældende information.
		% \item Success - d) Systemet melder tilbage at båden er registreret.
	   % \end{enumerate}

% \end{enumerate}



%\section{Om Systemet}
\label{sec:om_systemet}

Systemet der omtales i \cref{cha:problemformulering} tænkes at være en konsolapplikation skrevet i C\#. Programmet interagerer med en database. Denne databse gemmer alle informationer der er tilgængelige gennem programmet.

\subsection{Eksempler På Kommandoer}
\label{sub:eksempler_p_kommandoer}

Da systemet giver mulighed for at håndtere gæster, skal det være muligt at tildele en gæst til en given vandlejeplads. Nedenstående \cref{lst:add_guest} tilføjer en gæst til databasen, fortæller at han holder ved vandlejeplads 42, hedder Jens Jensen og at han har betalt indtil 1. juni 2014.

\begin{lstlisting}[language=bash, label=lst:add_guest] 
  $ program add guest --boatarea=42 --name="Jens Jensen" --until="1/6/2014" 
\end{lstlisting}


