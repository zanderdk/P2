\chapter{Problemløsning}
\label{cha:problemlosning}
\frnote{mangler intro}

Der er blevet udarbejdet en problemformulering og en kravspecifikation. Ud fra dette skal der laves en løsning. Ifølge studieordningen skal der udarbejdes et større program af høj kvalitet. Det er også et krav at programmet skal skrives i C\#. I dette kapitel vil der være en beskrivelse af denne løsning. Der vil blive beskrevet hvordan løsningen blev lavet. Der vil blive gjort rede for det overordnede design af løsningen. C\# er oprindeligt lavet af Microsoft og bruges primært på Microsofts Windows styresystem. Derfor laves løsningen et Windows program i Microsoft Visual Studio. Til programmet er også lavet en simpel brugergrænseflade. Et Windows program passer godt til Vestre Baadelaug, da de allerede anvender Windows-baserede systemer til administrationen.

\kanote{ovenstående skal tages op til overvejelse}

%!TEX root = ../../Master.tex
\section{Usecases}

\subsection{Usecases:}

\begin{itemize}
  \item Gæst vil finde en plads:
  \begin{itemize}
    \item 1) Ny gæst melder til systemet, hvor lang tid han vil ligge til.
    \item 2) Systemet returnerer en liste over pladser som vil passe hans behov.
    \item 3) Gæsten vælger en plads som passer til hans båd.
    \item 4) Systemet melder at pladsen er nu reserveret (rød lampe).
    \item 5) Gæsten betaler for pladsen.
    \item 6) Systemet melder til havnefogeden at der er nye ankomne.

    \item Alternativ: De tre første 2 skridt er ikke nødvendige.
  \end{itemize}

  \item Medlem forlader sin plads:
  \begin{itemize}
    \item 1) Medlemmer melder til systemet tidspunkt for afrejse og hjemkomst.
    \item 2) Systemet venter til efter afrejse tidspunktet.
    \item 3) Når pladsen tømmes, da melder systemet at pladsen er fri (grøn lampe).
  \end{itemize}

  \item Medlem vender tilbage til sin plads:
  \begin{itemize}
    \item 1) Systemet registrerer når medlemmet ligger til, og melder at plads er optaget (rød lampe).
  \end{itemize}

  \item Havnefogeden melder plads optaget:
  \begin{itemize}
    \item 1) Medlem ringer til havnefogeden og fortæller om tidlig hjemkomst.
    \item 2) Havnefogeden indtaster (ny) dato i systemet.
    \item 3) Systemet returnere enten a eller b
    \item a) Dato er accepteret
    \item 1b) En gæst har lejet pladsen for en længere periode.
    \item 2b) Systemet foreslår en ny plads til gæsten.
    \item 3b) Havnefogeden snakker med gæsten.
  \end{itemize}

  \item Automatisk arrangementsforslag:
  \begin{itemize}
    \item 1) Antallet af medlemmer i havnen overstiger et bestemt niveau.
    \item 2) Systemet melder at det ville være en god dag for et arrangement.
    \item 3) Et medlem af klubben ser denne melding, og foreslår et arrangement.
  \end{itemize}

  \item Automatisk medlemstjek
  \begin{itemize}
    \item 1) Systemet registrerer uoverensstemmelse med angivet rejseplan.
    \item 2) Systemet notificerer havnefogeden omkring overensstemmelsen.
  \end{itemize}
  

\end{itemize}




\kanote{intro tekst}




%\section{Om Systemet}
\label{sec:om_systemet}

Systemet der omtales i \cref{cha:problemformulering} tænkes at være en konsolapplikation skrevet i C\#. Programmet interagerer med en database. Denne databse gemmer alle informationer der er tilgængelige gennem programmet.

\subsection{Eksempler På Kommandoer}
\label{sub:eksempler_p_kommandoer}

Da systemet giver mulighed for at håndtere gæster, skal det være muligt at tildele en gæst til en given vandlejeplads. Nedenstående \cref{lst:add_guest} tilføjer en gæst til databasen, fortæller at han holder ved vandlejeplads 42, hedder Jens Jensen og at han har betalt indtil 1. juni 2014.

\begin{lstlisting}[language=bash, label=lst:add_guest] 
  $ program add guest --boatarea=42 --name="Jens Jensen" --until="1/6/2014" 
\end{lstlisting}


