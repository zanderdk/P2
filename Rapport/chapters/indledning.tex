%!TEX root = ../Master.tex
\chapter{Indledning}
 \frnote{Brian mener at denne indledning er lidt skudt forbi (Legepladser, lukket havneområde). Vi vender tilbage til dette afsnit senere senere.}
I denne rapport vil vi undersøge havne og bådforeninger. Havne er beskyttede vandområder, der gør det muligt for en båd at ligge fortøjret. I Danmark er der mange havne, og vi har haft stor glæde af fiskeri og transport industrien. Efter en stærk stigning af privatejede lystbåde i 1960'erne og 1970'erne, er der blevet bygget mange lystbådehavne \cite{gyldendal_redaktionen_havn_2013}. I disse havne kan privatpersoner have deres både liggende. De fleste lystbådehavn er gennem tiden blevet udvidet med flere faciliteter som toilet, restaurant, legeplads, vaskeri og butikker. Lystbådehavne er typisk ejet af staten. Staten udlejer hele havnen eller bestemte dele/pladser til foreninger. Foreninger lejer pladserne videre til deres medlemmer. Det er derfor tit et krav, at man skal være medlem af en forening for at få en fast plads i en lystbådehavn. Da foreningen har fuldt rådighed over pladserne i havnen kan den leje dem ud til gæster. Gæster er andre bådejere der komme forbi havnen på deres lystture. I nogle foreninger er der dedikerede pladser til gæstebåde. I andre foreninger bruges de regulære medlemmers pladser når medlemmerne er på tur.

Havnen styres typisk af en havnefoged. Havnefogeden står for opsyn med havneområdet og trafik i og omkring havnen. Det er også havnefogedens job at skaffe plads til alle både ved broerne \cite{undervisningsministeriet_havnefoged_2014}. En stor del arbejdet går med administration. I Danmark findes der ikke nogen uddannelse til havnefoged, men Foreningen af Lystbådehavne I Danmark tilbyder uddannelse i administration af en havn \cite{lystbadehavne_i_danmark}.

Foreninger udgør en stor rolle i mange danskeres liv. Disse foreninger er ofte drevet af frivillige sjæle, som tilbyder deres fritid til fællesskabet. Disse frivillige er motiveret af deres... \sinote{mangler fortsættelse}

\section{Initierende problem}

Der kan gå meget tid med administrationsarbejde, hvilket kan bruges på andre opgaver. Derfor vil vores problem analyse tage udgangspunkt i dette initierende problem:

\textit{Hvilke problemer opstår i forbindelse med administrationen af en bådklub og er der mulighed for at forbedre administrations arbejdet med en software løsning?}

For at undersøge dette, interviewes flere repræsentanter fra en bådklub. Laudon og Laudon modellen benyttes til at sætte et IT system i kontekst. Vi vil komme frem til en problemformulering og lave et program der løser det formulerede problem.
