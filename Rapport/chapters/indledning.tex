%!TEX root = ../Master.tex
\chapter{Indledning}
 \kanote{kilder mangler - måske skal der laves en bedre kobling fra foreninger til bådklubber}

To af Danmarks største traditioner er foreninger og søfart \cite{moller1997}. Begge har spillet en stor rolle i Danmarkshistorien. Med traditioner som disse er en hel del søfartsforeninger blevet grundlagt langs de danske kyster. 

Foreninger, søfart eller ej, er opbygget af medlemmer med en eller flere fælles personlige interesser. De fleste foreninger drives ved frivillig arbejdskraft, som folk giver i bytte for at kunne udleve deres interesse i fællesskab med andre. Nogle foreninger er små, og kan sagtens drives uden stor koordination. Men der findes også store foreninger, som råder over en masse medlemmer, dyre faciliteter eller måske et globalt netværk. I takt med at foreninger vokser, stiger behovet for administrationsarbejde. Nogle gange sker det i en sådan grad, at de frivillige, som påtager sig administrationsarbejde, ikke længere kan deltage i foreningen på niveau med et menigt medlem. Nogle gange skal de frivillige håndtere store mængder af foreningens likvide midler, hvilket ofte kan give anledning mistro og ønsker om mere kontrol med administrationen, fra foreningens medlemmer. Dette skaber igen mere behov for administration, og generalforsamlinger og budgetter og meget mere. Foreninger der administrerer faciliteter, har ofte behov for en grad af vedligeholdelse, til tider af en professionel grad. 

Alle disse faktorer af foreningsadministration er ofte langt fra den originale personlige interesse for foreningen. Af denne grund kan det være svært at finde frivillige til at administrere klubber.

De administrative opgaver indebærer ofte primært informations håndtering, eksempelvis medlemslister, tidsplaner og kontingenter. Systematisering af håndteringen af denne slags information, kan med fordel gøres ved hjælp af software. Hvis software laves og anvendes på en god måde, kan der bidrage til store optimeringer af arbejdsgange. Dette er et mål i de fleste foreninger, da det nedsætter krav til de frivillige, som så enten kan lave mindre arbejde, eller lave arbejde på andre områder. 

Efter en kraftig stigning af privatejede lystbåde i 1960'erne og 1970'erne, er der blevet bygget mange lystbådehavne \cite{gyldendal_redaktionen_havn_2013}. De fleste lystbådehavne har en forening af bådejere tilsluttet, og flere steder er det nødvendigt at være medlem, for at kunne have hjemstavn i havnen \cite{int_vb_sl}.

I dette projekt vil muligheden for brugen af en softwareløsning undersøges i forhold til netop sådan bådklub i Aalborg. Bådklubben i fokus er Vestre Baadelaug, som blev stiftet i 1915 \cite{vb_historie}. Projektet vil munde i en løsning, som er tilpasset efter Vestre Baadelaug, men der stræbes imod en general anskuelse, som vil kunne adapteres af andre bådklubber.


%I denne rapport vil vi undersøge havne og bådforeninger. I Danmark er der mange havne, og vi har haft stor glæde af fiskeri og transportindustrien. Efter en stærk stigning af privatejede lystbåde i 1960'erne og 1970'erne, er der blevet bygget mange lystbådehavne \cite{gyldendal_redaktionen_havn_2013}. I disse havne kan privatpersoner have deres både liggende. De fleste lystbådehavn er gennem tiden blevet udvidet med flere faciliteter som toilet, restaurant, legeplads, vaskeri og butikker. Lystbådehavne er typisk ejet af staten. Staten udlejer hele havnen eller bestemte dele/pladser til foreninger. Foreninger lejer pladserne videre til deres medlemmer. Det er derfor tit et krav, at man skal være medlem af en forening for at få en fast plads i en lystbådehavn. Da foreningen har fuldt rådighed over pladserne i havnen kan den leje dem ud til gæster. Gæster er andre bådejere der komme forbi havnen på deres lystture. I nogle foreninger er der dedikerede pladser til gæstebåde. I andre foreninger bruges de pladser, der bliver ledige når et medlem tager på en længere sejltur.

%Havnen styres typisk af en havnefoged. Havnefogeden står for opsyn med havneområdet og trafik i og omkring havnen. Det er også havnefogedens job at skaffe plads til alle både ved broerne \cite{undervisningsministeriet_havnefoged_2014}. En stor del arbejdet går med administration. I Danmark findes der ikke nogen uddannelse til havnefoged, men Foreningen af Lystbådehavne I Danmark tilbyder uddannelse i administration af en havn \cite{lystbadehavne_i_danmark}.

%Foreninger udgør en stor rolle i mange danskeres liv. Disse foreninger er ofte drevet af frivillige sjæle, som tilbyder deres fritid til fællesskabet. Disse frivillige er motiveret af deres... \sinote{mangler fortsættelse}

\section{Initierende problem}
\label{initierende}
Projektet startede med et udgangspunkt i udlejning af både. Men det blev hurtigt klart at, der ikke var så mange klubber der lejer både ud, som forventet. I stedet skiftede fokus til administration af bådklubber. Projektets problemanalyse tager udgangspunkt i dette initierende problem:

\textit{Hvilke problemer opstår i forbindelse med administrationen af en bådklub og er der mulighed for at forbedre administrations arbejdet med en software løsning?}

For at undersøge dette vil der blive lavet interviews med flere repræsentanter fra bådklubber i Vestre Bådehavn. Laudon \& Laudon modellen benyttes til at sætte et IT system i kontekst. Ud fra problemanalysen vil der blive udarbejdet en problemformulering og lave et program der løser det formulerede problem.
