%!TEX root = ../Master.tex
\pagenumbering{arabic}
\pagestyle{BasicStyle}
\setcounter{page}{1}
\chapter{Indledning}
To af Danmarks største traditioner er foreninger og søfart \cite{forening2010,moller1997}. Begge har spillet en stor rolle i Danmarkshistorien. Med traditioner som disse, er en hel del søfartsforeninger, blevet grundlagt langs de danske kyster. 

Foreninger, søfarts eller ej, er opbygget af medlemmer med én eller flere fælles personlige interesser. De fleste foreninger drives ved frivillig arbejdskraft, som folk giver i bytte, for at kunne dele deres interesse i fællesskab med andre. 

Nogle foreninger er små, og kan sagtens drives uden større koordination. Men der findes også store foreninger, som råder over en masse medlemmer, dyre faciliteter eller måske et globalt netværk. I takt med at foreninger vokser, stiger behovet for administrationsarbejde. Nogle gange sker det i en sådan grad, at de frivillige, som påtager sig administrationsarbejde, ikke længere kan deltage i foreningen på niveau med et menigt medlem. Nogle gange skal de frivillige håndtere store mængder af foreningens likvide midler, hvilket ofte kan give anledning til mistro, og ønsker om mere kontrol med administrationen fra foreningens medlemmer. Dette skaber endnu mere behov for administration, generalforsamlinger, budgetter og meget mere.

Alle disse faktorer af foreningsadministration, er ofte langt fra den originale personlige interesse for foreningen. Af denne grund kan det være svært, at finde frivillige, til at administrere klubber.

De administrative opgaver indebærer ofte primært informationshåndtering, eksempelvis medlemslister, tidsplaner og kontingentinformation. Systematisering, af håndteringen af denne slags information, kan med fordel gøres ved hjælp af software. Hvis software laves og anvendes på en god måde, kan det bidrage til store optimeringer af arbejdsgange. Dette er et mål i de fleste foreninger, da det nedsætter krav til de frivillige, som så enten kan lave mindre arbejde, eller lave arbejde på andre, mere primære, områder.

Efter en kraftig stigning af privatejede lystbåde i 1960'erne og 1970'erne, er der blevet bygget mange lystbådehavne \cite{gyldendal_redaktionen_havn_2013}. De fleste lystbådehavne har en forening af bådejere tilsluttet, og flere steder er det nødvendigt, at være medlem, for at kunne have hjemstavn i havnen \cite{int_vb_sl}.

I dette projekt vil muligheden for brugen af en softwareløsning undersøges, i forhold til netop sådan en bådklub i Aalborg. Bådklubben i fokus er Vestre Baadelaug, som blev stiftet i 1915 \cite{vb_historie}. Projektet vil udmunde i en løsning, som er tilpasset til Vestre Baadelaug, men der stræbes imod en general anskuelse, som vil kunne adapteres af andre foreninger.

\section{Initierende Problem}
\label{initierende}

Projektet startede med udgangspunkt i udlejning af både. Men det blev hurtigt klart at, der ikke er så mange klubber, der lejer både ud, som forventet. Derfor blev fokus skiftet til administrationen af bådklubber. Projektets problemanalyse tager udgangspunkt i dette initierende problem:

\textit{Hvilke problemer opstår i forbindelse med administrationen af en bådklub, og er der mulighed for, at forbedre administrationsarbejdet med en software løsning?}

For at undersøge dette, blev der foretaget interviews med repræsentanter fra bådklubber i Vestre Bådehavn. Laudon \& Laudon modellen blev benyttet, til at analysere de kontekstuelle forhold omkring en mulig softwareløsning. Ud fra problemanalysen vil der blive udarbejdet en problemformulering, og implementeret et program, der løser det formulerede problem.