%!TEX root = ../Master.tex
\chapter{Indledning}
 \frnote{kilde mangler i hele dette afsnit}
I denne rapport vil vi undersøge havne og bådforeninger. Havne er et beskyttede vandområder, der gør det muligt for en båd at ligge fortøjret. I Danmark er der mange havne, og vi har haft store glæde af fiskeri og transport industrien. Efter en stærk stigning af privatejede lystbåde i 1960'erne og 1970'erne er der blevet bygget mange lystbådehavne. I disse havne kan en privat personer have deres både liggende. De fleste lystbådehavn og gennem tiden blevet udvidet med flere faciliteter som toilet, restaurant, legeplads, vaskeri og butikker. Lystbådehavne er typisk ejet af staten. Staten udlejer så hele havnen eller bestemte dele/pladser til foreninger. Foreninger lejer så pladserne videre til deres medlemmer. Det er derfor tit et krav at man skal være medlem af en forening for at få en fast plads i en lystbådehavn. Da foreningen har fuldt rådighed over pladserne i havnen kan den leje dem ud til gæster. Gæster er folk der komme forbi havnen på deres lystture. I nogle foreninger er der dedikerede pladser til gæste både, i andre foreninger bruges de tomme pladser som er tilbage efter nogen er på tur.  

Havnen styres typisk af en havnefoged. Havnefogeden står for opsyn med havneområdet og trafik i og omkring havnen. Det er også havnefogedens job at skaffe plads til alle både ved molerne. En stor del arbejdet går med administration. I danmark findes der ikke nogen uddannelse til havnefoged, men Foreningen af Lystbådehavne I Danmark tilbyder uddannelse i administration af en havn.

Foreninger udgør en stor rolle i mange danskeres liv. Disse foreninger er ofte drevet af frivillige sjæle, som tilbyder deres fritid til fællesskabet. Disse frivillige er motiveret af deres…

\section{Initierende problem}

Der kan gå meget tid med administrationsarbejde som kan bruges på ting. Derfor vil vores problem analyse tage udgangspunkt i dette initierende problem:

\textit{Hvilke problemer opstår i forbindelse med administrationen af en bådklub?}

Alternativ problem:
Hvilke muligheder er der for at forbedre administrations arbejdet i en sejlklub med en software løsning.

For at undersøge dette vil vi ud for at snakke med en fra administrationen i en bådklub. Vi vil bruge L og L til til at sætte et it system i kontekst. Vi vil komme frem til en problemformuleringen og lave et program der løser det formuleret problem.
