\subsection{MVC}
Model-view-controller (MVC) er et designmønster der bruges til at lave software med en brugergrænseflader \cite{mvcLecture}. Det opdeler en given software applikation i tre sammenhængende dele. Den centrale komponent, \enquote{Model} indeholder data og modellerer problemområdet. Et \enquote{View} står for den visuelle repræsentation af information. Den tredje del, \enquote{Controlleren}, implementere systemes funktioner. Fordelen ved MVC er at programmet for lav kobling og det blive let at skifte en del ud. Man kan for eksemple let skifte til et andet grafiskframwork ved bare at skifte viewet ud.

\frnote{læs igennem, evt mere}