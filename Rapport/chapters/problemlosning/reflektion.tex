\section{Refleksion}
\label{sec:refleksion}

\kanote{mangler noget omkring hvorfor refleksion}

Refleksion er en teknik i C\# til undersøgelse af objekter i et program under kørselstid \cite{michaelis2012essential}. Et eksempel på brugen af refleksion kan være, at programmøren ønsker at finde et felt i en klasse ved navn.

I dette projekt bruges refleksion til at gøre metoder der tilgår databasen, meget generiske, hvilket sikrer stor kodegenbrug. Som beskrevet i \cref{sec:database}, består databasen af forskellige tabeller, herunder \enquote{Users}, \enquote{Boats}, \enquote{BoatSpaces} og \enquote{Travels}. Lad os forestille os en metode der skal slette et objekt fra en specifik tabel. Metodesignaturen for en sådan metode kunne se ud som vist i \cref{lst:refleksion_remove1}.


\begin{lstlisting}[label=lst:refleksion_remove1]
public void RemoveUser(User user)
\end{lstlisting}

Problemet med denne metode er at denne metode kun virker på \enquote{User} objekter. Dette betyder at der udover denne metode også skal skrives en \enquote{RemoveBoat}, \enquote{RemoveBoatSpace} og en \enquote{RemoveTravel} metode. For at undgå dette, kan der laves en \enquote{Remove} metode, med en parameter der afgør hvilken tabel der skal slettes noget i. Metodesignaturen kunne se ud som i \cref{lst:refleksion_remove2}.


\begin{lstlisting}[label=lst:refleksion_remove2]
public void Remove<T>(T item, string table)
\end{lstlisting}

Problemet er, at metoden \enquote{Remove} skal have \enquote{hard-codet} alle de forskellige navne på alle eksisterende tabeller ind i denne metode. Hvis navnet på en tabel så ændrer sig, vil metoden ikke længere virke.

For at løse ovenstående problemer kommer refleksion ind i billedet. Metoden VerifyTable som ses i \Cref{lst:refleksion_verifytable} er et kodeeksempel fra programmet. VerifyTable bruger refleksion til at gennemsøge en klasse, her LobobContext, for at finde en egenskab af typen DbSet<T>. DbSet<T> er en databasetabel, der gemmer oplysninger af en arbitrær type \enquote{T}. Hvis den ønskede tabel bliver fundet, returneres denne. Hvis ingen tabel bliver fundet, returneres null. Først finder metoden alle egenskaber der er defineret på typen \enquote{LobobContext}. DBsetType er den egenskab der ønskes fundet. Metoden itererer nu alle egenskaber igennem, indtil en egenskab der matcher DBsetType er fundet. Hvis der er et match i foreach løkken, findes egenskab navnet, og lægges over i dbSetTarget. Det er nu muligt at dynamisk oprette en DbSet<T> ud fra typen \enquote{LobobContext}. Denne DbSet<T> returneres, og kan nu bruges i en \enquote{Remove} metode.

\begin{lstlisting}[label=lst:refleksion_verifytable, caption={Metode der tjekker om en tabel af typen T findes i databasen.}]
private DbSet<T> VerifyTable<T>(LobopContext context) where T : class
{
    Type lobobContextType = typeof(LobopContext);
    PropertyInfo[] properties = lobobContextType.GetProperties();
    Type DBsetType = typeof(DbSet<T>);
    string dbSetTarget = string.Empty;

    foreach (PropertyInfo item in properties)
    {
        if (DBsetType == item.PropertyType)
        {
            // table found
            dbSetTarget = item.ToString().Split(' ')[1];
            DbSet<T> dbSet = (DbSet<T>)lobobContextType.GetProperty(dbSetTarget).GetValue(context, null);
            return dbSet;
        }
    }
    // table of type not found
    return null;
}
\end{lstlisting}

\begin{lstlisting}[label=lst:refleksion_remove3, caption={Remove metode der kan tage en vilkårlig klasse ind, finde den rette tabel og derefter slette det parametiserede objekt}]
  public void Remove<TInput>(TInput item) where TInput : class
  {
      using (var db = new LobobContext())
      {
          DbSet<TInput> dbSet = VerifyTable<TInput>(db);
          if (dbSet == null) throw new KeyNotFoundException("table ikke fundet");
          dbSet.Remove(item);
          db.SaveChanges();
      }
  } 
\end{lstlisting}

Den færdige Remove metode der er generisk, virker nu på tværs af alle klasser, ved hjælp af refleksion. En forsimplet implementation der benytter VerifyTable af Remove metoden ses i \cref{lst:refleksion_remove3}. Hvis en Titanic båden skal fjernes fra \enquote{Boats} tabellen, kaldes koden set i \cref{lst:refleksion_remove4}. Der er nu ingen behov for at specificere i hvilken tabel Titanic skal fjernes fra, da Remove selv kan regne dette ud, ud fra den generiske typeparameter <Boat>.

\begin{lstlisting}[label=lst:refleksion_remove4]
Remove<Boat>(titanic);
\end{lstlisting}
