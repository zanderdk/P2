\section{Klasse Design}
\label{sec:klasse_design}
C\# er et objektorienteret programerings sprog. I objektorienteret programerings sprog er det let at dele programmet op og på den måde lave et mere organiseret program, der er mere vedligeholdeligt?. Og en af grudpillerne bag er som navnet antyder objekter. For at lave et objekt skal man definere klasser, som er skabelonen for et objekte. Klasser udgør en abstraktion eller model, der ofte er et koncept fra den virkelige verden. Der skal derfor laves en model, der kan anvendes i forbindelse med løsningen af problemformuleringen.

Dette afsnit vil beskrive hvilke valg der er taget i forbindelse med modelleringen af forskellige koncepter til programmet.

\subsection{UML-klassediagram}
For at få et overblik over modellering er der lavet et UML-klassediagram. Et UML-klassediagram er et diagram der beskriver strukturen i et system ved at vise systemet klasser og deres relationer.


\frnote{Her skal der være et billede at vores flotte UML diagram}

\subsection{Brugere af Programmet}
\label{sub:brugere_af_programmet}

I klasse hierakiets top ligger den abstrakte klasse \enquote{User}. Denne klasse definerer alle fællestræk de efterfølgende specifikke klasser skal have. Dette inkluderer brugerrettigheder og basale informationer som telefonnummer, navn og adresse.

Subklasserne til \enquote{User} opdeler brugere som medlemmer af klubben, gæster og havnefogeden. Denne opdeling er lavet fordi der indgår forskellige felter på tværs af disse subklasser. For at differentiere subklasserne, implementerer disse forskellige interfaces. F.eks. må en havnefoged ikke have en båd, og derfor implementerer klassen \enquote{HarbourMaster} ikke \enquote{ISailer} interfacet. Medlemmer af klubben og gæster må derimod gerne have mulighed for at have en båd, og implementerer derfor \enquote{ISailer}.

\sinote{Mere..}
