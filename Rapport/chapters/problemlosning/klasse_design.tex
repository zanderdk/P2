\section{Klasse Design}
\label{sec:klasse_design}
C\# er et objektorienteret programmeringssprog. I objektorienteret programmeringssprog er det let at dele programmer op, og på den måde lave mere velorganiserede programmer, der er nemmere vedligeholdes. En af grundsøjlerne bag det objektorienterede paradigme, er objekter. For at lave et objekt skal man definere en klasse, som er skabelonen for objektet. Klasser er ofte en abstraktion eller en modellering, af et koncept eller gestande fra den virkelige verden. Da det objektorienterede programmeringsparadigme ønskes anvendt, skal en model konstrueres, der kan anvendes i forbindelse med løsningen af problemformuleringen. I denne løsning ønskes der en modellering af de koncepter og genstande, der findes på en havn, og dette er derfor blevet undersøgt. For hvert relevant koncept og genstand, er der indgået overvejelser, omkring den optimale måde at modellere dem i klasser.

I følgende afsnit beskrives hvilke valg der er taget i forbindelse med modelleringen af forskellige koncepter og genstande i løsningen. 

\subsection{UML-klassediagram}

For at bedre kunne opnå en optimal modellering, sættes klasser ind i et klassehierarki for at øge overskueligheden. Et klassehierarki er en organisering af et sæt af klasser og deres indbyrdes relationer. Et klassediagram beskriver opbygningen af et system ved at vise systemets klasser, deres attributter, metoder, og relationerne mellem objekter \cite{martin2006agile}. For at give et overblik over klassehierarkiet er der lavet et klassediagram efter UML standarden. UML-klassediagram for løsningen kan ses herunder.

\frnote{Her skal der være et billede at vores flotte UML diagram}

% Kasper har læst igennem og rettet hertil

\subsection{Brugere af Programmet}
\label{sub:brugere_af_programmet}

I klassehierarkiets top ligger den abstrakte klasse \enquote{User}. En \enquote{User} er en generel bruger af systemet. Denne klasse definerer alle fællestræk de efterfølgende specifikke klasser skal have. Dette inkluderer brugerrettigheder og basale informationer som telefonnummer, navn og adresse.

Der er tre forskellige interfaces der indeholder forskellige informationer om brugere, IFullPersonalInfo, ISailor, ILoginable

Subklasserne til \enquote{User} opdeler brugere som medlemmer af klubben, gæster og havnefogeden. Til disse tre person typper bruges klasserne \enquote{HarbourMaster}, \enquote{Guest} og \enquote{Member}. Denne opdeling er lavet fordi der indgår forskellige felter på tværs af disse subklasser. For at differentiere subklasserne, implementerer disse forskellige interfaces. F.eks. må en havnefoged ikke have en båd, og derfor implementerer klassen \enquote{HarbourMaster} ikke \enquote{ISailer} interfacet. Medlemmer af klubben og gæster må derimod gerne have mulighed for at have en båd, og implementerer derfor \enquote{ISailer}. Derudover er der interfacet IFullPersonalInfo som som includere alle de informationer man skal havde om et medlem men som gæster for.eks. ikke skal have. Gæster skal ikke havde et password til at logge ind med derfor implementere klassen i I logiable. Permissions vil der blive talt om senere.

En rejse modelleres med klassen Travel. Den indelholder information om rejsen start- og slutdato. Rejser bruges når et medlem er ude for at rejse. Rejser bruges også til at angive hvor lang tid en gæst ligger i havnen. Travel implementere IEquatable som gør det muligt at sammenligne med andre klasser af samme type. Klasser der implementere ISailor indenholder en list af Travels.

\subsection{Både?}
\label{sub:bade}

Der er en klasse til både der hedder Boat. Den bruges til at geme data tilhørende en båd. Der gemmes dens navn og størelse, så man kan finde en plads der er stor nok.

Der er to klasser til er repræsentere lejepladser. Der er WaterSpace som er en vandlegeplads og LandSpace som er en landlejeplads. Begge disse klasser ned arveler fra klassen BoatSpace, som indenholder generalle informationer om en plads.

\frnote{ syntax i forbindelse med navn af klasser og interfaces}
