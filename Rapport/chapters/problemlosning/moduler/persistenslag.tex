\subsection{Persistenslag}
\label{sub:persistenslag}

\subsubsection{Funktionalitet}
\label{ssub:Funktionalitet}

Persistenslag modullet har til opgave at manipulere og læse data fra en datastruktur. En datastruktur kunne f.eks. være en database eller en anden fil på harddisken. Modullet tilbyder basale CRUD (create, read, update og delete) operationer, hvorfra alle tænkelige persistenslagsoperationer kan opbygges af.

Et eksempel på brugen af persistenslag modullet, kunne være login modulet. Login modulet skal verificere om et indtastet brugernavn matcher et indtastet kodeord. Til dette kan login modulet tilgå databasen igennem persistenslagets read operation, og herfra arbejde videre på den fundne data.


\subsubsection{Overordnet implementation}
\label{ssub:Implementation}

Persistenslag modullet er opbygget således at selve datastrukturen der manipuleres og læses fra, kan ændres. Herved indkapsles tilgangen til datastrukturen på en sådan måde, at et program kan gå fra at benytte en xml fil som data lager, til at benytte en database uden at ødelægge anden eksisterende kode.

Dette implementeres ved at persistenslag modullet er bevidst om en konkret implementation af tilgangen til en datastruktur. Når et andet modul ønsker at lave en læse operation på persistenslag modullet, videredelegeres denne operation til den underliggende implementation af tilgangen til en datastruktur.
