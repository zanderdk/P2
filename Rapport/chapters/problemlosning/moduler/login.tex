\subsection{Login}
\label{sub:login}

Det første modul der interageres med, er login modulet, som har til formål at validere, og videresende brugeren til programmets hovedfunktioner. 

\subsubsection{Funktionalitet}
\label{ssub:Funktionalitet}

Login modulet har det primære ansvar for at angive adgangsniveauer, og sikre databeskyttelse. Fra login modulet kan man vælge at oprette sig som ny gæst, eller logge ind i systemet. Hvis brugeren vælger at oprette sig som ny gæst, så vil brugeren blive sendt til et separat modul.

\subsubsection{Implementation}
\label{ssub:login_implementation}

Login modulet består af et standby- , medlemslogin-, validerings-, samt videresendings-element.

Standby-elementet er udgangspunktet for al interaktion med programmet, og har henvisninger til hvordan brugeren anvender programmet. 

Fra standby-elementet kan brugeren vælge at gå til medlemslogin-elementet, hvor en bruger der er medlem kan logge sig ind med sit medlemsnummer, og et kodeord. Herfra bør man også kunne vende tilbage til standby-elementet, i tilfælde af fejl.

Alternativt kan brugeren indsætte et chipkort fra standby-elementet, og så sender standby-elementet brugeren direkte til validerings-elementet.

Medlemsnummeret og kodeordet bliver videresendt til validerings-elementet, som krydstjekker informationerne via databasemodulet. Validerings-elementet videresender herefter en boolsk statusmelding, samt en brugerprofil fra databasen til videresendings-elementet.

Videresendings-elementet tjekker status meldingen. Hvis meldingen er falsk, sender den brugeren tilbage til enten medlemslogin eller standby, med en passende fejlmelding. Hvis valideringen gik godt, så videresendes brugeren til Hovedmodulet.