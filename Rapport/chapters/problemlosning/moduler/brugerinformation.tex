\subsection{Brugerinformation}
\label{sub:Brugersinformation}

Brugerinformationsmodulet har til opgave at kommunikere mellem brugeren og databasen.

\subsubsection{Funktionalitet}
\label{ssub:brugerinformation_funktionalitet}

Kommunikation mellem databasen og brugeren indebærer, at vise information omkring en bruger, modtage inputs fra brugeren samt læse fra og skrive i databasen. At læse fra databasen indebærer, at den relevante information bliver hentet fra databasen, og vist på en acceptabel måde. Når der skrives til databasen menes der, at brugerens input bliver gemt i databasen. 

\subsubsection{Implementation}
\label{ssub:brugerinformation_implementation}

For at implementere dette består brugerinformationsmodulet af et båd-, medlem- og rejse-delmodul. Delmodulerne har hvert deres tilsvarende ansvarsområde og er designet til at kunne læse, tilføje, redigere eller slette data fra databasen ud fra brugerens input. Læsningen af data foregår ved at et delmodul viser dets data i de respektive tekstfelter. For at redigere eller tilføje data, bliver der benyttet popup vinduer. Ved sletning af data markeres det ønskede elementer og der klikkes på fjern-knappen. Ved at bruge popup vinduer, kan brugeren lettere kan skelne mellem at læse data og skrive data.