\chapter{Diskussion}
\label{cha:diskussion}

\alnote{skriv om sikkerheds afgrænsning}

Dette afsnit diskuterer nogle udvalgte dele af projektet, for at beskrive tilblivelsen, eller udeblivelsen, af disse dele, samt nogle af de tanker og overvejelser der har påvirket beslutningerne.

\section{Anvendelse af systemet} 
\label{sec:anvendelse_af_systemet}
Systemet er tiltænkt som et program der kan tilgå fra havnen ved brug af en stander eller lignende, begrundet at programmet inderholder funktionaliter som kun kan benyttes efter scanning af et chipkort. Et eksempel på en sådan funktionalitet er at gæster er begrænset til kun at kunne loge ind ved brug af et chipkort. Andre funktioner er af sikkerhedsmæssige årsager bundet til kun at forkomme på havnen eller gennem et administrator login. Det bør kun være muligt at registrer sig som gæst fra havnens standere grundet at funktionen kan misbruges forudsat at det er muligt at oprette gæster undenfor havnen. 

Andre af programmets funktioner bør havnens medlemmer kunne tilgå fra andre steder. Administrering af egne både bør medlemmer kunne gør uafhængigt af tilgang til standeren på havnen. Registerenig og ændringer af rejser kunne også være fordelagtigt at gøre tilgængeligt via fjerneadgang. 



% section anvendelse_af_systemet (end)

\section{Adgangsbegrænsninger}

En stor del af programmets funktionalitet, udgøres af muligheden for forskellige adgangsniveauer til forskellige brugere. Som beskrevet i \cref{tilladelser}, er dette implementeret med forskellige tilladelseskategorier. Disse kategorier kan have ét af tre niveauer: ingen-, læse- eller skrive-tilladelse. Med denne model, er det let at dynamisk ændre en bestemt brugers tilladelser. Som et alternativ til denne løsning, kunne der have eksisteret flere subklasser, med faste adgangsniveauer, således at adgangsniveauerne automatisk er sat, ved instantieringen af klassen. Dette havde simplificeret brugeroprettelsen, men det blev vurderet at fleksibiliteten ved oprettelser og vedligeholdelse var at foretrække.

\section{MVC Under Systemdesignprocessen}

Ved udviklingsforløbets start, var \enquote{Model-View-Controller}, herefter kaldt MVC, ikke en bekendt teknik. Derfor indgik det ikke i den grundlæggende designprocess, hvilket medførte, at en betydelig del af funktionaliteten blev kodet tæt sammen med brugergrænsefladen. Der er efterfølgende blevet foretaget refaktoreringer, der har haft til formål at frigøre funktionaliteten fra brugergrænsefladen. Samtidigt blev det besluttet, at det originale modulære design skulle bibeholdes, da det ikke påførte mærkbare begrænsninger, samt grundet projektets begrænsede tidshorisont. Denne beslutning bidrog til behovet for en selvudviklet diagramform, som kan ses i \cref{sec:moduler}, da det blev vurderet, at anvendelse af en etableret diagramstandard, ville være misvisende, da en sådan standard ikke lå til grund for udviklingen. Dertil kunne denne misvisning tolkes, som en misforståelse af MVC udviklingsformen.

\section{Anvendelse af Test Driven Development}

For at opnå den fulde effekt af \enquote{Test Driven Development}, herefter kaldt TDD, som står beskrevet i \cref{sub:test_driven_development}, bør det følges konsekvent igennem hele udviklingsforløbet. Der opstod dog under forløbet konsensus om, at afvige fra en konsekvent afvikling af denne udviklingsform. Begrundelsen for denne konsensus, ligger i en estimeret begrænset tidsvinding, hvor udbyttet af flere test vil være utilstrækkeligt. Kombineret med en kort tidshorisont for udviklingsperioden, blev det derfor vurderet fordelagtigt, at nedprioritere TDD. Dette åbner op for en bedre disponering af tiden, således at der opnås en bredere forståelse for flere aspekter af softwareudvikling.

\kanote{Ting vi ville lave anderledes. Forskellige dele af rapporten. herunder ikke færdige emner}

For at differentiere de forskellige typer af brugere til programmet, blev der besluttet at konstruere 3 grundlæggende klasser, Member, Guest og HarbourMaster. Dette giver nogle logiske fordele
