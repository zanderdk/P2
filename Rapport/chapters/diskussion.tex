\chapter{Diskussion}
\label{cha:diskussion}

Dette afsnit diskuterer nogle udvalgte dele af projektet, for at beskrive tilblivelsen, eller udeblivelsen, af disse dele, samt nogle af de tanker og overvejelser der har påvirket beslutningerne.

\section{Idriftsættelse af Løsningen} 

\kanote{skriv om stander}

Systemet er tiltænkt anvendelse på en havn, hvor programmet afvikles på en stander eller anden forudinstalleret hardware. Programmet indeholder funktionaliteter, som kun kan benyttes ved brug af speciel hardware. Eksempelvis er gæster begrænset til at logge ind ved brug af et chipkort. Andre funktioner bør, af sikkerhedsmæssige årsager, være bundet til udelukkende at blive anvendt på forudbestemt hardware eller gennem et administrator login, da systemet er sårbart overfor destruktivt misbrug af programmet. Her tænkes på muligheden for at oprette sig som gæst i systemet, en funktionalitet, der i nuværende tilstand, er tilgængelig uden adgangsgodkendelse eller andre begrænsninger.

Andre af programmets funktioner, kunne derimod med fordel tilgås, af verificerede brugere, fra andre steder og med andet hardware. Administrering af egne både bør medlemmer kunne gøre, uafhængigt af tilgang til standeren på havnen. Registreringer og ændringer af rejser, kunne også fordelagtigt være tilgængeligt via fjernadgang.

Hvis en sådan fjernadgang skulle implementeres, vil det yderligere kræve en revurdering af det nuværende valideringssystem, da der ikke foretages nogen kryptering, hashing eller anden forplumring af den data der sendes til og fra databasen. En oplagt løsningsmodel ville være, at foretage tjek på adgangsniveauer ved hver tilgang til databasen.

\section{Adgangsbegrænsninger}

En stor del af programmets funktionalitet, udgøres af muligheden for forskellige adgangsniveauer til forskellige brugere. Som beskrevet i \cref{tilladelser}, er dette implementeret med forskellige tilladelseskategorier. Med denne model, kan man dynamisk ændre brugerenes tilladelser. Som et alternativ til denne løsning, kunne der have eksisteret flere subklasser, med faste adgangsniveauer, således at adgangsniveauerne automatisk sættes, ved instantieringen af klassen. Dette havde simplificeret brugeroprettelsen, men det blev vurderet at fleksibiliteten ved oprettelser og vedligeholdelse var at foretrække. 

\section{MVC Under Systemdesignprocessen}

Ved udviklingsforløbets start, var \enquote{Model-View-Controller}, herefter kaldt MVC, ikke en bekendt teknik. Derfor indgik det ikke i den grundlæggende designprocess, hvilket medførte, at en betydelig del af funktionaliteten blev kodet tæt sammen med brugergrænsefladen. Der er efterfølgende blevet foretaget refaktoreringer, der har haft til formål at frigøre funktionaliteten fra brugergrænsefladen. Samtidigt blev det besluttet, at det originale modulære design skulle bibeholdes, da det ikke påførte mærkbare begrænsninger, samt grundet projektets begrænsede tidshorisont. Denne beslutning bidrog til behovet for en selvudviklet diagramform, som kan ses i \cref{sec:moduler}, da det blev vurderet, at anvendelse af en etableret diagramstandard, ville være misvisende, da en sådan standard ikke lå til grund for udviklingen. Dertil kunne denne misvisning tolkes, som en misforståelse af MVC udviklingsformen.

\section{Anvendelse af Test Driven Development}

For at opnå den fulde effekt af \enquote{Test Driven Development}, herefter kaldt TDD, som står beskrevet i \cref{sub:test_driven_development}, bør det følges konsekvent igennem hele udviklingsforløbet. Der opstod dog under forløbet konsensus om, at afvige fra en konsekvent afvikling af denne udviklingsform. Begrundelsen for denne konsensus, ligger i en estimeret begrænset tidsvinding, hvor udbyttet af flere test vil være utilstrækkeligt. Kombineret med en kort tidshorisont for udviklingsperioden, blev det derfor vurderet fordelagtigt, at nedprioritere TDD. Dette åbner op for en bedre disponering af tiden, således at der opnås en bredere forståelse for flere aspekter af softwareudvikling.

\section{Informationsgivende Funktioner}

I projektets løsning er der adskillelige informationsgivende funktioner til rådighed. Kortfunktionen giver muligheden for et hurtigt overblik over havnen, og søgefunktionen gør det nemt at finde information om brugere. Disse funktioner muliggør dog ikke kommunikation rettet imod brugere af programmet. En fremtidig funktionalitetstilføjelse kunne derfor være et besked-modul. Dette ville åbne op for, at kunne levere notifikationer eller advarsler til brugere, enten via en form for opslag, eller direkte ved realtidsbeskeder. Realtidsbeskeder vil også muliggør implementering af et chat-modul, hvis dette ønskes af brugerne.

\kanote{Ting vi ville lave anderledes. Forskellige dele af rapporten. herunder ikke færdige emner}
