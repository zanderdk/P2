\chapter{Diskussion}
\label{cha:diskussion}

Dette afsnit diskutterer nogle udvalgte dele af projektet, for at beskrive tilblivelsen af disse dele. 

En stor del af programmets funktionalitet består i at have forskellige adgangsniveauer til forskellige brugere. Som beskrevet i \cref{tilladelser}, er dette implementeret med 3 grader af tilladelser. Ingen, læse eller skrive tilladelser. Med denne model, er det let at give en bestemt bruger flere eller færre tilladelser, dynamisk.  Dog defineres de forskellige adgangsniveauer ikke udelukkende på baggrund af \enquote{3-grads-modellen}.

For at differentiere de forskellige typer af brugere til programmet, blev der besluttet at konstruere 3 grundlæggende klasser, Member, Guest og HarbourMaster. Dette giver nogle logiske fordele

Ting vi ville lave anderledes. Forskellige dele af rapporten.

F.eks. harbourmaster vs permissions

MVC mere af det.
