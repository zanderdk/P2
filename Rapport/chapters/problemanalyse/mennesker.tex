% Kassereren

% sekretæren

% formanden & næstformanden

% fartøjsinspektøren

% aktivitetslederen

% medlemmer - diversiteten

% gæster

% havnefogeden

%!TEX root = ../../Master.tex
\section{Mennesker}
\sinote{mangler indlende tekst}
\frnote{, hvor vi husker at sige det er intressanter}
\subsection{Medlemmer}

Den vigtigste interessent for en forening er dens medlemmer. Enhver handling udført af en forening, bør i sidste ende betyde en fordel for foreningens medlemmer. Det er derfor vigtigt at skabe sig et godt indtryk af foreningens medlemmer, hvis der ønskes at lave en løsning til en given forening.


\subsubsection{Medlemskaber}

Når det kommer til Vestre Baadelaug og Sejlklubben Limfjordens medlemmer, er der tale om et bredt segment. Generelt for alle medskabere af Vestre Baadelaug gælder det, at der kun kan lejes pladser ud til én båd per medlemskab. Hvis båden har flere ejere skal de alle have et medlemsskab. Dette er dog noget Vestre Baadelaug ikke kontrollerer, da de ikke ønsker at kræve købskontrakter af medlemmerne. Det bør dog nævnes at et medlemsskab i Vestre Baadelaug dækker ejerens husstand, således at børn og ægtefælle er fritaget fra at have deres eget medlemsskab. Sejlklubben Limfjorden kræver ét medlemsskab per næse, også selvom de er fra samme husstand.

Vestre Baadelaug vil som grundregel undgå passive medlemmer, det værende medlemmer uden båd, hvor Sejlklubben Limfjorden kræver et passivt medlemsskab hvis personen ikke er ejer af båden der sejles.

Set i praktiske forhold, så kan nogle af disse regler midlertidigt ignoreres, hvis der er formildende omstændigheder. Dette kunne for eksempel være i tilfælde af køb og salg af båd hvor, i en periode, ét medlem kan have flere både liggende i havnen. Hvis sådan en situation strækker sig for langt, vil klubberne kræve et nyt medlemsskab.
\sinote{kasper fix situation strækker sig}

\subsubsection{Medlemmernes demografi}

Her følger en beskrivelse af de to klubbers medlemmer, fra et demografisk synspunkt.

Vestre Baadelaug har en bred demografi. Dette skyldes i stor grad få restriktioner på både der tillades i klubben. Sejlklubben Limfjorden tillader kun sejlbåde, og har af denne grund en smallere demografi. Ifølge Peter Hinrup og Dorte Carøe, som repræsenterer henholdsvis Vestre Baadelaug og Sejlklubben Limfjorden, er dette den direkte grund til at Vestre Baadelaug har en større udskiftning af medlemmer, og at der oftere opstår konflikter imellem medlemmerne i Vestre Baadelaug.

\kanote{jeg ved ikke helt om jeg kan tillade mig at lave denne konklusion}

Visse ting er dog fælles for begge klubber. Aldersmæssigt er der fra unge medlemmer i tyverne, op til pensionister og efterlønnere. Der er enker og børnefamilier. Der er både registreret til firmaer, og ifølge Peter Hinrup er det også et miljø, som tiltrækker folk der \enquote{ikke er ædruelige døgnet rundt} og \enquote{folk som er relativt tørstige}.

\kanote{overvej tilføjelser og lav en opdeling af klubber (aldersgruppen er forskellige!!!)}

\subsubsection{Det Sociale Aspekt}

For mange af medlemmerne, repræsenterer deres båd den største investering de har foretaget i deres liv. Mange af medlemmerne deltager også aktivt i de arrangementer der bliver organiseret i havnen, og har erhvervet store dele af deres sociale omgangskreds igennem Vestre Baadelaug. En begrænset mængde af medlemmerne har bopæl i havnen, og bor året rundt i deres båd. Mange andre bruger båden som et kolonihavehus, hvor de bruger deres weekender i sommerhalvåret. Alt dette er aspekter som klubben støtter op om, ud fra et ønske om, at der altid er liv i havnen.



\kanote{et afsnit omkring de sociale ting. Forskellige måder at bruge sin båd på (kolonihavehus etc.). Betydningen af hvilke plads medlemmerne har. Medlemmernes prioriteter. Sociale arrangementer. Fastboende og klubbens holdning. Ønsket om liv i havnen}


\subsection{Gæstesejlere}

Den anden største indtægtskilde for Vestre Baadelaug efter vandlejepladser, er gæstesejlere, som betaler leje når de anvender Vestre Baadelaugs faciliteter. Dette udgør 14 \% af de samlede indtægter for Vestre Baadelaug \cite{vestre_arsregnskab}, hvilket derfor af samme grund repræsentere et væsentligt interesseområde for Vestre Baadelaug.

Gæstesejlerene er en meget bred demografi, med folk fra mange forskellige lande. Primært nævnes Danmark, Norge, Sverige, Tyskland, Holland og Polen. Fælles for disse er deres behov for en havn, hvori deres fartøj kan fortøjes, samt adgang til diverse faciliteter. Det forventes at der er adgang til toilet, bad, afskaffelse af affald, strøm. Det er essentielt for Vestre Baadelaugs interesser og gæster, at disse faciliteter er i ordentlig stand, da Vestre Baadelaug repræsenterer Danmark på et internationalt plan.

Gæsterne hos Vestre Baadelaug betaler for deres forbrug af plads, strøm samt diverse serviceydelser. Derfor er det vigtigt for Vestre Baadelaug, at deres service og faciliteter er tilfredsstillende for gæsterne. Udover at se gæsterne som en målgruppe, er det også en eksemplarisk handling fra Vestre Baadelaug side. Det er et forsøg på at forbedre bådmiljøet, således at den gode behandling vil blive reflekteret af andre havne, og på den måde give igen til Vestre Baadelaugs medlemmer når de sejler ud.

\kanote{afsnittet skal indeholde: Demografi omkring gæster, (nationalitet bl.a.). m.m. (?)}

\subsection{Havnefogeden}

Havnefoged, Per Jensen, er ansat af Vestre Baadelaug, og Sejlklubben Limfjorden betaler Vestre Baadelaug for arbejde udført på deres områder. Havnefogeden står for den daglige drift af havnen, hvilket indkluderer vedligeholdelse af grønne arealer, daglig rengøring og oprydning, samt mindre reparationer. Han skal dertil fungere som et service organ, der kan hjælpe medlemmer og gæster der har spørgsmål, eller mindre problemer. Havnefogeden beskriver selv sin funktion som værende at mindske bestyrelsens arbejde mest muligt, hvilket han gør ved at tage problemerne i opløbet \cite{int_hf}.

Havnefoged er et sæsonpræget job. Arbejdet er primært i sommerhalvåret, hvor der er bådsæson, med et ekstra pres i juni, juli og august. I vinterhalvåret holdes der ferier, og der afspadseres så meget som muligt, for at kunne arbejde mere om sommeren. I sæsonen er der behov for at havnefogeden skal være tilgængelig de fleste timer af døgnet, og skal kunne møde på havnen i akuttilfælde, med en reaktionstid på omtrent en halv time eller mindre.

Havnefogeden står for håndteringen af gæstesejlere. Dette betyder at han skal kontrollere hvorvidt gæstesejlere har betalt for leje af vandplads, og hjælpe med at diverse problemer. Han skal også kunne tage imod betaling fra gæster som ikke anvender billet maskinen. I forbindelse med gæstesejlere, skal han også administrere det system der bruges til at repræsentere pladsernes status. Pladsernes status omhandler hvorvidt et medlems plads kan lejes ud til gæster, eller hvorvidt medlemmet snart skal bruge pladsen igen. Dette system består af en simpel plade eller klods ved hver plads der kan vendes, så der enten vises en grøn eller en rød side. Den grønne side betyder at pladsen er fri, og den røde betyder at ejeren af pladsen snart skal bruge den igen. Medlemmer kan meddele havnefogeden om hjemkost, således at havnefogeden kan frigøre deres plads, inden de er i havnen. Dette sker via telefon eller forgående mundtlig aftale.

\kanote{havnefoged: skal kigges igennem.. er det stadig tyndt?}

