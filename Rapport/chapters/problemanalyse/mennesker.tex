%!TEX root = ../../Master.tex
\section{Mennesker}

I det følgende afsnit vil de kontekstuelle forhold for det, som beskrives af Laudon \& Laudon som \enquote{mennesker}, blive analyseret. De vigtigste interessenter til en mulig løsning, vil blive beskrevet. Derudover vil disse interessenters relation til løsningen blive analyseret. Når man skal designe en løsning til en brugergruppe, er det relevant at kende til brugerenes forudsætninger, samt deres værdinormer. Dette afsnit er delvist forbundet med organisations afsnittet, og bør læses i forlængelse af hinanden, for at opnå en samlet forståelse. Der vil blive samlet op på dette afsnit i \cref{sec:refleksioner}

\subsection{Medlemmer}

Den vigtigste interessent for en forening er dens medlemmer. Enhver handling udført af en forening, bør i sidste ende betyde en fordel for foreningens medlemmer. Det er derfor vigtigt at skabe sig et godt indtryk af foreningens medlemmer, hvis der ønskes at lave en løsning til en given forening.

\subsubsection{Medlemskaber}

Når det kommer til Vestre Baadelaug og Sejlklubben Limfjordens medlemmer, er der tale om et bredt segment. Generelt for alle medskabere af Vestre Baadelaug gælder det, at der kun kan lejes pladser ud til én båd per medlemskab \cite{vestre_vedtagter}. Hvis båden har flere ejere skal de alle have et medlemsskab. Dette er dog noget Vestre Baadelaug ikke kontrollerer, da de ikke ønsker at kræve købskontrakter af medlemmerne. Det bør dog nævnes at et medlemsskab i Vestre Baadelaug dækker ejerens husstand, således at børn og ægtefælle er fritaget fra at have deres eget medlemsskab. Sejlklubben Limfjorden kræver ét medlemsskab per næse, også selvom de er fra samme husstand \cite{int_vb_sl}.

I Vestre Baadelaug har det været muligt at blive passivt medlem, hvis man ønskede at forblive medlem uden at have en båd, men de optager ikke nye passive medlemmer \cite{vestre_vedtagter}. I modsætning til Vestre Baadelaug, hvor der kun kræves medlemsskab af ejeren af en given båd, er det i Sejlklubben Limfjorden nødvendigt for alle der vil sejle båden mindst at være passivt medlem \cite{int_vb_sl}.

Set i praktiske forhold, så kan nogle af disse regler midlertidigt ignoreres, hvis der er formildende omstændigheder. Dette kunne for eksempel være i tilfælde af køb og salg af båd hvor, i en periode, ét medlem kan have flere både liggende i havnen. Hvis sådan en situation forekommer ofte eller over en længere periode, vil klubberne kræve et nyt medlemsskab \cite{int_vb_sl}.

\subsubsection{Medlemmernes demografi}

Her følger en beskrivelse af de to klubbers medlemmer, fra et demografisk synspunkt, begrundet i \cite{int_vb_sl}.

Vestre Baadelaug har en bred demografi. Dette skyldes i stor grad få restriktioner på både der tillades i klubben. Sejlklubben Limfjorden tillader kun sejlbåde, og har af denne grund en smallere demografi. Ifølge Peter Hinrup og Dorte Carøe, som repræsenterer henholdsvis Vestre Baadelaug og Sejlklubben Limfjorden, er dette den direkte grund til at Vestre Baadelaug har en større udskiftning af medlemmer, og at der oftere opstår konflikter imellem medlemmerne i Vestre Baadelaug. Aldersmæssigt er der hos Vestre Baadelaug fra unge medlemmer i tyverne, op til pensionister og efterlønnere, hvor Sejlklubben Limfjorden i højere grad har ældre medlemmer. Visse ting er dog fælles for begge klubber. Der er enker og børnefamilier. Der er både registreret til firmaer, og ifølge Peter Hinrup er det også et miljø, som tiltrækker folk der \enquote{ikke er ædruelige døgnet rundt} og \enquote{folk som er relativt tørstige}.

\subsubsection{Det Sociale Aspekt} \label{subs:social}

For mange af medlemmerne, repræsenterer deres båd den største investering de har foretaget i deres liv. Mange af medlemmerne deltager også aktivt i de arrangementer der bliver organiseret i havnen, og har erhvervet store dele af deres sociale omgangskreds igennem Vestre Baadelaug. En begrænset mængde af medlemmerne har bopæl i havnen, og bor året rundt i deres båd. Mange andre bruger båden som et kolonihavehus, hvor de afsætter deres weekender i sommerhalvåret. Der er medlemmer som aldrig sejler ud af havnen, og nogle som aldrig forlader limfjorden \cite{int_vb_sl}.

Vestre Baadelaug ønsker at støtte op omkring liv i havnen og bidrage til et godt miljø for bådejere af flere årsager. Kasserer Peter Hinrup fortæller: \enquote{Det sociale er et stort aspekt \ldots specielt sejlture og fester}. Først og fremmest ønsker de at styrke de sociale bånd imellem bådejere, og give folk med interesse i både, mulighed for at møde andre ligesindede. De ønsker at skabe et miljø, hvor man kan dele historier og erfaring. Vestre Baadelaug betragter sig selv som en del af foreningsdanmark, og vil som en gammel forening, holde den danske tradition for foreninger i live. Dette betyder at der skal være plads til så mange typer som muligt \cite{int_vb_sl}.

Et sekundært mål, ved at holde liv i havnen, er at give medlemmerne en øget tryghed med hensyn til deres båds sikkerhed. Vestre Baadelaug mener at man i nogen grad kan forebygge ubudne gæster, ved at skabe et sammenhold, således at medlemmerne holder øje med hinandens både \cite{int_vb_sl}.


\subsection{Gæstesejlere}\label{sec:gaste}

Den anden største indtægtskilde for Vestre Baadelaug efter vandlejepladser, er gæstesejlere, som betaler leje når de anvender Vestre Baadelaugs faciliteter. I Vestre Baadelaug har de ikke en dedikeret bro til gæstesejlere. I stedet bruges de pladser, der bliver ledige når et medlem tager på en længere sejltur. Gæstesejlere udgør 14 \% af de samlede indtægter for Vestre Baadelaug \cite{vestre_arsregnskab}, hvilket derfor af samme grund repræsenterer et væsentligt interesseområde for Vestre Baadelaug.

Gæstesejlerene er en meget bred demografi, med folk fra mange forskellige lande. Primært nævner Peter, Danmark, Norge, Sverige, Tyskland, Holland og Polen. Fælles for disse er deres behov for en havn, hvori deres fartøj kan fortøjes, samt adgang til diverse faciliteter. Det forventes at der er adgang til toilet, bad, strøm og afskaffelse af affald. Det er essentielt for Vestre Baadelaugs interesser og gæster, at disse faciliteter er i ordentlig stand, da Vestre Baadelaug repræsenterer Danmark på et internationalt plan \cite{int_vb_sl}.

Gæsterne hos Vestre Baadelaug betaler for deres forbrug af plads, strøm samt diverse serviceydelser. Derfor er det vigtigt for Vestre Baadelaug, at deres service og faciliteter er tilfredsstillende for gæsterne. Udover at se gæsterne som en målgruppe, er det også en eksemplarisk handling fra Vestre Baadelaug side. Det er et forsøg på at forbedre bådmiljøet, således at den gode behandling vil blive reflekteret af andre havne, og på den måde give igen til Vestre Baadelaugs medlemmer når de sejler ud \cite{int_vb_sl}.


\subsection{Havnefogeden} \label{sub:havnefoged}

Havnefoged, Per Jensen, er ansat af Vestre Baadelaug. Sejlklubben Limfjorden betaler Vestre Baadelaug for arbejde udført på deres områder. Følgende afsnit er skrevet ud fra et interview med Per Jensen \cite{int_hf}. Havnefogeden står for den daglige drift af havnen, hvilket indkluderer vedligeholdelse af grønne arealer, daglig rengøring og oprydning, samt mindre reparationer. Han skal dertil fungere som et serviceorgan, der kan hjælpe medlemmer og gæster der har spørgsmål, eller mindre problemer. Havnefogeden beskriver selv sin funktion som værende at mindske bestyrelsens arbejde mest muligt, hvilket han gør ved at tage problemerne i opløbet \cite{int_hf}.

Havnefoged er et sæsonpræget job. Arbejdet er primært i sommerhalvåret, hvor der er bådsæson, med et ekstra pres i juni, juli og august. I vinterhalvåret holdes der ferier, og der afspadseres så meget som muligt, for at kunne arbejde mere om sommeren. I sæsonen er der behov for at havnefogeden skal være tilgængelig de fleste timer af døgnet, og skal kunne møde på havnen i akuttilfælde. Havnefogeden siger at han bor \enquote{ti minutter til et kvarters kørsel [\ldots] det er en betingelse at de ikke vil have en fra Hjørring der har en time [kørsel] derind}. Han tilføjer \enquote{jeg kan godt sidde derhjemme en sommeraften klokken halv elleve, hvor strømmen er gået ude på nogle af broerne. Der er jeg jo nød til at køre herud [\ldots] der er en del af jobbet}.

Havnefogeden står for håndteringen af gæstesejlere. Dette betyder at han skal holde øje med til- og afrejse af gæstesejlere, samt kontrollere hvorvidt gæstesejlere har betalt for leje af vandplads, og hjælpe med diverse problemer. Han skal også kunne tage imod betaling fra gæster som ikke anvender billetmaskinen, som beskrives i \cref{sub:tek_betaling}. De travleste dage om sommeren kan der være op til 150 gæster \cite{int_hf}. Da Per Jensen blev adspurgt om det er svært at holde styr på så mange gæster, svarede han: \enquote{Det var jeg selv skeptisk over da jeg startede. Hvordan man holder styr på det [overblik i havnen].}

I forbindelse med gæstesejlere, bruger han et skilte-system beskrevet i \cref{sub:gaster_havnefogeden}. Med systemet kan gæsterne se om en given vandlejeplads er ledig.
