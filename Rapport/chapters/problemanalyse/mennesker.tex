% Kassereren

% sekretæren

% formanden & næstformanden

% fartøjsinspektøren

% aktivitetslederen

% medlemmer - diversiteten

% gæster

% havnefogeden

%!TEX root = ../../Master.tex
\section{Mennesker}

\subsection{Medlemmer}

Den vigtigste interessent for en forening er dens medlemmer. Enhver handling udført af en forening, bør i sidste ende betyde en fordel for foreningens medlemmer. Det er derfor vigtigt at skabe sig et godt indtryk af foreningens medlemmer, hvis man ønsker at lave en løsning til en given forening.


\subsubsection{Medlemskaber}

Når det kommer til Vestre Baadelaug og Sejlklubben Limfjordens medlemmer, er der tale om et bredt segment. Generelt for alle medskaber af Vestre Baadelaug, gælder det, at der kun kan lejes pladser ud til én båd per medlemskab. Hvis båden har flere ejerer skal alle ejerene have et medlemsskab. Dette er dog noget som Vestre Baadelaug ikke kontrollerer, da de ikke ønsker at kræve købskontrakter af medlemmerne. Det bør dog nævnes at et medlemsskab i Vestre Baadelaug dækker ejerens husstand, således at børn og ægtefælle er fritaget fra at have deres eget medlemsskab. Sejlklubben Limfjorden kræver ét medlemsskab per næse, også selvom de er fra samme husstand.

Vestre Baadelaug vil som grundregel undgå passive medlemmer, det værende medlemmer uden båd, hvor Sejlklubben Limfjorden kræver et passivt medlemsskab hvis man ikke er ejer af båden man sejler.

Set i praktiske forhold, så kan nogle af disse regler midlertidigt ignoreres, hvis der er formildende omstændigheder. Dette kunne for eksempel være i tilfælde af køb og salg af båd, hvor, i en periode, ét medlem kan have flere både liggende i havnen. Hvis sådan en situation strækker sig for langt, vil klubberne kræve et nyt medlemsskab.


\subsubsection{Medlemmernes demografi}

Her følger en beskrivelse af de to klubbers medlemmer, fra et demografisk synspunkt.

Vestre Baadelaug har en bred demografi. Dette skyldes i stor grad, få restriktioner på både der tillades i klubben. Sejlklubben Limfjorden tillader kun sejlbåde, og har af denne grund en smallere demografi. Ifølge Peter Hinrup og Dorte Carøe, som repræsenterer henholdsvis Vestre Baadelaug og Sejlklubben Limfjorden, er dette den direkte grund til at Vestre Baadelaug har en større udskiftning af medlemmer, og at der oftere opstår konflikter imellem medlemmerne i Vestre Baadelaug.

\kanote{jeg ved ikke helt om jeg kan tillade mig at lave denne konklusion}

visse ting er dog fælles for begge klubber. Aldersmæssigt er der fra unge medlemmer i tyverne, op til pensionister og efterlønnere. Der er enker og børnefamilier. Der er både registeret til firmaer, og ifølge Peter Hinrup er det også et miljø, som tiltrækker folk der \enquote{ikke er ædruelige døgnet rundt} og \enquote{folk som er relativt tørstige}.

\kanote{overvej tilføjelser og lav en opdeling af klubber (aldersgruppen er forskellige!!!)}

\subsubsection{Det Sociale Aspekt}

For mange af medlemmerne, repræsenterer deres båd den største investering de har foretaget i deres liv. Mange af medlemmerne deltager også aktivt I de arrangementer der bliver organiseret i havnen, og har erhvervet store dele af deres sociale omgangskreds igennem Vestre Baadelaug. En begrænset mængde af medlemmerne har bopæl i havnen, og bor året rundt i deres båd. Mange andre bruger båden som et kolonihavehus, hvor de bruger deres weekender i sommerhalvåret. Alt dette er aspekter som klubben støtter op om, ud fra et ønske om, at der altid er liv i havnen.



\kanote{et afsnit omkring de sociale ting. Forskellige måder at bruge sin båd på (kolonihavehus etc.). Betydningen af hvilke plads man har. Medlemmernes prioriteter. Sociale arrangementer. Fastboende og klubbens holdning. Ønsket om liv i havnen}


\subsection{Gæstesejlere}

Den anden største indtægtskilde for Vestre Baadelaug efter vandlejepladser, er gæstesejlere, som betaler leje når de anvender Vestre Baadelaugs faciliteter. Dette udgør 14 procent af de samlede indtægter for Vestre Baadelaug, hvilket derfor af samme grund repræsentere et væsentligt interesse område for Vestre Baadelaug.

Gæstesejlerene er en meget bred demografi, med folk fra mange forskellige lande, primært nævnes Danmark, Norge, Sverige, Tyskland, Holland og Polen. 

\kanote{afsnittet skal indeholde: Demografi omkring gæster, (nationalitet bl.a.). m.m. (?)}

\subsection{Havnefogeden}

\kanote{mangler tanker}

