%!TEX root = ../../Master.tex
\section{Laudon \& Laudon Modellen}
Denne rapport er opbygget ud fra Laudon \& Laudon modellen~\cite{laudon}. Denne model er designet til at analysere et informationssystem, med henblik på at lave et produkt. Et informationssystem er beskrevet i~\cite{laudon2006management}: \enquote{Et informationssystem kan teknisk defineres som en mængde af sammenhørende komponenter. Disse komponenter indsamler (eller henter), processerer, gemmer og distribuerer information til at hjælpe med beslutningstagen, koordination og kontrol i en organisation. Udover dette, kan informationssystemet også hjælpe bestyrere og ansatte til at analysere problemer, visualisere komplekse emner og kreere nye produkter.}

\begin{figure}
  \centering
  \includegraphics[width=\textwidth]{laudon}
	\caption{Oversigt over Laudon \& Laudon modellen}
	\label{fig:oversigt_laudon}
\end{figure}

I denne rapport vil informationssystemet blive omtalt som løsningen, og det der i modellen hedder løsning, er den effekt informationssystemet giver. Et overblik af modellen kan ses på \cref{fig:oversigt_laudon}. 

Modellen sørger altså for at løsningen er kontekstuel relevant i forhold til problemet. De forskellige elementer i modellen der skal undersøges, er mennesker, teknologi, organisation, problem, løsning og effekten. Modellen giver mulighed for at lave en løsning der tager udgangspunkt i virkeligheden, og derved løser de reelle krav der er til løsningen.

\subsection{Interessent analyse}

Som en del af problemanalysen vil der blive foretaget en interessent analyse. Denne analyse vil blive foretaget sideløbende med de kontekstuelle analyser i Laudon \& Laudon modellen, og vil beskrive de organisatoriske og menneskelige interessenter. Dette gøres fordi der er mange komplimenterende analyser, som med fordel kan integreres, således at læseren gives et samlet indtryk at kontekstuelle interessenter. 
