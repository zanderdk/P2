\section{Teknologi} % (fold)
\label{sec:Teknologi}

I dette afsnit bliver forskellige eksisterende teknologier bearbejdet med henblik på at give et overblik over disse forskellige teknologier. Hver teknologi vil blive introduceret kort, hvorefter styrker og svagheder vil blive præsenteret. Dette kan senere bruges ved løsningsdesign.


\subsection{Navision} % (fold)
\label{sub:Navision}

Navision eller Microsoft Dynamics NAV er et dansk software regnskabsprogram. Det blev oprindeligt udviklet af Jesper Balser, Torben Wind og Peter Bang tilbage i 1984. Siden da har det haft forskellige navne. I 2002 opkøbte Microsoft programmet, og integrede det i deres Microsoft Business Solutions program.

Programmet samler en virksomheds aktiver i en grafisk brugerflade, der gør det enkelt at manipulere data som f.eks. medlemmer, materialer og økonomi. Navision er designet til at kunne håndtere alle typer virksomheder, og kan skræddersyes efter behov ved tekniker besøg.


\sinote{overvej flere problemer, eller omskriv til ental}
Nogle af problemerne med Navision er ifølge Vestre Baadelaug, at man ikke kan lave en regning og sende denne via e-mail \cite{int_vb_sl}. Dette kan dog som tidligere nævnt udbredres ved betaling til Microsoft.

% subsection Navision (end)

\subsection{Chip-system} % (fold)
\label{sub:Chip}


chipsystem til faciliteter \& strøm samt `parkering'
% subsection Chip (end)

\subsection{Håndtering af gæster - havnefogeden} % (fold)
\label{sub:havnefogeden}

% subsection Håndtering af gæster - havnefogeden (end)

\subsection{MarinaBooking.dk} % (fold)
\label{sub:MarinaBooking.dk}

Formålet med MarinaBooking \cite{marinabooking} er at gøre vandpladsreservationer lettere for gæster til lystbådehavne i Danmark. Systemet fungerer ved at gæster indtaster ankomst- samt afgangsdata, hvilken havn de vil lægge til og deres båds dimensioner. MarinaBooking kan med disse informationer nu præsentere et oversigtskort over havnebroerne, hvor gæsten kan vælge den ønskede vandplads på kortet. Systemet opkræver betaling, og gæsten har nu reserveret denne vandplads.

MarinaBooking gør det enkelt for gæster fra Danmark såvel som fra udlandet, at bestille havepladser fra internettet. Dette er en bonus for gæsterne, men også for administrationen i havnen, da tildeling af pladser herved ordner sig selv.

Ulemperne ved MarinaBooking er, at havnen specifikt skal have en bro til gæster. \sinote(link til hvor vi skriver om VB's layout angående gæster/medlemsbro) Dette kan være en af grundene til at udvalget af lystbådehavne ikke er specielt stort.

% subsection MarinaBooking.dk (end)

\subsection{E-conomic} % (fold)
\label{sub:E-conomic}

E-conomic er både navnet på en virksomhed og deres online regnskabdsprogram. Formålet med programmet er at gøre økomnomi-håndtering let for selv ikke-regnskabsuddannede. E-conomic henvender sig primært til små- og mellemstore virksomheder.

Fordelene ved e-conomic er den intuitive brugergrænseflade som letter mange bogførings handlinger. Derudover er programmet skrevet som et web-program hvilket gør at det kan tilgås fra alle platforme samt fra mange lokationer.

% subsection E-conomic (end)

\subsection{ForeningLet} % (fold)
\label{sub:ForeningLet}

% subsection ForeningLet (end)

% section Teknologi (end)
