%!TEX root = ../Master.tex
\section{Interview Metode}

I forbindelse med kontekstanalyse for projektet, blev der foretaget et interview med kasseren i Vestre Baadelaug, Peter Hinrup, og sekretæren i Sejlklubben Limfjorden, Dorthe Carøe. Interviewet blev foretaget i Vestre Baadelaugs klubhus, den 13. Februar 2014, af Kasper Terndrup fra SW2-A405a og Martin Raunkjær Andersen fra SW2-A304, med Mikkel Madsen fra SW2-A405a som referent. Interviewet bestod af 3 dele. Den første del var en verbal fremlæggelse fra Peter Hinrup og Dorte Carøe, omkring klubbernes generelle funktionalitet og interesser. Efterfølgende foretog Peter Hinrup en demonstration af Vestre Baadelaugs IT-system, Navision, med kommentarer angående systemet i forhold til klubbens behov. Til slut blev der stillet en række forberedte spørgsmål, som skulle sikre at de alle ønskede emner blev dækket. Første og sidste del af interview blev optaget på lyd, og demonstrationen fra anden del, blev filmet.

Interviewet resulterede i et generelt indblik i miljøet, der eksisterer i Vestre Baadehavn og Skudehavnen, samt delvist miljøet blandt alle lystbådehavne i sammenarbejdet Aalborg-Nørresundby Fritidshavne (ANF). Dertil gav interviewet en god forståelse af Vestre Baadelaug som organisation med fokus på de administrative dele af bestyrelsen, samt en beskrivelse af demografien, som udgør medlemmerne.

Store dele af kontekstanalysen vil blive begrundet i udtalelser fra interviewet.

