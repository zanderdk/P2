%!TEX root = ../Master.tex

I forbindelse med kontekstanalyse for projektet, blev der foretaget et interview med kasseren i Vestre Baadelaug, Peter Hinrup, og sekretæren i Sejlklubben Limfjorden, Dorthe Carøe. Interviewet blev foretaget i Vestre Baadelaugs klubhus, den 13. Februar 2014, af Kasper Terndrup fra SW2-A405a og Martin Raunkjær Andersen fra SW2-A304, med Mikkel Madsen fra SW2-A405a som referent. Interviewet bestod af 3 dele. Den første del var en verbal fremlæggelse fra Peter Hinrup og Dorte Carøe omkring klubbernes generelle funktionalitet og interesser. Efterfølgende foretog Peter Hinrup en demonstration af Vestre Baadelaugs IT-system, Navision, med kommentarer angående systemet i forhold til klubbens behov. Til slut blev der stillet en række forberedte spørgsmål, som skulle sikre at de alle ønskede emner blev dækket. Første og sidste del af interviewet blev optaget på lyd, og demonstrationen i anden del, blev filmet.

Interviewet resulterede i et generelt indblik i miljøet, der eksisterer i Vestre Baadehavn og Skudehavnen, samt miljøet blandt alle lystbådehavne i sammenarbejdet Aalborg-Nørresundby Fritidshavne (ANF). Dertil gav interviewet en god forståelse af Vestre Baadelaug som organisation med fokus på de administrative dele af bestyrelsen, samt en beskrivelse af demografien, som udgør medlemmerne.

Store dele af kontekstanalysen vil blive begrundet i udtalelser fra interviewet. Interviewet gav os indblik i Vestre Baadelaugs nuværende IT-system. Systemet som kasseren brugte var velfungerende og dækkede deres behov. Udover han eget system udtalte han sig også om havnefogedens system. Havnefogeden havde ikke noget IT-system til håndtering af pladserne på havnen. Kun havnefogeden kendte systemet og uden hans tilstedeværelse var der ingen i Vestre Baadelaug som kunne finde rundt i hvor gæsterne ligger til. 

Interviewet med kasseren informerede os om at havnefogeden umiddelbart har administrative problemer. På baggrund af kasserens udtalelser om havnefogeden valgte vi efterfølgende at spørge havnefogeden om han ville stille op til et interview.
Interview med havnefogeden forløb på samme måde som 3. del af interviewet med Peter Hinrup. Interview blev udført af 3 medlemmer fra gruppen SW2-A405a, med Simon Vandel. Helte interview blev optaget på lyd. Alle interviews er transskriberet og ved lagt på cd'en. 

