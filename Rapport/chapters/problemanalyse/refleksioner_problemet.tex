%!TEX root = ../Master.tex
\chapter{Refleksioner om Problem}

Under det indledende interviewet med Peter Hinrup go Dorthe Carøe, afviste alle de interviewede parter, at der var behov for forbedring af systemet hvorved gæster blev håndteret på. Der var dog en usikkerhed omkring hvordan systemet fungerede, hvilket gav anledning til at undersøge hvorvidt der kunne foretages forbedringer af systemet, som virkede til at være improviseret. Et opfølgende interview blev foretaget med havnefogeden for de to havne, Vestrebådehavn og Skudehavnen, for at klargøre hvorvidt systemet gav anledning til forbedring.

Udmeldingen fra havnefogeden var klar. Han havde et system som fungerede godt, og var velintegreret med andre opgaver havnefogeden udførte. Det var klart at se, at det tilstedeværende system fungerede godt, og at en direkte forbedring af systemet ville være problematisk. 

\kanote{uddyb hvorfor systemet var så godt}

De enkelte steder hvor det tilstedeværende system gav anledning til problemer var få, og de problemer der var tale om var letløselige. Visse af disse problemer kunne også tilskrives andre årsager, som falder udenfor ethvert tænkeligt systems forebyggelsesevne. Eksempler kunne være fysisk mangel på plads, hvilket enten kunne klares ved at nye både ligger til på ydersiden af langsidefortøjede både. Det kunne også være menneskelige fejl, som er sket grundet misforståelser. Disse er dog også ofte nemt løst af havnefogeden.

Havnefogeden opererer også som et serviceorgan på havnen, som hjælper til med spørgsmål fra både klubbernes medlemmer, samt gæster. Han er bestyrelsens repræsentative autoritet, og har ansvaret for at løse de dagligt opstående problemer, samt general vedligeholdelse. Disse ting gør det nødvendigt for havnefogeden at være tilstede og synlig på havnen. Dette mindsker genvindsten ved at automatisere håndteringen af gæsterne, da der ikke kan foretages væsentlige besparelser af havnefogedens tid.