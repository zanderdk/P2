%!TEX root = ../Master.tex
\section{Refleksioner om Problem}

Vestre Baadelaug er på de fleste områder langt fremme, med hensyn til at anvende ny IT til at løse deres problemer. De anvender dog stadig et manuelt system, når det kommer til registrering af gæstesejlere og et helt andet system til registrering af medlemmer. Det ville være oplagt at forsøge at inkorporere disse aspekter af havneadministrationen i helhedsindtrykket, eftersom klubberne selv har udtrykt et ønske om ét fælles system \cite{int_vb_sl}. Men en digitalisering af det eksisterende system er ikke trivielt.

Under det indledende interview med Peter Hinrup og Dorthe Carøe, udtrykte alle de interviewede parter, at de ikke kunne se hvordan man skulle forbedre systemet til håndtering af gæster. Der var dog en usikkerhed omkring hvordan systemet fungerede, og det blev anerkendt at en ny havnefogeden vil have problemer med at lære systemet, eftersom havnefogeden, ifølge Peter Hinrup, \enquote{kører [systemet] på bagsiden af en kuvert}. Dette gav anledning til at undersøge hvorvidt der kunne foretages forbedringer af systemet, som virkede til at være improviseret. Et opfølgende interview blev foretaget med havnefogeden for de to havne, Vestrebådehavn og Skudehavnen, for at klargøre hvorvidt systemet gav anledning til forbedring.

Udmeldingen fra havnefogeden var klar. Han har et system som fungerer godt for ham, og det er velintegreret med andre opgaver havnefogeden udfører. Det er åbenlyst at det tilstedeværende system fungerer godt, og at en direkte forbedring af systemet er problematisk \cite{int_hf}. 

En beskrivelse af Havnefogedens funktion kan findes i \cref{sub:havnefoged}.

De enkelte steder hvor det tilstedeværende system giver anledning til problemer er få, og de problemer der er tale om er letløselige. Visse af disse problemer kan også tilskrives andre årsager, som falder udenfor ethvert tænkeligt systems forebyggelsesevne. Eksempler kunne være fysisk mangel på plads, hvilket enten kunne klares ved at nye både ligger til på ydersiden af langsidefortøjede både. Det kunne også være menneskelige fejl, som er sket grundet misforståelser. Disse bliver dog også ofte nemt løst af havnefogeden \cite{int_hf}. Havnefogeden opererer også som et serviceorgan på havnen, som hjælper til med spørgsmål fra både klubbernes medlemmer og gæster. Han er bestyrelsens repræsentative autoritet, og har ansvaret for at løse de dagligt opstående problemer, samt general vedligeholdelse. 

Disse ting gør det nødvendigt for havnefogeden at være tilstede og synlig på havnen. Dette mindsker gevinsten ved at automatisere håndteringen af gæsterne, da der ikke kan foretages væsentlige besparelser af havnefogedens tid; selvom et nyt system kan automatisere håndteringen af gæsterne, skal havnefogeden alligevel være tilstede på havneområdet. Systemet med de røde og grønne skilte, som anvendes til indikation af pladsernes status, er et system som kendes af mange lystbådesejlere. Et eventuelt alternativt system, som ikke anvender lingende signalering, vil risikere at skabe forvirring hos gæstesejlere.

På trods af, at det nuværende system er velfungerende, og at alle adspurgte parter mener at der ikke er behov for et andet system, er der stadig steder hvor der kan argumenteres for ændringer.

Et digitalt bådregistreringssystem vil passe bedre ind i den samlede administration, og vil gøre det nemmere at forene de forskellige administrationssystemer. Et IT-system vil også give mulighed for mere data tilgængelighed. I nuværende situation er det kun havnefogeden som har adgang til et overblik over havnen, og pladsernes statushistorik og fremtidige planer. Med et digitalt system, vil der være grundlag for at klubbernes medlemmer kunne have direkte adgang til data vedrørende deres egen plads, og undgå at skulle kontakte havnefogeden. Der kunne også oprettes andre adgangsniveauer, som ville kunne have adgang til forskellig data, og have forskellige redigeringsrettigheder. Her tænkes på muligheder for bestyrelsen, og deres evne til at se tendenser i havnen, og planlægge derefter. Det ville også være muligt at give gæstesejlere adgang til data. Hvis en gæstesejler har ønske om at ligge i havnen i et antal dage, er det fordelagtigt at ligge sig på en plads, hvor pladsejeren ikke har meldt ankomst, inden gæsten har planlagt afrejse. Der kan eventuelt også foretages reservationer af pladser, såfremt klubben ønsker at dette blive muligt.

Adgang til informationer vil mindske behovet for havnefogedens synlighed på havnen, da han ikke i samme grad vil kunne blive en flaskehals i systemet.

Der er dog problemstillinger ved et sådant IT-system, siden det ikke vides hvorvidt medlemmerne og gæster vil udnytte de funktioner der gives, og hvorvidt havnefogeden vil kunne bibeholde overblikket over havnens status, hvis medlemmer og gæster kan ændre information i systemet, uden at skulle gå igennem ham. En risiko ved IT-systemet som en helhed, er at havnefogeden ender med at bruge meget tid på at holde styr på systemet, og at medlemmerne ikke føler en forbedring af havnemiljøet. Der vil stadig være mulighed for at gæsterne vil få glæde af systemet, hvilket kan resultere i en øget tilbagevending. Det skal dog medtages, at klubberne ikke er grundlagt med profit for øje, og at gæsteleje er en sekundær opgave for havnen.