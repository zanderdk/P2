%!TEX root = ../Master.tex
\section{Refleksioner om Problem}
Vestre Baadelaug er på de fleste områder langt fremme, med hensyn til at anvende ny IT til at løse deres problemer. De anvender dog stadig et manuelt system, når det kommer til registrering af gæstesejlere og et helt andet system til registrering af medlemmer. Det ville være oplagt at forsøge at inkorporere disse aspekter af havneadministrationen i helhedsindtrykket, eftersom klubberne selv har udtrykt et ønske om ét fælles system \cite{int_vb_sl}. Men en digitalisering af det eksisterende system er ikke trivielt.

Under det indledende interview med Peter Hinrup og Dorthe Carøe, udtrykte alle de interviewede parter, at de ikke kunne se hvordan man skulle forbedre systemet til håndtering af gæster. Der var dog en usikkerhed omkring hvordan systemet fungerede, og det blev anerkendt at en ny havnefogeden vil have problemer med at lære systemet, eftersom havnefogeden, ifølge Peter Hinrup, \enquote{kører [systemet] på bagsiden af en kuvert}. Dette gav anledning til at undersøge hvorvidt der kunne foretages forbedringer af systemet, som virkede til at være improviseret. Et opfølgende interview blev foretaget med havnefogeden for de to havne, Vestrebådehavn og Skudehavnen, for at klargøre hvorvidt systemet gav anledning til forbedring.

Udmeldingen fra havnefogeden var klar. Han har et system som fungerer godt for ham, og det er velintegreret med andre opgaver havnefogeden udfører. Det er åbenlyst at det tilstedeværende system fungerer godt, og at en direkte forbedring af systemet er problematisk \cite{int_hf}. 

\kanote{uddyb hvorfor systemet er så godt - fordele og ulemper}

De enkelte steder hvor det tilstedeværende system giver anledning til problemer er få, og de problemer der er tale om er letløselige. Visse af disse problemer kan også tilskrives andre årsager, som falder udenfor ethvert tænkeligt systems forebyggelsesevne. Eksempler kunne være fysisk mangel på plads, hvilket enten kunne klares ved at nye både ligger til på ydersiden af langsidefortøjede både. Det kunne også være menneskelige fejl, som er sket grundet misforståelser. Disse bliver dog også ofte nemt løst af havnefogeden \cite{int_hf}.

Havnefogeden opererer også som et serviceorgan på havnen, som hjælper til med spørgsmål fra både klubbernes medlemmer og gæster. Han er bestyrelsens repræsentative autoritet, og har ansvaret for at løse de dagligt opstående problemer, samt general vedligeholdelse. Disse ting gør det nødvendigt for havnefogeden at være tilstede og synlig på havnen. Dette mindsker gevinsten ved at automatisere håndteringen af gæsterne, da der ikke kan foretages væsentlige besparelser af havnefogedens tid; selvom et nyt system kan automatisere håndteringen af gæsterne, skal havnefogeden alligevel være tilstede på havneområdet.


På trods af, at det nuværende system er velfungerende, og at alle adspurgte parter mener at der ikke er behov for et andet system, er der stadig steder hvor der kan argumenteres for ændringer.

\kanote{grunde til at det kan betale sig at lave et system}

Disse aspekter, for og imod, har været baggrund for en masse overvejelser, om hvorvidt det er relevant at lave en IT løsning eller ej. Et system vil passe bedre ind i den samlede administration, og vil gøre det nemmere at forene systemerne. Dertil vil et IT-system være mere tilgængeligt for andre parter end administrationen, hvilket kunne gøre det muligt for klubbernes medlemmer, direkte at tilgå statusmeldingen på deres plads, uden at skulle henvende sig til havnefogeden. Dette vil kunne spare havnefogedens tid, og gøre det muligt for medlemmerne at melde ændringer i planen 

Der er dog problemstillinger ved et sådant IT-system, siden det ikke vides hvorvidt medlemmerne vil udnytte en sådan funktion, og hvorvidt havnefogeden vil kunne bibeholde overblikket over havnens status, hvis medlemmer og gæster kan ændre information i systemet. Den øgede tilgængelighed vil også gøre det muligt at vise informationer om havnen til gæster, uden at de nødvendigvis skal have fat i havnefogeden. Dette kunne resultere i mere frihed til havnefogeden, så han ikke var nær så bundet af at skulle være synlig på havnen. Derudover ville dette også åbne for muligheden for pladsreservationer på internettet.

En risiko ved IT-systemet som en helhed, er at havnefogeden kunne ende med at bruge mere tid på at holde styr på systemet, end hvad der kan betale sig.

