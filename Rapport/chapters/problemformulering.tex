\chapter{Problemformulering}
\label{cha:problemformulering}


Problemanalysen dækker de forskellige aspekter af pladsadministrering for en bådklub. Analysen består af en interessentanalyse, som skaber en forståelse af hvilke behov brugerne af systemet har. Derudover består analysen af en vurdering af den nuværende teknologi som indgår i det brugte system. På baggrund af denne analyse, lavet ud fra det initierende problem \ref{initierende}, er problemet blevet fokuseret i form af en problemformulering.


\begin{displayquote}
	\textbf{\textit{Hvordan kan man modellere en bådklubs ressourcer, og implementere et software program der automatiserer administrationen af bådpladser?}}
\end{displayquote}

\section{Kravspecifikation} % (fold)
\label{sec:Kravspecifikation}

Ud fra problemformuleringen er der opstillet en kravspecifikation, som har til
formål at udspecificere de krav til løsningen, som er underbygget i problemanalysen. 
Kravspecifikation er opstillet i punktform med en kort beskrivelse 
under punktet. Beskrivelsen vil desuden indeholde en reference til problemanalysen, hvor 
behovet for dette krav er dokumenteret.
 

\begin{itemize}
  \item Systemet skal registrere afrejse og ankomst af gæster såvel som medlemmer.
  \item Systemet skal gemme hvilke pladser der er ejet, hvilken båd på pladsen, og af hvem?
  \item Systemet skal have mulighed for at informere havnefogeden i tilfælde af fejl eller begivenheder.
  \item Systemet skal kunne fremvise information omkring pladser, medlemmer, samt gæster.
  \item Systemet skal kunne søge igennem programdatabaser på forskellige felter.	
  \item Systemet skal kunne interagere med bådklubbens nuværende betalingssystem.
\end{itemize}


% section Kravspecifikation (end)
