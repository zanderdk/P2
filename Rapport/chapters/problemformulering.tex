\chapter{Problemformulering}
\label{cha:problemformulering}


Problemanalysen har dækket de forskellige aspekter af pladsadministrering for en bådklub.
Dette har skabt en forståelse af hvilke behov som de brugerne af systemet har.
Ud fra denne forståelse er en interessentanalyse blevet udarbejdet, og teknologierne
som indgår i systemet blevet analyseret. På baggrund af denne analyse lavet ud fra det initierende problem\ref{initierende}, 
er problemet da blev afgrænset i form af en problemformulering.

\begin{displayquote}
	\textbf{\textit{Hvordan kan man modellere en bådklubs ressourcer, og implementere et software program der automatisk administrerer bådpladser?}}
\end{displayquote}

\section{Kravspecifikation} % (fold)
\label{sec:Kravspecifikation}

På baggrund af vores problemanelyse er der opstillet en Kravspecifikation som har til
formål at udspecificere de krav som er opstillet på baggrund af de behov som interessenterne har. 
Kravspecifikation vil være opstillet i punktform med en kort beskrivelse af kravet stående
under punktet. Beskrivelsen vil desuden inder holde en reference til hvor i teksten at
behovet for dette krav er dokumenteret.
 

\begin{itemize}
  \item Systemet skal registrere afrejse og ankomst af gæster såvel som medlemmer.
  \item Systemet skal gemme hvilke pladser der er ejet, hvilken båd på pladsen, og af hvem?
  \item Systemet skal have mulighed for at informere havnefogeden i tilfælde af fejl eller begivenheder.
  \item Systemet skal kunne fremvise information omkring pladser, medlemmer, samt gæster.
  \item Systemet skal kunne søge igennem databaser på forskellige felter.	
  \item Systemet skal kunne interagere med bådklubbens nuværende betalingssystem.
\end{itemize}


% section Kravspecifikation (end)
