\chapter{Problemformulering}
\label{cha:problemformulering}

I \cref{sec:refleksioner} blev problemerne ved Vestre Baadelaugs nuværende system udspecificeret, og konkretiseret. Her bliver det beskrevet, at havnefogeden har overbliksproblemer i højsæsonen med havnen og de ankommende både. Endvidere har det nuværende system ikke mulighed for at modtage betaling for mere end en dag af gangen, som beskrevet i \cref{sub:gaster_havnefogeden}. 

Vestre Baadelaug fremstår i problemanalysen til at være en bådklub der imødekommer teknologi, og de føler selv at deres system er effektivt. Det vurderes dog på baggrund af samme analyse, at der kan foretages nogle fordelagtige ændringer fra manuelle systemer til elektroniske systemer, hvorved der på sigt ville kunne opnås en bedre, og mere forenet administration af klubben og havnen.

Disse ændringer er ikke tænkt som en opgradering af et allerede eksisterende system, men et forsøg på at finde fordele i at se problemet fra en anden vinkel, end hvad det nuværende system gør.

Det er på baggrund af disse tanker, med rødder i analysen, lavet ud fra det initierende problem i \cref{initierende}, at problemet blevet fokuseret i form af en problemformulering. 

\begin{displayquote}
\textit{Hvordan kan man modellere en bådklubs administrative ansvarsområder indenfor både, bådpladser, medlemmer og gæster, samt implementere et software program der kan benytte modellerne til formidling af informationer mellem havnefoged, medlemmer og gæster, med det formål at forbedre overblikket over havnen?}
\end{displayquote}

\input{chapters/Kravspecifikation.tex}
