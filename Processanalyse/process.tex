\documentclass[a4paper,12pt,oneside,article]{memoir}

%%%% PACKAGES %%%%

\usepackage[T1]{fontenc}
\usepackage[utf8]{inputenc}
\usepackage{fourier}
\usepackage[danish]{babel}
\usepackage[danish]{cleveref}
\usepackage{graphicx}
\graphicspath{{Images/}}

%   ¤¤ Afsnitsformatering ¤¤ %
\setlength{\parindent}{0mm}             % Størrelse af indryk
\setlength{\parskip}{3mm}                   % Afstand mellem afsnit ved brug af double Enter
\linespread{1,2}                                                % Linie afstand
% length of abstract indentations
\setlength{\absleftindent}{0mm}
\setlength{\absrightindent}{0mm} 

\begin{document}

\chapter{Procesanalyse}

    \section{Beskrivelse - Hvad gjorde vi i P1} 

        \subsection{Projektplanlægning}

            %I hvilket omfang har gruppens medlemmer haft samme opfattelse af hvad projektplanlægning indebærer?

            % Har I haft nogle projektplaner? I så fald: Hvilke, hvad har I anvendt dem til og hvordan har de fungeret? Hvis ikke: Vil I lave projektplaner i P1? I så fald: Hvilke og hvad vil I bruge dem til?

        \subsection{Problemformuleringer}


        

\subsection{Rapportstrukturering}



\section{Vurdering - Hvordan gik det}


\subsection{Problemformuleringer}


    \section{Analyse - Hvorfor gik det som det gik}
    % Dernæst skal I analysere jeres arbejdsprocesser og få klarlagt hvorfor noget gik godt mens andet gik dårligt. Med andre ord: Hvad er det for faktorer, som har indvirket på arbejdsprocesserne? 


        \subsection{Problemformuleringer}

        \subsection{Tegninger}

        \subsection{Rapportstrukturering}

        \subsection{Andet}

    \section{Syntese - Gode råd til P2}


    \subsection{Problemformuleringer}


    \subsection{Samarbejde med vejlederen}


    % Hvis jeres vurdering og analyse skal bidrage til at forbedre jeres evne til at håndtere det 
    % problemorienterede og projektorganiserede gruppearbejde, skal I til slut konkretisere jeres 
    % erfaringer i nogle ’Gode råd’ til jer selv og jeres medstuderende. En god måde at formulere sådanne 
    % gode råd på er som en *start-stop-fortsæt*-liste, dvs. en liste med følgende tre sektioner: 
    % Dette vil vi begynde at gøre i P1, som vi ikke gjorde i P0 
    % Dette vil vi ikke gøre i P1, som vi gjorde i P0 
    % Dette vil vi fortsætte med at gøre (gerne anderledes og bedre) i P1, som vi også gjorde i P0 

\end{document}
