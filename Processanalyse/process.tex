\documentclass[a4paper,12pt,oneside]{article}

%%%% PACKAGES %%%%

\usepackage[T1]{fontenc}
\usepackage[utf8]{inputenc}
\usepackage{fourier}
\usepackage[danish]{babel}
\usepackage[danish]{cleveref}
\usepackage{graphicx}
\graphicspath{{Images/}}


\title{Procesanalyse}
\author{SW2-A405a}
\date{21. maj 2014}

\begin{document}
\maketitle


%\section{Indledning}
%Denne note er ment som en kortfattet vejledning i og hjælp til at skrive en P0-procesanalyse på basisåret på Aalborg Universitet. Målgruppen er de studerende på Basisåret, først og fremmest i faggrupperne Datalogi, Software, Bachelor i IT og Informatik. 

\section{Beskrivelse - Hvad gjorde vi i P2} 
% I skal beskrive jeres arbejdsprocesser i P0 så detaljeret som muligt

\subsection{Projektplanlægning}
%I hvilket omfang har gruppens medlemmer haft samme opfattelse af hvad projektplanlægning indebærer? 
%Har I haft nogle projektplaner? I så fald: Hvilke, hvad har I anvendt dem til og hvordan har de fungeret? Hvis ikke: Vil I lave projektplaner i P1? I så fald: Hvilke og hvad vil I bruge dem til? 
%Har I haft en projektleder i gruppen? Hvorfor/Hvorfor ikke? Fordele og ulemper. 
%Hvordan har projektopgaver som f.eks. tidsplanlægning, ressourcefordeling, uddelegering af arbejdsopgaver, kontakt til vejleder, kontakt udadtil mm. været varetaget i jeres gruppe? 

\subsection{Problemformuleringer}
%Hvordan lavede I indledning til problemformuleringen?
%Hvilken argumentation brugte I for at motivere problemformuleringen?
%Hvad har I lært om hvordan problemformuleringer udarbejdes?

\subsection{Rapportstrukturering}
%Hvordan bestemte I hvilke afsnit der skulle være i jeres P0 rapport?
%Hvordan bestemte I strukturen i rapporten?
%Hvordan sikrede I en rød tråd i rapporten?
%Hvilke afsnit havde I i jeres rapport, som I mener er generelle for alle projekter?

\subsection{Gruppesamarbejde}
%Har I talt om jeres forventninger til hinanden? (F.eks. om hvad der motiverer jer, jeres ambitionsniveau, social samvær etc.) 
%Har I lavet en samarbejdsaftale (mundtlig/skriftlig)? Hvorfor/Hvorfor ikke? 
%Hvor tit har I holdt møder?  
%Hvad gjorde I hvis en person kom for sent/ikke kom til møderne? 
%Hvordan har I afviklet møderne? (F.eks. med mødeleder, via runder om bordet, fri diskussion etc.) 
%Hvordan har kommunikationen været i jeres gruppe? Var der nogen, der talte hele tiden? Var der nogen, der aldrig sagde noget? Brugte gruppen uforholdsvis lang tid på diskussioner? Hvorfor? 
%Hvordan har motivationen været hos de enkelte gruppemedlemmer? Har I oplevet problemer med forskelle i motivation? I så fald: Hvad har I gjort for at løse problemerne? 
%Efter hvilke kriterier har I fordelt arbejdsopgaverne mellem jer? Har det fungeret tilfredsstillende for alle? 
%Hvordan sikrede I konstruktiv kritik af hinandens arbejdsblade til rapporten? 

\subsection{Samarbejde med vejleder}
%Har I haft en samarbejdsaftale med jeres vejleder? I så fald: har den fungeret tilfredsstillende? 
%Hvordan forberedte I møder med jeres vejleder? 
%Hvilken type vejledning ønskede I fra vejlederen? Hvilken type vejledning fik I?  


\section{Vurdering - Hvordan gik det}
% Når I er færdige med at beskrive hvad I gjorde, skal I vurdere hvordan det gik. Med andre ord: Hvad gik godt i P1? Hvad gik dårligt i P1? 

\subsection{Projektplanlægning}

\subsection{Problemformuleringer}

\subsection{Rapportstrukturering}

\subsection{Gruppesamarbejde} 

\subsection{Samarbejde med vejleder}



\section{Analyse - Hvorfor gik det som det gik}
% Dernæst skal I analysere jeres arbejdsprocesser og få klarlagt hvorfor noget gik godt mens andet gik dårligt. Med andre ord: Hvad er det for faktorer, som har indvirket på arbejdsprocesserne? 

\subsection{Projektplanlægning}

\subsection{Problemformuleringer}

\subsection{Tegninger}

\subsection{Rapportstrukturering}

\subsection{Andet}

\subsection{Rapportstrukturering}

\subsection{Gruppesamarbejde} 

\subsection{Samarbejde med vejleder}




\section{Syntese - Gode råd til P3}
%Hvis jeres vurdering og analyse skal bidrage til at forbedre jeres evne til at håndtere det 
%problemorienterede og projektorganiserede gruppearbejde, skal I til slut konkretisere jeres 
%erfaringer i nogle ’Gode råd’ til jer selv og jeres medstuderende. En god måde at formulere sådanne 
%gode råd på er som en *start-stop-fortsæt*-liste, dvs. en liste med følgende tre sektioner: 
%Dette vil vi begynde at gøre i P1, som vi ikke gjorde i P0 
%Dette vil vi ikke gøre i P1, som vi gjorde i P0 
%Dette vil vi fortsætte med at gøre (gerne anderledes og bedre) i P1, som vi også gjorde i P0 
 
\subsection{Problemformuleringer}

\subsection{Rapportstrukturering}

\subsection{Projektplanlægning}

\subsection{Rapportstrukturering}

\subsection{Gruppesamarbejde} 

\subsection{Samarbejde med vejleder}

\end{document}