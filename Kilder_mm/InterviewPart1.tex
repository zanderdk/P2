\documentclass[12pt,a4paper]{article}
\usepackage[UTF8]{inputenc}
\usepackage[danish]{babel}
\usepackage{amsmath}
\usepackage{amsfonts}
\usepackage{amssymb}
\begin{document}
P- Peter Hinrup
D- Dorthe
K- Kasper
M- Mikkel
M2- Martin

P(00.00): der er 4 havne i aalborg hvor kommunen ejer havnen, hvor klubberne lejer havnen gratis imod at de selv vedligeholder den. Det er typisk en bådklub pr havn, ud over det bliver havnene nogle gange benyttet af andre mindre klubber så som kajakhavne eller søspejdere.

P(02.19): Der er 4 store havne i Aalborg, den her, Skudehavn, Marinafjordparken (aalborg sejlklub) og Nørresundby. De er ejet af Aalborg kommune imod at klubberne selv står for vedligeholdesen.

P(03.35): Vi er en forening med medlemmer som er medlemmer fordi de ejer en båd, hvorimod i Nibe kan man leje en plads, trods man ikke er medlem. I VB kan man ikke have en bådplads unden at være medlem.

P+D(05.00) viser gruppen en brochure med et kort over havnen.

P(06.25) Til sammen har VB og SL omkring 450 bådpladser, og i VB har de omkring 380 medlemmer og SL har omkring 150 medlemmer. I VB er det et familie medlemsskab, hvilket vil sige 1 båd = 1 medlem. 

D(07.31) Vores medlemskaber (SL) 	er pr hovede samt aktive og passive medlemmer.

P(08.00) Et medlem kan kun have en båd, dog i praksis ses det igennem fingrene, fordi det henner at et medlem køber en båd, også skal de have den gamle solgt. Det løses ved at et andet familiemedlem køber et edlemsskab

P(09.20) Dog sker det modsatte hvor der er flere ejere på en båd. Det er ofte unge mennesker der slår sig sammen omkring båden, men de skal alle være medlemmer, hvilket der ikke bliver tjekket op på.

D(11.40) SL og VB deler vandpladsen og betaler hver for sig leje til ANF. SL's medlemmer afregner pr kvadratmeter båd, hvorimod i VB, hvor de afregner pr pladskvadratmeter. Alle både beholder eller får nye pladser til foråret. Men man kan ikke købe en permanetplads, dog er der ballade hvis de flytter en båd der har ligget det samme sted i 20 år (Joke).

P(13.55) Pladser bliver også tildelt udfra forholdene på pladsen, f.eks. hvor dybt der er, men primært længde*bredde. Det er svært at få til at gå op i en højere enhed siden der er så mange inputs og interessanter man skal tage højde for.

P(14.49) hvert forår og januar måned går VB ud til alle medlemmer hvor de sender den information de har omkring deres medlemmer til dem, for at få opdateret den information der står i systemet. Derudover skal de også svare på deres ønsker til bådplads til sommer.	

D(16.00) Vi har kun sejlbåde i forhold til VB som har alle slags både

M2(16.35) "Hvad gør i med bådene fra medlemmer der er blevet smidt ud?"
P: Vi meddeler dem at de skal fjerne deres båd, og inden for en rimelig tidsrammer vil båden blive flyttet eller solgt

P(19.16) medlemmerne har numre, ældst har lavest nummer og vice versa. Dem med det mindste nummer har første ret på alt og det er besverligt for administrationen

P(20.00) Peter viser os regnearket med alt informationen "Ønskelisten"

P(22.00) VB er størrere end SL, men har også størrere fluktuering i blandt deres medlemmer. Dog afhængig af hvornår de startede nummersystemet

P(23.26) Peter viser os pladserne på "Ønskelisten" og forklare nummer systemet. Nummer ex. XXYY, XX= bro nummer YY plads på broen

P(24.40) Efter allokeringen af pladser, udsendes der regnigner hvor nogle medler fra og dernest sendes der rykkere hvor flere melder fra.

P(27.40) Der er en masse personlige, menneskelige og interesser der ikke fremgår af listen når der skal planlægges. Det er vigtigt for nogle mennesker hvor de ligger, det er afgørende.

P(31.56) VB har forskellige pladser, herunder Pælepladser, som er en fast bro, med en pæl man kan binde sin båd fast i både pælen og broen. Det er dyrt at fjerne pælepladser på grund af omkostninger. Anholds bruger Bøjer.

P(35.10) Peter viser os en Flydebro og den er praktisk fordi den aldrig oversvømmer, og pladserne er pratiske især hvis de er Y-bomme hvor man kan ændre på pladsens størrelse.

\end{document}