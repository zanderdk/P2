

P: Men det så selvfølgelig umiddelbart smartere ud end dette her. Fordi det er moderne. F.eks. kunne man få det grafisk op. Dine både altså, Man tegnede simpelthen sin havn ind på skærmen. Og hvor vi kan få bådpladserne op her, så kan vi så se hvad for nogle der er ledige, og hvad for nogle som er optaget, og hvem der har en plads. og det er meget meget fint, og det var grafisk. Man kan selv tegne broerne, og så er der alle pladserne, hvor de røde er optaget og de grønne er ledige. Og så kunne man så flytte rundt med en kurser, und so weiter.

D: Det ligner så det de lavede til Marine booking?

P: Jo, du kan så køre booking oveni.

D: Ja, for du kan vidst booke en plads online i nogle havne, og så skal de jo vide hvor de sender båden hen, når de kommer ind af havnemunden(?).

P: Ja, det er jo en anden ting som vi ikke har snakket om. Det er jo gæster. For en ting er at man har en bådplads, du har en hjemmehavn. Det adskiller sig jo meget fra f.eks. camping verdenen. Lang de fleste har en campingvogn derhjemme, og så kører de rundt hist og her. Vi har jo i meget højere grad et hjemsted, for du kan ikke have en 40 fods båd liggende hjemme i din indkørsel. Det er vildt besværligt, fordi sådan en vejer jo 12 tons. Så du har en havn, hvor du skal ligge, og hvis du skal ligge i en havn, så skal du være medlem af en klub. Det kan så være meget udemærket, fordi for mange er det en vigtigt del af det sociale liv, nogle de har ikke andet en den her klub, havde jeg nært sagt, end den her klub. Og de kommer og feste og bruger klub faciliteterne, og laver grill party, og sågar var der nogle som holdte nytår her i klubben.

D: Ja vi holdte da nytår henne i vores klub, nogle af os.

P: Ja ja, og det er super. En stor del af vores omgangskreds er også bådfolk, som vi har fundet sammen med. Men udover dette så sejler vi ud i den store verden. De fleste af os. Nogle meget nogle lidt og nogle aldrig, men det er meget forskelligt. Det er gæstesejling. Dette sker i alle havne. En stor del af vores indtægtsgrundlag er gæster. Vi har tusindvis af gæster om året. Nogle én dag, nogle mange dage. Det betaler de for.

D: vi ligger også strategisk godt her i limfjorden.

P: Alle der skal igennem; hollændere, tyskere nordmænd og mange flere. Det er en meget vigtigt del af vores indtægtsgrundlag. Det skal man også kunne styre. De kommer og så skal de betale. Der står sådan en automat herude, hvor de kan trække en billet som de sætter på en båd når de har betalt. Der er så facilitet, med bad mm. ligesom på en campingplads. Det bliver nogle steder mere og mere med at man kan booke en plads. Det tror vi aldrig at vi kommer tid at bruge i vores klub, fordi vi fylder havnen op. Men der står i vedtægterne, at når et medlem ikke bruger sin plads, kan klubben leje den ud. Vi har i Danmark et rødt/grønt system. På pladsen nede på broen er der et skildt, hvor grønt betyder ledigt. Så når man kommer ind i en havn kigger man efter grønne skilte og en plads der er stor nok til ens båd. Så lejer man den plads i en periode. Så står der en dato på skiltet, med hjemkomst dato.

Nogle havne dedikerer broer direkte til gæster. Især i sverige. Dette gør at halvdelen af havnen er tom og den anden halvdel er proppet. Sådan gør de i sverige. Men hvis man har den slags gæstebroer, så er det mere og mere normalt at lave online booking. Det har nogle regler og nogle gebyr.

5:32:

D: Grafisk er kun interessant for gæster

P: Ja, det er kun farvelade

Martin: Er der medlemmer i klubben, som ikke kender layoutet, som kunne få hjælp af noget grafisk?

P: tjoo, men der ligger luftphotos og pladsliste på vores hjemmeside, samt vores brochure.

P: Det vores system ikke kan klare, er når det bliver mere advanceret adminstration. Det kræver at man både kan være EDB-mand, og økonomi man, og det tager en masse tid. Alle klubber uden undtagelse døjer med at finde folk der har evner og interesse. Min forgænger holdte et halvt år.

D: Jeg aner ikke noget omkring excel ark, og skal forsøge at bruge det nuværende.

P: Det er meget forskelligt, og det er frivillige i forengninger.

D: jeg er heldigvis kun afløser.

P: Det er et stigende problem, og flere og flere betaler sig fra det. Aalborg Sejlklub har en bogholder de lønner. Vi vil gerne slå os sammen i vores havn, og så lave en over-paraply. Det vil være ét system, som skal kunne holde styr på to klubber. Nogle ting er fælles, andre ting er seperate. f.eks. priser på vandpladser. Vi er en momsregisteret virksomhed, med en millionomsætning. Man ser i stigende grad, sammenslåning af administration på tværs af klubber der deler samme havn. Hvis vi nu f.eks. for et nyt medlem, og ikke har plads til ham, er det fjollet at vi ikke kan ligge ham på den plads lige ved siden af, som tilhører Sejlklubben limfjorden. Sådan kan vi ikke gøre nu.

10:19:

P: gæstesejlere kan ligge overalt, og det kunne være rart hvis vi kunne optimere noget der. Og når i snakker IT, så skal systemet kunne håndtere 2 havne delt på kryds og tværs imellem to klubber. Og det har vi ikke fundet noget system der kan.

P: Det er dyrt at få lavet ting til systemet.

D: Men man bruger også meget tid, som kasser.

P: Og hvis man så bruger tid på at lave hjemmeside og alt sådan noget, det kræver også tid. Det kræver også nogle evner.

D: Jeg administrerer vores fane på hjemmesiden.

P: Hvis vi skal have en ny bro koster det 1 million, 2 hvis det skal være med y-bom. At fjerne den gamle koster en halv, pga. miljøregler. Vi skal overholde nogle regler, hvis vi skal have tilskud fra kommunen.

P: For mange er en båd, den største investering i deres liv. Min båd koster mere end vores ejerlejlighed på havnefronten. Så det betyder noget. Nogle bor i båden, og har adresse her, med postkasser. Nogle klubber tillader ikke det her, men vi kan godt lide at der er folk i havnen.

D: nogle bruger båden som kolonihavehus, og sejler aldrig ud.

P: Vi har 2 ældre damer, hvor deres mænd er døde, og de er her næsten altid, men de sejler næsten ikke.


D: Efter vi har fået betaling til el, er det blevet nemmere at holde styr på. Vi fik installeret målere, så el bliver købt nu. Vi har også en fuldtlønnet havnefodge.

Kasper: Kommer han udefra?

P: Ja, han kommer udefra. Han må slet ikke have båd. Han arbejde 12-14 timer om dagen, om sommeren, og så holder han fri om vinteren. Det er lørdag, søndag, påske og pinse. Lidt som campingfatter på en camping plads. Det er lidt forberedelse før sæson. Havnefodgen er bl.a. service organ, og kan hjælpe til med informationer til turister, og f.eks. motorhjælp.

17:00:

D: Hvis vi har været ude og sejle, og kommer hjem en dag før planlagt, så ringer vi til Per (fogeden), og ber ham om at vende den rød/grønne plade. Hvis der så ligger nogle på pladsen, informerer han dem om at de skal finde en anden plads.

P: Dette sker hele tiden. og det er træls, men det er vilkårne, når man vil være fleksible. Fogeden står også for generel vedligeholdelse, som f.eks. maling og ukrudtfjernelse.


Kasper: Hvordan er fordeling af gæsterindtjeningerne?

P: engang var det efter pladser, men nu kører det efter nogle fordelingstal. I hovedhavnen for VB 60% og SL 40%. I den anden havn har VB 100% pga. suverænt flest pladser. Omkostninger på landpladser og andre faciliteter, bliver også delt. VB har ansat havnefoged, og skriver timesedler med fællesting. Denne løsning virker kun sålænge vi er gode venner.


P: historisk set, har der været lidt krig imellem klubberne. Efter den nye havn er der ikke så meget pladsmangel, men så er der nogle andre ting.


P: Frihavn er i bund og grund en bytning af pladser imellem klubber.


D: lysten til frihavnsordningen ligger bl.a. i en historisk holdning. Det bliver bestemt på generalforsamlinger.


P: I automaten kan man gøre 2 ting. Fylde penge på sit chipkort, eller leje en plads. Hvis man vil leje en plads skal man oplyse størrelsen på sin båd. der er 3 størrelses intervaller, samt et frihavns interval. Via en farvekode kan fogeden se hvor lang tid de har holdt der. hidtil har det været den samme pris hvorvidt man betaler havnefogeden eller i automaten, men det vil vi gerne have lavet om på, sådan at dem der betaler via automaten.


Kasper: er der udlejning af både?

P: Nej. Men der er visse firmaer som har både af representative årsager, men det vurderer vi ikke til at være "kommericelt"


Kasper: Hvordan er det med undervisning?

P: Vi har ikke noget undervisning.

D: Vi har lidt specifikt til sejlsskibe.

P: Nogle af skibene kræver beviser, rent lovgivningsmæssigt. 

Martin: Når der kommer nye medlemmer, tjekker i så om de har beviserne?

P: Det er et krav, men vi følger ikke rigtigt op på det.

D: Sejlbåde kræver ikke beviser.

P: vi kræver en ansvarsforsikring, ikke casco, og vi har en mulighed for en fælles forsikring.

D: Det har vi ikke, det er folks ejet problem.

P: Halvdelen af vores medlemmer har tilmeldt sig den kollektive. Men vi kan ikke holde styr på vores gæster.

Kasper: Ved i om der er nogle i ANF som udlejer både.

P & D: ikke hvad vi ved af.

P: Nogle klubber tillader at man kan udeleje private både fra en klubhavn, men jeg har ikke hørt om en klub der udlejer.

33:00:


Martin: Kan jeres system klare udervisning?

P: Vi har slet ikke undervisning, men Aalborg Sejlklub har undervisning, og der er en uskreven aftale om at de ordner det for sejlklubberne i Aalborg. Der er ikke nok til at understøtte flere sejlhold.


Kasper: Hvor lang tid bruges der på administration?

P: Min kone siger 700 timer på et år på Vestre Baadelaug.


Peter: Det er svært at få besat bestyrelsesmedlemmer. Hos os skal man ikke betale kontingent når man sidder i bestyrelsen - en lille symbolsk betaling. Lovgivningen giver nogle muligheder i form af dækkende betalinger som f.eks. kørsel og telefon. Det kan ikke passe at det skal koste penge at være frivillig, så derfor udlignes udgifterne via specificerede takster.

Kasper: Så det er lysten der er drivkraften? Peter: Ja, hvis lysten ikke er der, skal man ikke gøre det.

Kasper: Omkring forsikringer. Peter: Vi har forsikret bestyrelsen, så hvis f.eks. kranen fejler og skader en person. Da vi har folk ansat, har vi også forsikring der. Også ved frivilligt arbejde i weekender hvor folk kan blive skadet osv. osv.

Kasper: Bare lige for at ridse op, så har i en stor turnaround af medlemmer, korrekt? Peter: Ja, for de (SL) har valgt kun at have sejlbåde, hvor vi (VB) er for flere bådtyper. Det gør at flere aldersgrupper er medlem i VB. I SL er medlemmerne ældre. Der er også forskel i filosofier i klubberne. I VB vægtes det sociale aspekt meget højt. Der er sejlture, fester osv. Mange andre klubber gør stort set ikke sådan noget.

Kasper: Lidt omkring ANF. Samarbejdet mellem klubberne - hvordan er det? Peter: Nogle af de penge fra kontingentbetalinger ryger hen til ANF. Disse penge går til vedligeholdelsen af alle havne tilknyttet ANF. ANF har en aftale med kommunen, den såkaldet brugsaftale. Aalborg kommune har udlejet havnearealerne til ANF som så distribuerer dette til klubberne. Det kører rigtig godt, og har gjort sådan i over 20 år.

Kasper: Deler alle klubber samme holdning? Du nævnte noget klubfjenskab mellem AS. Peter: Klubfjenskab hænger sammen med nogen af de nye folk de har fået. For 30 år siden var der trængsel om pladsen. Kommunen blev overbevist om at bygge en ny havn - Marina Fjordparken.


P(50.00) Kommunen byggede den nye havn fordi de 5 klubber skændtes omkring havne
pladserne. De gik til kommunen som bygge den nye havn og de tilbød at flytte 
derud hvis de selv stod for vedligeholdelse. Aalborg Sejlklub som var størst og ældst
flyttede til havnen. Betingelsen var også at de selv skulle betale for rennoveringen
af de gamle havne.

P(51.30) ANF bruger mange af pengene til rennovation men Aalborg Sejlklub 
udviser ikke interesse for at være med i det fællesskab, trods der mangler
8 millioner kroner i renovationer for at komme på samme niveau som
Aalborg Sejlklub (Der er strid omkring dette).

P(53.10) ANF er unikt for aalborg siden det plejere at være individuelle havne
som har hver deres klub som ikke samarbejder med de nærliggende, ifølge Peter

P(54.00) Klap tilladelse på ANF niveau, gør at de må depositere det jord de graver ud,
hvis da ikke det er forurenet, i en nærliggende havn, som har plads til det.
Dette skulle eftersigende ændre prisen fra 400.- pr kubikmeter til 40.- pr kubikmeter