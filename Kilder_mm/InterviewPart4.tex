\documentclass[a4paper]{article}
\usepackage[utf8]{inputenc}
\begin{document}
Kasper: Indførelse af det nye IT-system, hvornår var det?
Peter: Jeg mener det var i 05'. Jeg blev kasserer i 07. Siden det har vi fået lavet en hel del tilføjelser. F.eks. det der mail-fletning, mail-afsending. Vi fik også mail adresser på alle medlemmer. Før det, sendte vi breve ud til alle medlemmer.

Martin: Ved du hvad i gjorde før 05'?
Peter: Ja, der havde man et eller andet ældgammelt system. Også IT-baseret. Giro-kort blev sendt ud til medlemmerne, og breve blev sendt flere gange om året.

Kasper: Hvordan er IT-systemerne i ANF?
Peter: Det er meget forskelligt. Sejlklubben Limfjorden kører med Excel, og er derved meget bagud. Vi (Vestre bådlaug) snakkede om at gå sammen med SL i stedet for. Problemet i SL er at kasseren ofte er ude at rejse, hvorfor en midlertidig kasser benyttes. Det er svært at finde nye kasserer. Min forgænger holdt kun et halvt år; det er et fuldtidsjob. Det er tidskrævende og der er masser af deadlines, f.eks. NETS til betalingsservice. Man skal være temmelig skrap til f.eks. NEM-ID, hjemmesider. Det hører også sammen med at det er en relativ stor forening med en relativ stor omsætning.

Kasper: Registreringer af diverse ting; har i nogle registreringssystemer som ikke er inkluderet i diverse IT-systemer.
Peter: Ja, udover navision har vi et system til håndtering af nøglekort. Alle medlemmer skal have mindst 1 kort med en chip. Dette er deres medlemskort. Det virker til døre, bad osv.

Kasper: Ville det være smart hvis dette system blev inkluderet i medlemssystemet?
Peter: Ja, det ville være smart at lægge systemerne sammen, men det er meget bøvlet, fordi nøglesystemet er meget integreret med boksen der holder styr på nøglerne.Her er der redundans. Dvs. her er der også et medlemskartotek. Blokering af kort osv. 

Kasper: Hvis disse problemer ikke havde været til stede, havde i så valgt at sammenlægge systemerne?
Peter: Ja, det er klart, redundans information er fandens værk.

Martin: Hvis man kunne optimere noget af den tid der bruges på indtastning af medlemmer, f.eks. ved at de selv kunne indtaste noget. Er der noget du (Peter) bruger meget lang tid på?
Peter: Når vi skal have informationer fra medlemmer, kommer de typisk ind fra mails. Så skal de skannes, og så skal man se om der er sket rettelser i forhold til gamle information. Hvis dette er tilfældet, skal dette bogføres i systemet. Jeg ville aldre få medlemmerne til selv at indtaste dette. Skemaerne fra medlemmerne skal arkiveres. De bliver smidt ind i en mappe på computeren. Man skal også se om der er ønsker til bådpladser. Der ligger en helvedes masse arbejde. Nogle pladser skal flyttes. Det er et stort arbejde.

Minut 10

Kasper: Hvis brugerne kunne indberette i nogle formularer, som systemet kunne bogføre nemmere..
Peter: Nej.. Et andet aspekt er kontingentbetalinger. Man kan fra NETS hver dag downloade hvilke betalinger der er foretaget. Dette kan man automatiske opdatere, men det har jeg valgt fra. Så har jeg bedre styr på hvem der har betalt. Det kan man dog sagtens lave automatisk. Man skal dog holde styr på special cases. Til gengæld hvis der havde været 4000+ medlemmer, havde det været relevant.



Peter: Hvis et medlem får en større båd, skal dette medlem selvfølgelig også have en større plads. Nye medlemmer der kommer ind, skal betale kontingent. Det sker via indbetalingskort, uden brug af NETS. Der er altså ikke meget tid at hente ved indtastning.

Kasper: Er der nogle udefrakommende lovmæssige krav til hvad der skal registreres?
Peter: Der er det skattemæssige, vi skal ikke betale skat. Til gengæld skal vi registrere MOMS, digital postkasse, NEM-ID. Registreing til forskudsberetning, registrering til Aalborg kommune. Dette giver et tilskud på 200.000 kr. om året. Dette er i øvrigt alt for meget, efter min mening. Derudover kommer medlemskab af DSU osv. Der er ingen der siger at man skal bogføre med et IT-system. I princippet kunne man godt have pengene i en cigarkasse. Registreringer er altså ikke det værste.

Kasper: Hvem er i kontakt med IT-systemet?
Peter: Det er mig. Der er flere der bruger det, men det er mig der styrer det. Jeg sørger for backup og administrationen af systemet.

Martin: Hvem bruger det ellers?
Peter: I teorien burde alle bruge det, men i praksis bruger sekretæren det, og vores aktivitetsleder. Derudover fartøjsinspektøren. Næstformanden og formanden bruger det ikke. Det varierer på hvem der sidder i bestyrelsen. Vi bruger Dropbox i stor stil til deling af dokumenter. Jeg kunne godt ønske mig at nogen var mere med på beatet, teknologisk set. Et problem var at der ikke blev lavet backup i et helt år. Det kunne have gået meget galt. Det er altså vigtigt at der er en IT-administrator.

Kasper: Omkring platforme, du nævner at du har adgang fra klubhuset og hjemmefra. Er der andre steder i har adgang, f.eks. telefoner, iPad's.
Peter: Det kan man lave, men det har vi ikke.

Kasper: Er der interesse i et system med fjernadgangsmuligheder? Nu nævner du at der er en del unge. Kunne det være smart hvis de kunne se oplysninger omkring deres båd f.eks.?
Peter: Nej..
Martin: Der er heller ikke noget medlemmer skal indmelde løbende?
Peter: Nej ikke løbende, kun ændringer som f.eks. nyt telefonnummer eller email. Ellers 1 gang i året ved plads tildeling og opdatering af informationer. Hvis medlemmerne betaler de 2 regninger om året, og de ikke flytter, skal der ikke registreres nogle informationer.

Kasper: Hvilke udgifter har i til det nuværende IT-system?
Peter: Det sku' billigt. Altså vi har jo super internetforbindelse, den koster selvfølgelig lidt. Derudover den nøgleautomat. Derudover er der software-licens til Navision, office har vi købt. Navision licensen koster omkring 5000 kr/året. Så er der noget omkring hjemmeside hosting, men det er småtingsafdelingen. Så er der diverse nye printere osv.

Lige omkring minut 20

Kasper: Kan du opsummere nogle af de punkter, hvor man kunne ønske sig forbedringer?
Peter: Ved vores nuværende system, kan man ikke lave en regning og sende den via mail. Men det skal bare laves, der er ingen problemer i det. Det kan være vi alligevel skal have et nyt system. Et andet godt system til bogføring er e-conomic.dk. Det er helt vildt godt - cloud baseret og det hele. Det er dog ikke nødvendigt at det er cloudbaseret for vores behov. Jeg har alligevel remote-desktop adgang.

Martin: Havnefogeden, ved du hvordan han holder øje med hvorvidt folk er ude eller hjemme?
Peter: Han er der stort set i døgndrift i sæsonen og han kender efterhånden gud og hver mand. Der er ikke noget formelt system som sådan.
Martin: Det var måske en mulighed. Du talte om at andre havne har et system, hvor gæster kan se om folk er ude og/eller hvornår de kommer hjem.
Peter: Nej, ikke nødvendigvis. Det er fordi de har dedikeret gæstepladser, så de har booking af gæstepladser. Det kan vi ikke gøre, fordi vi ikke har dedikerede gæstepladser. I skal huske på at vi i sejlbranchen er meget afhængige af vejret. F.eks. regner man med at man skal sejle hjem søndag, men man finder ud af at der er stormvarsel. Man ligger nu på Anholt. Det gælder altså om at skynde sig hjem inden stormen kommer. Man laver altså ofte om i planen. Man kan altså ikke langtidsplanlægge. Vores system fungerere så ved at man kan melde havnefogeden hvornår man regner med at være hjemme igen. I tilfælde af at man skal før hjem, må man ringe til havnefogeden. Havnefogeden holder styr på alle disse informationer på sin egen måde ved en kuvert eller andet.

Martin: Det kan godt være det er fint for ham, men en ny havnefoged, vil have hjælp af en bedre organisation. Desuden vil gæster, hvis det var cloud-baseret kunne se hvor der var pladser.
Peter: Jo, men så skal dette system opdateres hele tiden når folk en plads er fri.Vi har 2 havne, over 400 havne. Om eftermiddagen kan der ske at der kommer op til 25 sejlbåde på en gang. Der er så kaos om at finde en plads. Gæsterne har mange præferencer i forhold til hvordan de vil ligge i forhold til vinden osv. osv. osv. Det er ikke som en campingplads, hvor man holder udenfor porten, og går ind til campingfatter og spørger om en plads. Så kigger han på kortet, og ser at en specifik plads er ledig. Andre steder kan man selv vælge mellem ledige pladser. Ved valg af plads, sætter campingfatter et kryds, og spørger om hvor lang tid man regner med at campere der. Når man tager hjem, går man ind til campingfatter og betaler. Så kan han registrere at pladsen er ledig. Det kan vi ikke gøre i sejlklubber. Nogen gange kommer de ind om natten.

Martin: Hvordan gør man med betaling når gæsterne ankommer?
Peter: Normalt skal gæsterne gå op til en automat, men ellers finder havnefogeden ud af at de er ankommet den næste morgen.

Peter: Om sommeren kan der være 100 gæster på en dag, og så kan man ikke styr det mere.
Martin: Du siger at de skal betale for pladsen i automaten. Kunne man ikke få gæsterne til at indtaste informationer ved pladskøb? 
Peter: Det kræver at de kender hvilken plads de holder på. Der er både hollændere, polakkere, svenskere osv. Alle mulige sprog, og derfor er registreringen for omfattende. Ofte er fejlagtig information værre end ingen information. Den store forskel på en sejlklub og en campingplads, er at vi ikke holder styr på hvem der kommer og går - det må vi ikke. Det er en offentlig havn. I tilfælde af nødstilfælde, skal man kunne komme ind i havnen.

Peter: Jeg kan ikke se hvordan man kan lave et smart system. Man er nødt til at indse at ikke alle folk er lige snue. 






\end{document}
